\documentclass[a4paper,12pt]{article}
\usepackage{amsmath,amssymb,amsfonts}
\usepackage{mathtext}
\usepackage[english,russian]{babel}
\usepackage[utf8]{inputenc}
\usepackage[T2A]{fontenc}
\usepackage{graphicx}
\usepackage{textcomp}
\usepackage{geometry}
\geometry{left=3cm}
\geometry{right=1.5cm}
\geometry{top=2cm}
\geometry{bottom=2cm}
\usepackage{tikz}
\usepackage{titling}
\usepackage{indentfirst}
\setlength{\parindent}{1cm}
\usepackage{soul}
\usepackage{enumitem}
\usepackage{listings}
\usepackage{fvextra}
\usepackage{tabularx}
\usepackage{fancyhdr}
\usepackage{setspace}
\usepackage{tikz}
\usepackage{tikz-cd}
\usepackage{ifthen} % provides \isempty test
\onehalfspacing
\makeatletter
\AddEnumerateCounter{\asbuk}{\russian@alph}{щ}
\makeatother
\fvset{breaklines=true, breakafter=\space}

\DeclareRobustCommand{\divby}{%
	\mathrel{\vbox{\baselineskip.65ex\lineskiplimit0pt\hbox{.}\hbox{.}\hbox{.}}}%
}

\newcommand{\lection}[1]{\section*{#1}
	\addtocounter{section}{1}
	\addcontentsline{toc}{section}{#1}
	\setcounter{subsection}{0}
}


\newcommand{\defn}{\stepcounter{paragraph}\paragraph{Определение \arabic{section}.\arabic{subsection}.\arabic{paragraph}}
}


\newcounter{theorem}
\newcommand{\theorem}[2][]{\stepcounter{theorem}\paragraph{Теорема \arabic{section}.\arabic{subsection}.\arabic{theorem}} \ifthenelse{\equal{#1}{}}{}{(#1)} {\itshape #2 }
}

\fancypagestyle{titlepage}{
	\fancyhf{
		\begin{center}
			\begin{tabularx}{\textwidth}[c]{c p{12cm}}
				\raisebox{-1.1\totalheight}{\includegraphics[width=0.15\linewidth]{./logo/bauman_logo.png}}
				&\begin{center}
					\textbf{ Министерство науки и высшего образования Российской Федерации \\
						Федеральное государственное бюджетное образовательное учреждение 
						высшего образования \\
						«Московский государственный технический университет
						имени Н.Э. Баумана \\
						(национальный исследовательский университет)» \\ (МГТУ им. Н.Э. Баумана)} \\
					
				\end{center}
			\end{tabularx}
	\end{center}}
	\fancyfoot[C]{Москва 2024 г.}
	\renewcommand{\headrulewidth}{0pt}
}

\title{~\\~\\~\\~\\~\\~\\~\\~\\~\\Криптографические методы защиты информации}
\author{Конспект лекций}
\date{МГТУ им. Н.Э. Баумана}

\begin{document}
	\maketitle
	\thispagestyle{titlepage}
	\pagebreak
	\tableofcontents
	
\section*{Disclaimer}
\addcontentsline{toc}{section}{Disclaimer}
Конспект создан студентами для подготовки к экзамену ввиду отсутствия удобного формата лекций. Поэтому он может содержать ошибки, опечатки и многое другое, за что можно получить автомат с сапогами в придачу.

\textit{(Тим Кравченко aka Fe-Ti)}

\pagebreak
\lection{Лекция 1: Очередное введение в криптографию}

\subsection{Криптография}
	\hl{начало лекции?}\\
	
	Основными задачами криптографии являются обеспечение:
	
	\begin{itemize}
		\item конфиденциальности --- защиты от ознакомления с содержанием сообщения лицами, не обладающими правами доступа\footnote{При этом не скрывается факт передачи сообщения и зашифрованное сообщение передаётся по открытому каналу связи.};
		\item целостности --- \hl{заполнить} ;
		\item аутентификации --- доказательного подтверждения подлинности сторон и передаваемой информации в процессе информационного взаимодействия.
	\end{itemize}
	
	
	Криптография условно делится на:
	\begin{itemize}
		\item криптосинтез,
		\item криптоанализ.
	\end{itemize}



\subsection{Отображения: Флешбэк}

Пусть $X = {x_1, ..., x_n}, Y = {y_1, ..., y_n}$ --- конечные множества из $n$ элементов.
\defn
Отображением (функцией) $f : X \rightarrow Y$ называется сопоставление, которое каждому элементу из $X$ ставит в соответствие один элемент из $Y$.
\defn
Если каждый из элементов $Y$ соответствует хотя бы одному элементу из $X$, то отображение называется сюръективным.
\defn
Если каждый из элементов $Y$ соответствует не более, чем одному элементу из $X$, то отображение называется инъективным.
\defn
Если у всех элементов $Y$ есть ровно один прообраз из $X$, то отображение называется биективным (т.е. одновременно инъективным и сюръективным, см. таблицу ниже).

\begin{table}
	\centering
	\label{table_mapping_types}
	\caption{\raggedright{Свойства отображений (функций)}}
	\begin{tabular}{|c|cc|}
		\hline	& инъективное & не инъективное \\
		\hline
		сюръективное 	& биекция 	& сюръекция \\
		не сюръектиное	& инъекция	& общий случай \\
		\hline 
	\end{tabular}\\
\end{table}

\paragraph{Пример}

$$
\begin{matrix}
f : X \rightarrow Y \\
X = \{0,1,2,3,4\}, ~ Y = \{a,b,c,d,e\}
\end{matrix}
$$

Пример биективного отображения:
$$
\begin{matrix}
	x & 0 & 1 & 2 & 3 & 4 \\
	f(x) & a & e & d & c & b \\
\end{matrix}
$$

Пример отображения, которое не является биективным:
$$
\begin{matrix}
	x & 0 & 1 & 2 & 3 & 4 \\
	f(x) & \mathbf{a} & e & d & \mathbf{a} & b \\
\end{matrix}
$$

\subsection{Подстановки: Флешбэк №2}
\defn
Биективное отображение $f: X \rightarrow X$ называется подстановкой.

\paragraph{Пример}
$$
\begin{matrix}
	x & 0 & 1 & 2 & 3 & 4 \\
	f(x) & 3 & 2 & 1 & 4 & 0 \\
\end{matrix}
$$

\subsection{Симметричная шифрсистема}

Набор $(X,Y,K,e,d)$ --- симметричная шифрсистема, где:
\begin{itemize}
	\item $X$ --- множество открытых текстов;
	\item $Y$ --- множество шифртекстов;
	\item $K$ --- множество ключей;
	\item $e(x, k)$ --- функция зашифрования;
	\item $d(y, k)$ --- функция расшифрования;
	\item $e_k(x)$ --- подстановка, заданная ключом $k$ (функция зашифрования на ключе $k$);
	\item $d_k(y) = e_k^{-1}(y)$ --- подстановка обратная к $e_k$ (функция расшифрования на ключе $k$).
\end{itemize}

Очевидно: получатель и отправитель используют один и тот же ключ.

\subsection{Шифр сдвига}

Зашифрование сообщения происходит применения подстановки в виде сдвига алфавита на $k$ позиций. Математически это можно выразить следующим как сумму по модулю:
$$X = Y = \mathbb{Z}_n = \{0, ..., n-1\}, k,x,y \in \mathbb{Z}_n
$$
$$
\begin{aligned}
	e_k(x) &= x + k \mod n \\
	d_k(y) &= y - k \mod n
\end{aligned}
$$

Если $k = 3$, то шифр принято называть шифром Цезаря.

\subsection{Теоремы Эйлера и Ферма}

\theorem[Эйлера]{
	Пусть $m \in \mathbb{N}, ~ m > 1$. Тогда $$\forall b \in \mathbb{Z}, НОД(b,m) = 1 ~~~ \Rightarrow ~~~ b^{\phi(m)} \equiv 1 \mod m.$$
}

Замечание: если $p$ --- простое число, то функция Эйлера $\phi(p) = p - 1$. Отсюда следует малая теорема Ферма.

\theorem[Ферма]{Пусть p --- простое число. Тогда $$\forall b \in \mathbb{Z}, НОД(b,p) = 1 ~~~ \Rightarrow ~~~ b^{p-1} \equiv 1 \mod p.$$}

\subsection{Шифры простой замены}
\defn
Шифром простой замены называется такой шифр, у которого в качестве функции зашифрования выступает подстановка на множестве блоков открытого текста, т.е. он состоит из:

\begin{itemize}
	\item алфавита открытого текста $X = \{x_1, ..., x_n\}$;
	\item алфавита шифртекста $Y = \{y_1, ..., y_n\}$ (обычно $X = Y$);
	\item биективного отображения $s: X \rightarrow Y$ --- ключа (и одновременно функции зашифрования).
\end{itemize}

Расшифрование производится с помощью обратной подстановки $s^{-1}$.


\subsection{Дешифрование}

\defn
Дешифрование --- процесс аналитического раскрытия противником открытого сообщения без предварительного полного знания всех элементов шифрсистемы.

\defn
Метод полного опробования ключей --- метод криптоанализа, состоящий в переборе всех возможных ключей с отбраковкой ложных вариантов по некоторому критерию. \\

Наихудшие сложности перебора "<древних"> шифров ($n$ --- мощность алфавита открытых текстов):
\begin{itemize}
	\item для шифров сдвига --- $O(n)$;
	\item для шифров простой замены --- $O(|S_n|) = O(n!)$.
\end{itemize}


\pagebreak
\lection{Лекция 2: Шифры простой замены}

\subsection{Шифр простой замены: Криптоанализ}
{\itshape \footnotesize One Substitution to rule them all, One Substitution to find them, One Substitution to bring them all and in the darkness bind them. (c)} 

Основная проблема ШПЗ --- использование одной и той же подстановки на всём сообщении. Отсюда следует, что частоты исходных символов сохраняются в шифртексте. Из-за этого ШПЗ уязвимы перед частотным анализом:
\begin{enumerate}
	\item подсчитываем частоты в шифртексте;
	\item сравниваем полученное с референсной таблицей частот получая "<предварительный"> ключ;
	\item пытаемся получить осмысленный текст, корректируя ключ.
\end{enumerate}


\subsection{Шифр Виженера (XVI век)}
Шифр Виженера выглядит следующим образом:
\begin{itemize}
	\item $\mathbb{Z}_n, ~ n \ge 2 \in \mathbb{N}$ --- алфавит открытого и закрытого текстов, а также ключа;
	\item $x = (x_0, x_1, ..., x_m, ..., x_{l-1})$ --- открытый текст (длины $l$);
	\item $k = (k_0, ..., k_{m-1}) \in \mathbb{Z}_n^m$ --- ключ (длины $m$);
	\item $e_k(x_{jm},...,x_{jm+m-1}) = (x_{jm}+k_0, ..., x_{jm + m - 1}+k_{m-1}) \mod n$ --- функция зашифрования ($j=0,1,...$);
\end{itemize}

Идея этого шифра в использовании $m$ сдвиговых шифров вместо одного, которые образуют ключ дины $m$. Первый символ открытого текста шифруется первым сдвигом, второй символ --- вторым сдвигом, и так далее.   

$$
\begin{matrix}
	X & x_0& x_1& ...& x_{m-1}& x_m& x_{m+1}& ...& x_l\\
	K & k_0& k_1& ...& k_{m-1}& k_0& k_1    & ...& k_{l\mod m}
\end{matrix}
$$

Криптоанализ шифра Виженера состоит из двух этапов: поиска длины ключа и взлома сдвиговых шифров.

\subsubsection{Шифр Вернама (начало XX века)}
Шифр Вернама представляет собой шифр Виженера с длиной ключа равной длине открытого текста. В отечественной литературе ключ называется гаммой.

\theorem{Если гамма случайна и равновероятна, то шифр Вернама является теоретически стойким\footnote{Т.е. дешифровать можно только полным перебором.}.}


\subsection{Квантовые компьютеры}
Квантовый компьютер --- вычислительное устройство, которое использует явления квантовой мехоники для передачи и обработки информации. КК оперирует кубитами --- ячейками принимающими значения 0 и 1 одновременно с вероятностью.\\

На КК за полиномиальное время можно решить задачи:
\begin{itemize}
	\item факторизации числа;
	\item вычисления дискретного логарифма.
\end{itemize}


\pagebreak
\lection{Лекция 3}
\subsection{Открытые тексты}
\defn
Открытым текстом называется последовательность символов некоторого алфавита.

\defn
Последовательность $r \ge 2$ соседних букв текста называется $r$-граммой.\\

\begin{tikzcd}
	& \text{Модели открытых текстов} \drar \dlar & \\
	\text{детерминированные} & & \text{вероятностные}
\end{tikzcd}\\

\subsection{Детерминированные модели открытых текстов}

Источник открытого текста (ИОС) --- некий субъект (человек, группа людей, радиостанция и т.п.), порождающий открытые тексты --- характеризуется:
\begin{itemize}
	\item одним или несколькими языками общения;
	\item набором используемых алфавитов;
	\item тематикой генерируемых сообщений;
	\item частотными характеристиками сообщений.
\end{itemize}

Правила грамматики языка влияют на статистические характеристики сообщения. В частности ИОС характеризуется разбиением множества $r$-грамм на допустимые и запрещённые. Это разбиение задаёт детерминированную модель ИОС.

\paragraph{Пример}
В русском языке знаки Ь и Ъ никогда не следуют за гласными\footnote{ЪУЪ не в счёт}, т.е. запрещены биграммы: аь, аъ, еь, ..., юъ, яь, яъ.

\defn
Открытым текстом в детерминированной модели называется последовательность символов некоторого алфавита, которая не содержит запрещённых $r$-грамм.

\subsection{Вероятностные модели открытых текстов}
\defn
Источник открытого текста (ИОС) с точки зрения вероятностных моделей --- это источник случайных последовательностей.\\ 

Пусть ИОС генерирует конечный или бесконечный текст в алфавите $\mathcal{A} = \mathbb{Z}_m$. Тогда вероятностная модель такова:
\begin{itemize}
	\item ИОС генерирует последовательность дискретных случайных величин $\xi_1, ..., \xi_t, ...$ над $\mathbb{Z}_m$
	\item Вероятность случайного сообщения ($a_1, ..., a_t$) есть:
	$$
	p(a_1, ... , a_t) = P\{\xi_1 = a_1, ... , \xi_t = a_t\}.
	$$
\end{itemize}

Множество случайных сообщений образуют вероятностное пространство, если:
\begin{itemize}
	\item $\forall (a_1, ... , a_t) \in \mathbb{Z}_m^t: ~ p(a_1, ... , a_t) \ge 0$
	\item $\sum_{(a_1, ... , a_t) \in \mathbb{Z}_m^t} p(a_1, ... , a_t) = 1$
	\item  $\forall (a_1, ... , a_l) \in \mathbb{Z}_m^t, ~ l>t \in \mathbb{N}: ~ p(a_1, ... , a_t) = \sum_{(a_1, ... , a_t, ..., a_l) \in \mathbb{Z}_m^t} p(a_1, ..., a_t, ..., a_l) $\\
\end{itemize}

С точки зрения вероятностной модели, текст, порождаемый источником является вероятностным аналогом языка и такими же частотными характеристиками, как у него.

Задавая вероятностное распределение на множестве открытых текстов, мы получаем модель ИОС.

\subsection{Стационарный источник независимых символов алфавита (СИНСА)}

СИНСА является простейшей вероятностной моделью: вероятность сообщений полностью определяется вероятностями букв в случайном тексте.\\

Пусть алфавит открытого текста $\mathcal{A} = \mathbb{Z}_m$, $\xi$ --- дискретная случайная величина, $p(a) = P\{\xi = a\} \in [0,1]$ --- вероятность появления символа $a \in \mathbb{Z}_m$, при этом $\sum_{a} p(a) = 1$.

Тогда вероятность появления текста $a_1, ..., a_t$ длины $t$ равна:
$$
p(a_1, ..., a_t) = \prod_{i=1}^{t} p(a_i).
$$

\defn
Открытый текст является реализацией последовательности независимых испытаний в полиномиальной вероятностной схеме с числом исходов $m$.\\

Недостаток этой модели состоит в том, что: $$\forall (a_1, ..., a_t)\in \mathbb{Z}_m^t, t > 1: ~ p(a_1,...,a_t) > 0.$$ Этот факт противоречит наличию у языка запрещённых $t$-грамм.

\subsection{Стационарный источник независимых $t$-грамм}

\defn
Открытым текстом является реализация последовательности независимых испытаний в полиномиальной вероятностной схеме с числом исходов $m^2$. \hl{(походу это определение дано для 2-грамм)}

Задается исходное распределение случайной дискретной величины на множестве $t$-грамм (из букв алфавита $\mathcal{A} = \mathbb{Z}_m$):
$$
\begin{aligned}
	\overline{a} ~ =& ~(a_1, ..., a_t) \in \mathbb{Z}_m^t \\
	p(\overline{a}) ~ =& ~P\{\xi = (a_1, ..., a_t)\} \\
	&0 \le p(\overline{a}) \le 1,~ t \ge 2 \\
	&\sum_{\overline{a} \in \mathcal{A}^t}p(\overline{a}) = 1. 
\end{aligned}
$$

Пусть длина открытого текста равна $l = t \cdot s$, где $s$ --- количество $t$-грамм. Открытый текст выглядит как последовательность $(\overline{a}_{1}, ...,\overline{a}_{s})$, где $\overline{a}_{i} = (a_{i,1},..., a_{i,t}) \in \mathcal{A}^t.$

Вероятность появления определённого текста есть:
$$
p(\overline{a}_{1}, ...,\overline{a}_{s}) = \prod_{i = 1}^{s} p(\overline{a}_i).
$$

Модель СИНт точнее, чем СИНСА, но ни в одной из них не учитывается зависимости между $t$-граммами, которые обычно присутствуют в языке.

\subsection{Стационарный источник марковски зависимых букв (СИМЗБ)}

\defn
Открытый текст $(a_1,...,a_l)$ является реализацией простой однородной цепи Маркова с $m$ состояниями.

Вероятность появления определённого текста есть\footnote{В лекции начальное значение $j=1$, что похоже на опечатку.}:
$$
p(a_1, ..., a_l) = p(a_1) \prod_{j=2}^{l}p(a_j | a_{j-1}),
$$
где:
\begin{itemize}
	\item $p(a_1)$ --- вероятность первой буквы открытого текста;
	\item $p(a_j|a_{j-1})$ --- вероятность того, что на $j$-й позиции будет $a_j$, при условии, что на предыдущей позиции находится $a_{j-1}$.
\end{itemize}

Модель задаётся начальным вектором вероятностей $\overline{p}_0 = (p(0), ..., p(m-1))$ и матрицей переходных вероятностей $P = (p(u|v))$.

\paragraph{Утв.} Вероятность любого сообщения, содержащего запретную биграмму, равна нулю.\\

Вектор стационарного распределения $\overline{\beta} = (\beta_0, ..., \beta_{m-1})$ находится из системы уравнений:
$$
\begin{aligned}
	\beta_i = \sum_{j=0}^{m-1} \beta_j p(i|j),~~ i &= 0, ..., m-1,\\
	\beta_0 + ... + \beta_{m-1} &= 1.
\end{aligned}
$$

\paragraph{Задача}
$$
\begin{aligned}
	m &= 4, \\
	\mathcal{A} &= {0,1,2,3} \\
\end{aligned} ~~~~~
\begin{tabular}{|c|c|c|c|c|}
	\hline
	~	&0	&1	& 2	& 3	\\
	\hline
	0	&0   &0.25&0.60&0.15\\
	\hline
	1	&0.70&0.10&0.05&0.15\\
	\hline
	2	&0.15&0.35&0.20&0.30\\
	\hline
	3	&0.15&0.20&0.30&0.35\\
	\hline
\end{tabular}$$

Найти стационарное распределение.

\subsection{Критерии на открытый текст}
Статистические критерии являются частью криптографических методов анализа. Они связаны с опробованием ключей и отсевом ложных вариантов на основе того, является ли расшифрованный текст чем-то осмысленным или случайной последовательностью знаков.

Пусть дана последовательность в алфавите $\mathcal{A}$:
$$
\overline{a} = a_1, a_2, ..., a_l.
$$
Необходимо различить две простые гипотезы:
\begin{itemize}
	\item $H_0$ --- последовательность $\overline{a}$ является открытым текстом в заданной модели;
	\item $H_1$ --- последовательность $\overline{a}$ является случайной равновероятной последовательностью над алфавитом $\mathcal{A}$.
\end{itemize}

Важнейшие параметры любого статистического критерия --- это вероятности ошибок первого и второго рода:
\begin{itemize}
	\item $\alpha = P\{H_1|H_0\}$ --- открытый текст принят за случайный набор знаков;
	\item $\beta = P\{H_0|H_1\}$ --- случайный набор знаков принят за открытый текст.
\end{itemize}

Также важны следующие параметры: среднее число знаков, необходимых для отбраковки, и длина шифртекста, необходимая для отсева ложных ключей.


\pagebreak
\lection{Лекция 4}

\subsection{Критерии на открытый текст (продолжение)}

\paragraph{Пример}
Критерий на основе запретных биграмм.
Пусть среди $m^2$ всевозможных биграмм $\nu$ являются запретными. Открытым текстом будем считать последовательность чётной длины $l=2t$, состоящая из $t$ биграмм и не содержащая запретные биграммы.

Вероятности ошибок первого и второго рода, соответственно:
$$
\begin{aligned}
	\alpha &= P\{H_1|H_0\} = 0, \\
	\beta &=  P\{H_0|H_1\} = (1-\frac{\nu}{m^2})^t.
\end{aligned}$$

Задача --- оценить $\beta$, применив аппроксимацию $(1+x)^n \approx e^{nx}$. \\

Приближённая оценка равна
$$
\beta = (1-\frac{\nu}{m^2})^t \approx e^{-\tau t},
$$
где $\tau = \frac{\nu}{m^2}, ~ \tau \ll 1. $

Тогда средняя длина случайного текста, на котором срабатывает критерий:
$$
E\xi = \sum_{j=1}^{\infty}(1-\tau)^{j-1} \cdot \tau \cdot j = \tau^{-1}.
$$

Если критерий применяется независимо (к различным участкам текста) $r$ раз, то вероятность ошибки второго рода равна
$$
\beta^r \approx e^{-\tau t r}.
$$

Пусть $K$ --- ключевое множество. Тогда для приближённой оценки длины материала $d = t \cdot r$, при котором будут отсеяны все ложные ключи, можно применить равенство
$$
|K| = \beta^{-r} \approx e^{r \cdot d}.
$$

Тогда
$$
d \approx \tau^{-1} \ln|K| = \frac{m^2}{\nu} \cdot \ln |K|.
$$

\paragraph{Задача}
Оценить вероятность ошибки второго рода $\beta$ и длину текста $d$ (в биграммах) при известной $|K|$ для английского языка и русского языка (с сокращённым алфавитом из 31 буквы: "<е">="<ё"> и "<ъ">="<ь">).

\paragraph{Решение}
Для английского\footnote{Честно говоря, $\nu$ я не проверял. ТК}:
$$
\begin{aligned}
	m &= 26 \\
	\nu &= 250 \\
	\tau &= \frac{\nu}{m^2} = \frac{250}{676} \approx 0.37 \\
	\beta &= e^{-0.37t}, ~ d = 2.7 \ln|K|.
\end{aligned}
$$
Для русского:
$$
\begin{aligned}
	m &= 31 \\
	\nu &= 382 \\
	\tau &= \frac{\nu}{m^2} = \frac{382}{961} \approx 0.4 \\
	\beta &= e^{-0.4t}, ~ d = 2.5 \ln|K|.
\end{aligned}
$$


\subsection{Сбалансированные отображения}
Пусть $X, ~Y$ --- конечные множества.

\defn
Отображение $\phi: X \rightarrow Y$ называется сбалансированным, если у всех $y \in Y$ число прообразов равно, т.е. $$\forall y \in Y: ~|\phi^{-1} (y)| = |X|^{-1} \cdot |Y|.$$

\paragraph{Утв.} Для сбалансированности отображения необходимо, чтобы $|Y| \divby |X|$.

\paragraph{Утв.} Если $|X| = |Y|$, то сбалансированность $\phi$ равносильна биективности.

\paragraph{Утв.} Сбалансированность шифрующих отображений характеризует статистические свойства криптографических оторажений.

\paragraph{Утв.} Биективность шифрующего отображения необходима для однозначного восстановления исходного сообщения.

\subsection{Координатные функции отображения}
Пусть $P_{n,m} (X) = \{\mu|~\mu:X^n \rightarrow X^m\}$.

\paragraph{Утв.}
Каждое отображение $\phi \in P_{n,m}(X)$ однозначно задаётся набором функций
$$
\phi = (\phi_1, ..., \phi_m), ~\phi_i \in P_{n,1}(X), ~i=1,...,m
$$ 

\defn
$\phi_i: X^n \rightarrow X$ называется $i$-й координатной функцией отображения $\phi$.

\paragraph{Пример}
$$
\begin{aligned}
	q & = 2 \\
	n & = 3 \\
	m & = 2 \\
	X &=\mathbb{F}_2= \{0,1\}
\end{aligned}
~~~~~~~~
\begin{aligned}
	\phi&: \mathbb{F}_2^3 \rightarrow \mathbb{F}_2^2 \\
	\phi&= \begin{pmatrix}
		0&1&2&3&4&5&6&7 \\
		3&0&2&1&1&3&0&2
	\end{pmatrix} =\\
   &= \begin{pmatrix}
	000&001&010&011&100&101&110&111 \\
	11&00&10&01&01&11&00&10
\end{pmatrix}
\end{aligned}
$$
$$
\begin{matrix}
	\vec{x} & \phi_1 & \phi_2 \\
	000 & 1 & 1 \\
	001 & 0 & 0 \\
	010 & 1 & 0 \\
	011 & 0 & 1 \\
	100 & 0 & 1 \\
	101 & 1 & 1 \\
	110 & 0 & 0 \\
	111 & 1 & 0 
\end{matrix}
$$

\subsection{Критерии сбалансированности отображений}
В общем случае выявление сбалансированности отображения является сложной задачей даже при небольших $n$. Однако существуют достаточные условия несбалансированности, которые удобны для применения на практике.\\

\hl{???}

\paragraph{Следствие 2}
Если отображение $\phi: \mathbb{Z}_2^n \rightarrow \mathbb{Z}_2^m$ сбалансировано, то все его координатные функции тоже сбалансированы.


\paragraph{Следствие 3}
Пусть $X$ --- поле. Отображение $\phi = (\phi_1, ..., \phi_m) \in P_{n,m}(X)$ сбалансировано тогда и только тогда, когда сбалансирована каждая нетривиальная линейная комбинация его координатных функций, т.е.:
$$
\forall (a_1, ..., a_m) \in X^m \backslash {\textbf{0}}: ~a_1\phi_1 + ... + a_m \phi_m ~ \text{ является сбалансированной.}
$$

\paragraph{Задача 1}
Сбалансировано ли отображение $\phi: \mathbb{Z}_2^n \rightarrow \mathbb{Z}_2^n$?
\begin{itemize}
	\item $n=3, ~~\phi=(x_1x_2 + x_3, ~ x_2x_3 + x_1, ~x_1x_3 + x_2)$
	\item $n=8, ~~\phi = (x_1 + x_2, ~ x_2+x_3, ~..., ~ x_7+x_8, ~x_8 + x_1)$
\end{itemize}

\paragraph{Задача 2} Имеет ли нормальное весовое строение $\phi: \mathbb{Z}_2^6 \rightarrow \mathbb{Z}_2^6$:
$$
\phi = (x_1, ~ x_2+x_1, ~ x_3 + x_2,~..., ~ x_6 + x_5)
$$

\paragraph{Задача 3}
Пусть $\mu: \mathbb{Z}_2^{n-1} \rightarrow \mathbb{Z}_2$. Для каждого набора $(a_1,...,a_n) \in \mathbb{Z}_2^n$ найти вес булевой функции\footnote{Ответ: 1, если верить слайдам.}:
$$
(\mu(x_2,...,x_n)+x_1 + a_1)\cdot(x_2 + a_2)\cdot...\cdot(x_n+a_n).
$$
	
\pagebreak
\lection{Лекция 5}
\defn

	
\pagebreak
\lection{Лекция 6}

	
\pagebreak
\lection{Лекция 7}

	
\pagebreak
\lection{Лекция 8}

	
\pagebreak
\lection{Лекция 9}

	
\pagebreak
\lection{Лекция 10}

	
\pagebreak
\lection{Лекция 11}

	
\end{document}