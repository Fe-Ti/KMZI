\documentclass[a4paper,12pt]{article}
\usepackage{amsmath,amssymb,amsfonts}
\usepackage{mathtext}
\usepackage[english,russian]{babel}
\usepackage[utf8]{inputenc}
\usepackage[T2A]{fontenc}
\usepackage{graphicx}
\usepackage{textcomp}
\usepackage{geometry}
\geometry{left=3cm}
\geometry{right=1.5cm}
\geometry{top=2cm}
\geometry{bottom=2cm}
\usepackage{tikz}
\usepackage{titling}
\usepackage{indentfirst}
\setlength{\parindent}{1cm}
\usepackage{soul}
\usepackage{enumitem}
\usepackage{listings}
\usepackage{fvextra}
\usepackage{tabularx}
\usepackage{fancyhdr}
\usepackage{setspace}
\usepackage{tikz}
\usepackage{tikz-cd}
\usetikzlibrary{shapes.geometric, positioning, arrows}
\usepackage{ifthen} % provides \isempty test
\usepackage{hyperref}

\hypersetup{
  colorlinks,
  citecolor=black,
  filecolor=black,
  linkcolor=black,
  urlcolor=black
}

%\documentclass{beamer}
%\usepackage{lmodern}
%\usepackage{tikz}
\usetikzlibrary{decorations.pathmorphing}
\tikzset{wavy/.style={decorate, decoration=snake}}

\onehalfspacing
\makeatletter
\AddEnumerateCounter{\asbuk}{\russian@alph}{щ}
\makeatother
\fvset{breaklines=true, breakafter=\space}

\DeclareRobustCommand{\divby}{%
  \mathrel{\vbox{\baselineskip.65ex\lineskiplimit0pt\hbox{.}\hbox{.}\hbox{.}}}%
}

\newcounter{theorem}
\newcounter{lemme}
\newcounter{definition}
\newcounter{statement}
\newcounter{property}
\newcounter{corollary}
\newcommand{\lection}[1]{\pagebreak\section*{#1}
  \addtocounter{section}{1}
  \addcontentsline{toc}{section}{#1}
  \setcounter{subsection}{0}
  \setcounter{theorem}{0}
  \setcounter{definition}{0}
  \setcounter{statement}{0}
  \setcounter{property}{0}
%  \setcounter{corollary}{0}
}

\newcommand{\seminary}[1]{\pagebreak\section*{#1}
  \addcontentsline{toc}{section}{#1}
}

\newcommand{\definition}{\stepcounter{definition}\paragraph{Определение \arabic{section}.\arabic{definition}}
}

\newcommand{\statement}[1]{\stepcounter{statement}\setcounter{corollary}{0}\paragraph{Утв. \arabic{section}.\arabic{statement}} #1}

\newcommand{\proof}{\paragraph{Доказательство}}

\newcommand{\corollary}[1]{\stepcounter{corollary}\paragraph{Следствие \arabic{corollary}} #1}

\newcommand{\theorem}[2][]{\stepcounter{theorem}\setcounter{corollary}{0}\paragraph{Теорема
    \arabic{section}.\arabic{theorem}}
  \ifthenelse{\equal{#1}{}}{}{(#1)} {\itshape #2 }}
\newcommand{\lemme}[2][]{\stepcounter{lemme}\setcounter{corollary}{0}\paragraph{Лемма \arabic{section}.\arabic{lemme}} \ifthenelse{\equal{#1}{}}{}{(#1)} {\itshape #2 }}

\newcommand{\property}[1]{\stepcounter{property}\paragraph{Свойство \arabic{section}.\arabic{property}}\textit{#1}}

\newcommand{\angles}[1]{\langle #1 \rangle}
\newcommand{\nonlinty}[1]{n_{#1}}
\newcommand{\iftext}{\text{если}}
\newcommand{\xor}{\oplus}
%\newcommand{\dist}[2]{d(#1,#2)}
\newcommand{\weight}[1]{\|#1\|}
\newcommand{\transp}[1]{#1^\top}
\newcommand{\ind}[1]{\text{I}(#1)}
\newcommand{\stack}[1]{\begin{matrix} #1 \end{matrix}}
\newcommand{\partialfrac}[2]{\frac{\partial #1}{\partial #2}}
\newcommand{\mx}[1]{\mathbf{#1}}

\newcommand{\ZZ}{\mathbb{Z}}
\newcommand{\RR}{\mathbb{R}}
\newcommand{\FF}{\mathbb{F}}
\newcommand{\CC}{\mathbb{C}}
\newcommand{\NN}{\mathbb{N}}
\newcommand{\vecx}{\vec{x}}
\newcommand{\vecy}{\vec{y}}
\newcommand{\hamming}{\chi}
  

\newcommand{\refeq}[1]{(\ref{#1})}

\fancypagestyle{titlepage}{
  \fancyhf{
    \begin{center}
      \begin{tabularx}{\textwidth}[c]{c p{12cm}}
        \raisebox{-1.1\totalheight}{\includegraphics[width=0.15\linewidth]{./logo/bauman_logo.png}}
        &\begin{center}
          \textbf{ Министерство науки и высшего образования Российской Федерации \\
            Федеральное государственное бюджетное образовательное учреждение 
            высшего образования \\
            «Московский государственный технический университет
            имени Н.Э. Баумана \\
            (национальный исследовательский университет)» \\ (МГТУ им. Н.Э. Баумана)} \\
          
        \end{center}
      \end{tabularx}
  \end{center}}
  \fancyfoot[C]{Москва 2024 г.}
  \renewcommand{\headrulewidth}{0pt}
}

\title{~\\~\\~\\~\\~\\~\\~\\~\\~\\Криптографические методы защиты информации}
\author{Конспект лекций}
\date{МГТУ им. Н.Э. Баумана}

\begin{document}
  \maketitle
  \thispagestyle{titlepage}
  \pagebreak
  \tableofcontents
  
  \section*{Disclaimer}
  \addcontentsline{toc}{section}{Disclaimer}
  Конспект создан студентами для подготовки к экзамену ввиду отсутствия удобного формата лекций. Поэтому он может содержать ошибки, опечатки и многое другое, за что можно получить автомат с сапогами в придачу.
  
  \textit{(Тим Кравченко aka Fe-Ti)}
  

  \pagebreak
  \lection{Лекция 1}
  \section{Сложение случайных величин в кольце $\mathbb{Z}_m$}
  Свойства (принципы) К. Шеннона:
  \begin{itemize}
  \item усложнение -- сложная зависимость между ключом и шифртекстом
  \item перемешивание -- существенное усложнение взаимосвязи
    статистических и аналитических характеристик знаков
    шифртекста по сравнению с открытым текстом
  \item рассеивания -- каждый знак открытого текста влияет на
    большое число знаков шифртекста
  \end{itemize}

  В криптографии сложение в кольце $\mathbb{Z}_m$ используется при
  синтезе:
  \begin{itemize}
  \item генераторов гаммы
  \item (криптографических) генераторов псевдослучайных
    последовательностей
  \item блочных шифрсистем для обеспечения свойства усложнения
  \end{itemize}

  \subsection{Случайные величины на $\mathbb{Z}_m$}

  Рассматриваем случайную величину $\xi$, принимающую значение
  на $\mathbb{Z}_m$ с вероятностями \[ p_i^{(\xi) = P\{\xi =
      t}, t \in \mathbb{Z}_m \]
  Статистическое ``качество'' случайной величины $\xi$
  характеризуется отклонением вероятности \[ P\{\xi = t\}
    \textrm{от} \frac{1}{m} \] -- её близостью к равномерно
  распределённой случайной величине

  \subsection{Лемма мацуи}
  \statement{ Пусть $n \ge 2, \xi_1, ..., \xi_n$ -- (попарно)
  независимые случайные величинаы на $\mathbb{Z}_2$, \[ P\{\xi_i =
    0\} = \frac{1 + \delta_j}{2}, -1 \le \delta_j \le 1, j = 1,
    ..., n \]
  Тогда для случайной величины \[ \eta_n = \xi_i + ... + \xi_n
    mod 2 \] справедливо равенство \[ P\{\eta_n = 0\} = \frac{1
      + \delta^{(n)}}{2} \] \[ \delta^{(n)} = \prod_{j = 1}^n
    \delta_j \]}
  \proof с помощью формулы полной вероятности и индукции по $n$.

  Для $n = 2$ имеем \[ P\{\eta_2 = 0\} = P\{\xi_1 = 0\}P\{\xi_2 = 0\} + P\{\xi_1 = 1\}P\{\xi_2 = 1\} \]
  \[
    \begin{aligned}
      &P\{\eta_{n+1} = 0\} = P\{\xi_{n+1} = 0\}P\{\eta_n = 0\} +
        P\{\xi_{n+1} = 1\}P\{\eta_n = 1\} = \\
      &\frac{1 + \delta_1
        ... \delta_{n}}{2} * \frac{1 + \delta_{n+1}}{2} + \frac{1
        - \delta_1 ... \delta_{n}}{2} * \frac{1 - \delta_{n+1}}{2}
        = \\
      &\frac{1}{4}(1 + \delta_1 ... \delta_{n} + \delta_{n+1} +
        \delta_1 ... \delta_{n + 1} + 1 - \delta_1 ... \delta_{n} -
        \delta_{n+1} + \delta_1 ... \delta_{n+1}) = \\
      &\frac{1}{4}(2 +
        2 \delta_1 ... \delta_{n+1}) = \frac{1 + \delta_1
        ... \delta_{n+1}}{2}
    \end{aligned}
  \]

  Каждой случайной величине $\xi$ на $\mathbb{Z}_m$ поставим в
  соответствие характеристический многочлен
  \begin{itemize}
  \item $f_{\xi}(x) = \sum_{i = 0}^{m - 1} x^i P\{\xi = i\}$
  \item $f_{\xi}(x)$ рассматриваем как элемент кольца
    $\mathbb{R}[x]/(x^m - 1)$. \[ \mathbb{R}[x]/(x^m - 1) =
      \{f(x) \in \mathbb{R}[x] | deg(f) < m\}. \]
  \end{itemize}
  Если $g_1, g_2 \in \mathbb{R}[x]/(x^m - 1)$, то
  \[ g = g_1 + g_2 mod x^m - 1, \]
  \[ g' = g_1 * g_2 mod x^m - 1. \]

  \theorem{Пусть $n \ge  2, \xi_1, ..., \xi_n$ -- назависимые
    случайные величина на $\mathbb{Z}_m$, \[ \eta = \xi_1 + ... +
      \xi_n mod m. \]
    Тогда и только тогда $\eta$ имеет равномерное распределение,
    когда для каждого $i \in \{1, ..., r\}$ существует такой $j
    \in \{1, ..., n\}$, что \[ f_{\xi_i}(x) \equiv 0 (mod
      h_i(x)). \]}

  \section{Линейные преобразования в шифрсистемах. MDS-матрицы}
  \definition{Свойство рассеивания -- каждый знак открытого текста влияет на}
  большое число знаков шифртекста.
  MDS-матрицы применяются для реализации свойства рассеивания
  (преобразование L-слоя).
  \begin{itemize}
  \item блочные: AES, Кузнечик, FOX, Twofish, Khazad, SHARK,
    Square, Clefia
  \item функции хэширования: Стрибог, Whirlpool, ...
  \item ???
  \end{itemize}
  
  \subsection{MDS-матрицы}
  \begin{itemize}
  \item $V_n(q)$ -- n-мерное векторное пространство над
    $\mathbb{F}_q$
  \item $h: V_n(q) \rightarrow V_m(q)$ -- линейное преобразование
  \item $h = [[ h_{i, j} ]]$ -- $n \times m$-матрица над
    $\mathbb{F}_q$ преобразования $h$
  \end{itemize}
  
  Полагаем \[ x^h = h(x) = xh = (y_1, ..., y_m) \] где \[ y_i =
    \sum_{j=1}^n h_{j, i}x_j \textrm{при} i = 1, ..., m \]
  \begin{itemize}
  \item $x = (x_1, ..., x_n) \in V_n(q), (y_1, ..., y_m) \in
    V_m(q)$
  \item $h(x) \in V_m(q)$
  \end{itemize}
  
  \definition{Матрица h над $\mathbb{F}_q$ называется
    MDS-матрицей, если \[ m + 1  = min\{||(\alpha_1, ...,
      \alpha_n, \alpha_h)||: \alpha = (\alpha_1, ..., \alpha_n)
      \in V_n(q) \ \{0_n\}\} \]} где
  \begin{itemize}
  \item $(\alpha_1, ..., \alpha_n, \alpha_h) = (\alpha_1, ...,
    \alpha_n, \beta_1, ..., \beta_m) - (n + m)$ -- n-мерный вектор
    над $\mathbb{F}_q$
  \item $\alpha = (\alpha_1, .., \alpha_n) \in V_n(q) \ \{0_n\}$
  \item $\alpha h = (\beta_1, ..., \beta_m)$
  \end{itemize}
  $\implies$ Минимальный вес ненулевых $(n + m)$-мерных векторов
  над $\mathbb{F}_q$ равен $m + 1$

  \textit{Эквивалентное определение}:
  При изменении любых $r \in \{1, ..., n\}$ координат каждого
  вектора $\alpha \in V_n(q) \ \{0_n\}$ у вектора $\alpha h$
  изменяется не менее $m - r + 1$ координат $\implies$ реализация
  хорошего рассеивания

  \definition{(Разностным) показателем (коэффициентом) рассеивания
    $\rho_h^{(dif)}$ матрицы $h$ порядка $n$ над $\mathbb{F}_q$
    называется \[ \rho_h^{(dif)} = min\{||\alpha|| + ||\alpha h||
      | \alpha \in V_n(q) \ \{0\}\} \]}
  \definition{(Линейным) показателем (коэффициентом) рассеивания
    $\rho_h^{(lin)}$ матрицы $h$ порядка $n$ над $\mathbb{F}_q$ 
    называется \[ \rho_h^{(lin)} = min\{||\alpha|| + ||\alpha h^T||
      | \alpha \in V_n(q) \ \{0\}\} \]}
  \lemme{
  Пусть $h$ -- матрица порядка $n$ над $\mathbb{F}_q$,
  $p_1, p_2$ -- подстановочные матрицы порядка $n$. Тогда
  \[
  \rho_h = \rho_{p_1 h p_2}
  \]}
  
  \textbf{Замечание}
  \begin{itemize}
  \item У матриц $h$ и $p_1 h p_2$ показатель рассеивания одинаков
  \item Любая перестановка строк или столбцов матрицы $h$ не
    изменяет показатель рассеивания
  \end{itemize}

  \subsection{Задачи}
  \begin{itemize}
  \item Доказать эквивалентность определений MDS-матриц
  \item Существуют ли MDS-матрицы над $\mathbb{F}_2$
  \item Доказать что MDS-матрица остаётся MDS-матрицей после
    следующих преобразований:
    \begin{itemize}
    \item перестановка строк или столбцов
    \item умножения строк или столбцов на любой ненулевой элемент
      поля $\mathbb{F}_q$
    \item транспонирование
    \end{itemize}
  \end{itemize}

  \subsection{Классы MDS-матриц}
  \begin{itemize}
  \item Матрицы Коши
  \item Матрицы Вандермонда
  \item Циркулянтные матрицы
  \item Матрицы Адамара
  \item Степень сопровождающей матрицы многочлена (LFSR)
  \end{itemize}

  \subsubsection{Матрица Коши}
  Пусть $\mathbb{F}$ -- поле, $\alpha_0, ..., \alpha_{n-1}, b_0,
  ..., b_{n-1} \in \mathbb{F}$
  \definition{$(n \times n)$-матрица $\nu = [[\nu_{i, j}]]$ с
    элементами \[ \nu_{i, j} = (a_i + b_i)^{-1}, i, j \in \{0,
      ..., n - 1\} \] называется матрицей Коши}
  
  \textbf{Задача}. Доказать что \[ det(\nu) = \frac{\prod_{0 \le i
        < j \le n - 1}(a_j - a_i)(b_j - b_i)}{\prod_{0 \le i, j
        \le n - 1}(a_i + b_j)} \]
  \textbf{Задача}. Доказать что каждая квадратная подматрица
  матрицы Коши является матрицей Коши


  \subsubsection{Матрицы Вандермонда}
  Пусть $\mathbb{F}$ -- поле, $a_0, ..., a_{n - 1} \in \mathbb{F}$

  Матрица \[ \nu = van(a_0, ..., a_{n - 1}) =
    \begin{pmatrix}
      1 & a_0 & a_0^2 & ... & a_0^{n-1} \\
      1 & a_1 & a_1^2 & ... & a_1^{n-1} \\
      ... & ... & ... & ... & ... \\
      1 & a_{n-1} & a_{n-1}^2 & ... & a_{n-1}^{n-1}
    \end{pmatrix}
  \]
  называется матрицей Вандермонда

  \textbf{Задачи}

  \begin{itemize}
  \item Доказать что \[ det(\nu) = \prod_{0 \le i < j \le n - 1} (a_i -
      a_j) \]
  \item $det(\nu) \ne 0$, если $a_0, ..., a_{n - 1}$ попарно
    различны
  \item если $\mathbb{F} = \mathbb{R}$, то каждая квадратная
    подматрица матрицы Вандермонда является невырожденой
  \item если $\mathbb{F} = \mathbb{F}_q$, то в конечном поле может
    найтись вырожденная квадратная подматрица матрицы Вандермонда
  \end{itemize}

  \subsubsection{Циркулянтные матрицы}
  \definition{$(n \times n)$-матрица $a = [[ a_{i, j} ]]$ над
    полем $\mathbb{F}$ вида \[ a = circ(a_0, ..., a_{n - 1}) =
      \begin{pmatrix}
        a_0 & a_1 & ... & a_{n-1} \\
        a_{n-1} & a_0 & ... & a_{n-2} \\
        ... & ... & ... & ... \\
        a_{1} & a_{2} & ... & a_{0}
      \end{pmatrix}
    \]
    задаваемая первой строкой $(a_0, ..., a_{n-1})$, называется
    циркулянтной}

  \subsubsection{Связь MDS-матриц с MDS-кодами}
  MDS (maximum distance separable code)
  \begin{itemize}
  \item МДР-коды (разделимые коды с максимальным расстоянием)
  \item МДР-код имеет максимально возможное расстояние между
    кодовыми словами (относительно метрики Хемминга)
  \end{itemize}

  Проблема построения МДР-кодом максимально возможной длины при
  фиксированной размерности кода эквивалентна некоторым
  интереснейшим не полностью решённым комбинаторным задачам.

  Для кода \[ C = \{(\alpha_1, ..., \alpha_n, \alpha h) | \alpha =
    (\alpha_1, ..., \alpha_n) \in V_n(q) \ \{0_n\}\} \]
  над $\mathbb{F}_q$
  \begin{itemize}
  \item длины $n + m$
  \item размерности $n$
  \item с MDS-матрицей $h$
  \end{itemize}
  $\implies$ минимальное кодовое расстояние равно $m + 1$

  \property{граница Синглетона} Пусть $C$ -- линейный $(n, k,
  d)$-код над $\mathbb{F}_q$. Тогда \[ d \le n - k + 1 \]

  Для МДР-кода \[ d = n - k + 1 \] -- достигается максимальное
  расстояние между кодовыми словами

  Каждый линейный код $C$ однозначно задаётся своей проверочной
  $(n - k) \times b$-матрицей $H$, \[ H \alpha^T = 0_{n-k}
    \forall \alpha \in C, rank(H) = n - k \]
  \begin{itemize}
  \item $\alpha = \alpha_1, ..., \alpha_k, \alpha_{k + 1}, ...,
    \alpha_n$ -- кодовое слово, $\alpha \in C$
    \begin{itemize}
    \item $k$ -- число информационных символов
    \item $n - k$ -- число проверочных символов
    \end{itemize}
  \item Порождающая матрица линейного кода $C$ -- $(k \times
    n)$-матрица $G$, \[
      \begin{aligned}
        GH^T = 0, HG^T = 0 \\
        rank(G) = k
      \end{aligned}
    \]
  \end{itemize}

  % TODO Замечание 1, замечание 2

  \statement{Пусть $C$ -- линейный $(n, k, d)$-код над
    $\mathbb{F}_q$ с проверочной матрицей $H$. Тогда и только
    тогда $C$ -- МДР-код, когда каждые $n - k$ столбцов матрицы
    $H$ линейно независимы.}
  \proof Код $C$ содержит кодовое слово веса $j$ $\Leftrightarrow$
  % TODO

  \[ C^{\bot} = \{\beta \in V_n(q) | <\alpha, \beta> = 0 \forall
    \alpha \in C\} \]

  \statement{Пусть $C$ -- линейный $(n, k, d)$-код над $\mathbb{F}_q$
    с проверочной матрицей $H, C$ -- МДР-код $\Leftrightarrow$ для
    каждого набора $(i_i, ..., i_d), 1 \le i_1 < ... < i_d \le n$,
    существует кодовое слово}
  \begin{itemize}
  \item $\alpha \in C, |\alpha| = d, \alpha_{i_1}, ..., \alpha_{i_d}
    \in \mathbb{F}_q \ \{0\}$
  \item $d = n - k + 1$
  \item $\alpha$ -- кодовое слово минимального веса
  \end{itemize}
  $\implies$ Число кодовых слов МДР-кода $C$ веса $n - k + 1$ равно
  \[ (q - 1) \binom{n}{n - k + 1} \]

  \statement{Пусть $C$ -- линейный $(n, k, d)$-код над $\mathbb{F}_q$
    с порождающей матрицей $G = (E_k | A)$, где $A$ -- $k \times (n -
    k)$-матрица. $C$ -- МДР-код $\Leftrightarrow \forall$  квадратная
    подматрица матрицы $A$ невырождена
  \begin{itemize}
  \item квадратная подматрица матрицы $A$ образована любыми $i \in
    \{1, ..., min\{k, n-k\}\}$ строками и столбцами
  \item невырожденная -- определитель не равен 0 в $\mathbb{F}_q$
  \end{itemize}}

  \theorem{Каждая квадратная подматрица MDS-матрицы невырождена}
  \statement{Пусть $c = [[c_{i, j}]] - (n \times n)$- матрица над
    полем $\mathbb{F}$ $\Leftrightarrow \forall ((n - 1)\times(n -
    1))$- подматрица матрицы $c$ невырождена, когда все элементы
    обратной матрицы $c^{-1} = \nu = [[\nu_{i,j}]]$ отличны от 0}
  \begin{itemize}
  \item $\nu_{i, j} \ne 0 \forall i, j \in \{0, ..., n - 1\}$
  \end{itemize}
  \proof следует из \[ \nu_{i, j} = Adj(c_{i, j}) * det(c)^{-1} \]
  где $Adj(c_{i, j})$ -- определитель $((n - 1) \times (n -
  1))$-подматрицы полученной вычёркиванием $i$-й строки и $j$-го
  столбца матрицы $c$, $i, j \in \{0, ..., n - 1\} \blacksquare$

  \lemme{Пусть $\mathbb{F}$ -- произвольное поле, $a_0, ..., a_{n-1},
    b_0, ..., b_{n-1} \in \mathbb{F}$
    \begin{itemize}
    \item $a_0, ..., a_{n-1}$ попарно различны
    \item $b_0, ..., b_{n-1}$ попарно различны
    \item $a_i + b_j \ne 0 \forall i, j \in \{0, ..., n - 1\}$
    \end{itemize}
    то матрица Коши порядка $n$ -- MDS-матрица}
  \lemme{Каждая строка (столбец) MDS-матрицы Коши порядка $n$ имеет
    $n$ различных элементов}
  % TODO proof
  \lemme{Каждая циркулянтная матрица \[ a = circ(a_0, ..., a_{n -
        1}) \] порядка $n$ над полем $\mathbb{F}$ представима в
    виде \[ a = a_0 e + a_1 p + a_2 p^2 + ... + a_{n - 1} p^{n - 1} \]
    \begin{itemize}
    \item $p = circ(0, 1, 0, ..., 0)$
    \item $e$ -- единичная матрица
    \end{itemize}
  }
  \lemme{Пусть $a = circ(a_0, ..., a_{2^t - 1})$ -- циркулянтная
    матрица порядка $2^t$ над полем $\mathbb{F}_{2^d}$. Тогда
    \[ a^{2^t} = (\sum_{i = 0}^{2^t - 1} a_i^{2^t})e \]
    \[ det(a) = \sum_{i = 0}^{2^t - 1} a_i^{2^t} \]
  }

  \subsection{Эффективная реализация MDS-матриц}
  Большое число работ посвящено поиску эффективно реализуемых матриц с
  хорошими рассеивающими свойствами

  Ищут MDS-матрицы:
  \begin{itemize}
  \item с наибольшим числом единиц
  \item с наименьшим числом различных элементов
  \end{itemize}

  $MDS_{n,m}(q)$ -- множество всех MDS-матриц размера $(n \times m)$
  над $\mathbb{F}_q$

  \definition Пусть $c = [[ c_{i, j} ]]$ -- MDS-матрица размера $n
  \times m$ над $\mathbb{F}_{2^p}$
  \begin{itemize}
  \item $\nu_1(c) = |\{(i, j) \in \mathbb{Z}_n \times \mathbb{Z}_m |
    c_{i, j} = 1 \}|$ -- $\nu_1(c)$ -- число единиц среди всех
    элементов матрицы $c$
  \item $\nu_1^{(n, m)} = max\{\nu_1(c) | c \in MDS_{n, m}(2^p)\}$
  \end{itemize}

  \textbf{Замечание}. Для синтеза эффективно реализуемых (программно
  или аппаратно) MDS-матриц выбирают матрицы $c$:
  \begin{itemize}
  \item $\nu_1(c)$ как можно больше
    \begin{itemize}
    \item незначительно отличается от $\nu_1^{(n, m)}$
    \end{itemize}
  \item $u(c)$ как можно меньше
    \begin{itemize}
    \item незначительно отличается от $u_1^{(n, m)}$
    \end{itemize}
  \item элементы матрицы $c$ имеют минимальный вес Хемминга
  \end{itemize}

  \definition Пусть $f(x)$ -- многочлен над $\mathbb{F}, r =
  deg(f(x))$. Возвратный многочлен $f^{\star}$ над $\mathbb{F}$ задан
  условием: \[ f^{\star} = x^r f(x^{-1}) \]

  \subsection{Атаки на основе инвариантных подпространств}
  \textit{На примере \textbf{Khazad}}

  Khazad XSL-шифрсистема:
  \begin{itemize}
  \item 128-битный ключ
  \item длина блока -- 64 бита
  \end{itemize}
  \begin{itemize}
  \item 64-битный блок представим в виде вектора \[ \alpha = ( \alpha_0, ..., \alpha_7) \in V_8(2^8) \]
  \item $p(x) = x^8 + x^4 + x^3 + x^2 + 1 \in \mathbb{F}_2[x]$ -- неприводимый многочлен
  \item $\mathbb{F}_{2^8} = \mathbb{F}_2[x] / (p(x))$

    -- $\beta$ -- корень многочлена $p(x)$ в поле разложения
  \item соответствие
    \[ u_0 + u_1\beta + ... + u_7\beta^7 \in \mathbb{F}_{2^8} \Rightarrow \\
      \Rightarrow u_0 + u_1 2 + ... + u_7 2^7 \in \{0, ..., 255\} \]
  \item Нелинейное преобразование $\tilde{s}$ реализуется путём
    применения к координатам вектора $\alpha$ подстановки $s \in
    S(F_{2^8})$: \[ \tilde{s}: (\alpha_0, ..., \alpha_7) \Rightarrow
      (\alpha_0^s, ..., \alpha_7^s) \]
  \item Линейное преобразование осуществляется путём умножения
    вектор-строки $\alpha$ на матрицу $h = [[h_{i, j}]] \in
    GL_8(2^8)$, где \[ h =
      \begin{pmatrix}
        1  & 3  & 4  & 5  & 6  & 8  & 11 & 7 \\
        3  & 1  & 5  & 4  & 8  & 6  & 7  & 11 \\
        4  & 5  & 1  & 3  & 11 & 7  & 6  & 8 \\
        5  & 4  & 3  & 1  & 7  & 11 & 8  & 6 \\
        6  & 8  & 11 & 7  & 1  & 3  & 4  & 5 \\
        8  & 6  & 7  & 11 & 3  & 1  & 5  & 4 \\
        11 & 7  & 6  & 8  & 4  & 5  & 1  & 3 \\
        7  & 11 & 8  & 6  & 5  & 4  & 3  & 1
      \end{pmatrix}
      \]
  \item  Алгоритм развёртывания ключа -- зашифрования векторов $k_{--
      1}, k_{-- 2} \in V_8(2^8)$ с помощью преобразования Фейстеля с
    функцией усложнения
  \item \hl{Missed it}
  \item Вычисление раундовых ключей $k_0, ..., k_7$: \[ k_i = (k_{i -
        1})^{g_{c_i}} + k_{i - 2} \]
  \item Функция зашифрования за $r$ раундов:
  \[ f_{k_0, ..., k_r} =
      \nu_{k_0}\tilde{s}h\nu_{k_1}\tilde{s}h ... \nu_{k_{r-1}}\tilde{s}h\nu_{k_r}
  \]
  \end{itemize}

  \subsubsection{Атаки на шифрсистему Khazad}
  \begin{table}[h]
  \begin{tabularx}{\textwidth}{X|c|c|c}
    Вид атаки & Число раундов & Трудоёмкость & Объём материала \\\hline
    Интегральная атака [BarR 00] & 3 & \(2^{16}\) & \(2^9\) \\
    Запретные разности [NES] & 3 &\( 2^{64}\) & \(2^{13}\) \\
    Интегральная атака [BarR 00] & 4 & \(2^{80}\) & \(2^9 \)\\
    Слайд атака (\(2^{64}\) слабых ключей) [Bir 03] & 5 & \(2^{43} \)& \(2^{38}\)
    \\
    Улучшенная интегральная атака [Mul 03] & 5 & \(2^{91}\) & \(2^{64}\)
  \end{tabularx}
  \end{table}

  
  \subsubsection{Сохранение инвариантах подпространств}
  
  Наличие инвариантных подпространств у преобразования $\tilde{s}h$
  является потенциальной слабостью шифрсистемы

  \textbf{Замечание}. Из трёх преобразований в раунде только
  преобразование наложения ключа $\nu_k$ может перевести $W$ в другой
  смежный класс.
  \lemme{
    \[W_{\nu_k} = W \Leftrightarrow k \in W \]
    }

  Из вида алгоритма развёртывания ключа следует существование ровно
  $|W|^2$ возможных значений для векторов $k_{-- 1}, k_{-- 2}$ таких,
  что $k_1, k_2 \in W$. -- $k_i = (k_{i - 1})^{g_{c_i}} + k_{i - 2}$

  \statement{ Для каждого $j \in \{0, ..., 7\}$ имеет место равенство:
  \[ \sum_{i = 0}^{|W| - 1} \alpha_{i, j} = 0 \]}
  \proof Пусть \[ (W + c_3)^{\tilde{s}} = \{ (\delta_{i, 0}, ...,
    \delta_{i, 7}) | i = 0, ..., |W| - 1 \} \]
  Представим сумму в виде:
  \[ \sum_{i = 0}^{|W| - 1} \alpha_{i, j} = \sum_{i = 0}^{|W| -
      1}\sum{t = 0}^{7} \delta_{i, t} h_{t, j} = \\
    \sum_{t = 0}^{7} h_{t, j} \sum_{t = 0}^{|W| - 1} \delta_{i, t} \]
  Если \[ W \in \{U^{(1)}, ..., U^{(7)}, W^{(1)}, ..., W^{(7)}\}, \]
  то каждый элемент $x \in \mathbb{F}_{2^8}$ встречается в любой
  фиксированной координате $t \in \{0, ..., 7\}$ чётное число раз.

  Так как характеристика поля $\mathbb{F}_{2^8}$ равна 2, равенство
  доказано. $\blacksquare$

  \subsection{Атака на 5-раундовый Khazad}
  Пусть \[ B = \{\alpha^{f_{k_0, ..., k_5}} | \alpha \in W + c_2 \} \]

  Пронумеруем элементы $B$, $\beta_i$ -- $i$-й элемент $B$,
  \[ B = \{(\beta_{i, 0}, ..., \beta_{i, 7}) | i = 0, ..., |W| - 1
    \} \]
  \[ \beta_i = (\beta_{i, 0}, ..., \beta_{i, 7}) \in B \]

  \textbf{Алгоритм}\\
  \textbf{Вход}: множество $B = \{\alpha^{f_{k_0, ..., k_5}} | \alpha
  \in W + c_2 \}$
  \textbf{Выход}: множество возможных кандидатов для $k_5$.

  Для $i = 0, ..., 7$ выполняем:
  \begin{itemize}
  \item Опробовать $k_{5, i} \in \mathbb{F}_{2^8}$
  \item Для опробируемого $k_{5, i}$ проверить выполнение равенства
    \[ \sum_{j = 0}^{|W| - 1}(\beta_{j, i} + k_{5, i})^{s^{-1}} = 0 \]
  \item Если равенство не выполняется, то опробуемый вариант $k_{5,
      i}$ является ложным
  \end{itemize}
  \begin{itemize}
  \item Для случайной подстановки равенство будет выполняться с
    вероятностью $2^{-8}$
  \item Для каждого $i \in \{0, ..., 7\}$ в среднем будет оставаться
    два вариата ключей $k_{5, i}$, один из которых является истинным
  \item Значит трудоёмкость метода можно оценить 
    \begin{itemize}
    \item $8 * 2^8 * |W|$ операций применения подстановки $s^{-1}$
    \item Объём необходимого материала равен $|W|$ блокам текста
    \item Надёжность метода равна 1
    \item Число слабых ключей равно $|W|^2$
    \end{itemize}
  \item Если $W \in \{W^{(1)}, ..., W^{(7)}\}$, то
    \begin{itemize}
    \item трудоёмкость равна $2^{43}$ операций
    \item объём необходимого материала равен $2^{32}$ блоков текста
    \item число слабых ключей равно $2^{64}$
    \end{itemize}
  \end{itemize}

  
\end{document}
