\documentclass[a4paper,12pt]{article}
\usepackage{amsmath,amssymb,amsfonts}
\usepackage{mathtext}
\usepackage[english,russian]{babel}
\usepackage[utf8]{inputenc}
\usepackage[T2A]{fontenc}
\usepackage{graphicx}
\usepackage{textcomp}
\usepackage{geometry}
\geometry{left=3cm}
\geometry{right=1.5cm}
\geometry{top=2cm}
\geometry{bottom=2cm}
\usepackage{tikz}
\usepackage{titling}
\usepackage{indentfirst}
\setlength{\parindent}{1cm}
\usepackage{soul}
\usepackage{enumitem}
\usepackage{listings}
\usepackage{fvextra}
\usepackage{tabularx}
\usepackage{fancyhdr}
\usepackage{setspace}
\usepackage{tikz}
\usepackage{tikz-cd}
\usetikzlibrary{shapes.geometric, positioning, arrows}
\usepackage{ifthen} % provides \isempty test
\usepackage{hyperref}

\hypersetup{
	colorlinks,
	citecolor=black,
	filecolor=black,
	linkcolor=black,
	urlcolor=black
}

%\documentclass{beamer}
%\usepackage{lmodern}
%\usepackage{tikz}
\usetikzlibrary{decorations.pathmorphing}
\tikzset{wavy/.style={decorate, decoration=snake}}

\onehalfspacing
\makeatletter
\AddEnumerateCounter{\asbuk}{\russian@alph}{щ}
\makeatother
\fvset{breaklines=true, breakafter=\space}

\DeclareRobustCommand{\divby}{%
	\mathrel{\vbox{\baselineskip.65ex\lineskiplimit0pt\hbox{.}\hbox{.}\hbox{.}}}%
}

\newcounter{theorem}
\newcounter{lemme}
\newcounter{definition}
\newcounter{statement}
\newcounter{property}
\newcounter{corollary}
\newcommand{\lection}[1]{\pagebreak\section*{#1}
	\addtocounter{section}{1}
	\addcontentsline{toc}{section}{#1}
	\setcounter{subsection}{0}
	\setcounter{theorem}{0}
	\setcounter{definition}{0}
	\setcounter{statement}{0}
	\setcounter{property}{0}
%	\setcounter{corollary}{0}
}

\newcommand{\seminary}[1]{\pagebreak\section*{#1}
	\addcontentsline{toc}{section}{#1}
}

\newcommand{\definition}{\stepcounter{definition}\paragraph{Определение \arabic{section}.\arabic{definition}}
}

\newcommand{\statement}[1]{\stepcounter{statement}\setcounter{corollary}{0}\paragraph{Утв. \arabic{section}.\arabic{statement}} #1}

\newcommand{\proof}{\paragraph{Доказательство}}

\newcommand{\corollary}[1]{\stepcounter{corollary}\paragraph{Следствие \arabic{corollary}} #1}

\newcommand{\theorem}[2][]{\stepcounter{theorem}\setcounter{corollary}{0}\paragraph{Теорема
    \arabic{section}.\arabic{theorem}}
  \ifthenelse{\equal{#1}{}}{}{(#1)} {\itshape #2 }}
\newcommand{\lemme}[2][]{\stepcounter{lemme}\setcounter{corollary}{0}\paragraph{Лемма \arabic{section}.\arabic{lemme}} \ifthenelse{\equal{#1}{}}{}{(#1)} {\itshape #2 }}

\newcommand{\property}[1]{\stepcounter{property}\paragraph{Свойство \arabic{section}.\arabic{property}}\textit{#1}}

\newcommand{\angles}[1]{\langle #1 \rangle}
\newcommand{\nonlinty}[1]{n_{#1}}
\newcommand{\iftext}{\text{если}}
\newcommand{\xor}{\oplus}
%\newcommand{\dist}[2]{d(#1,#2)}
\newcommand{\weight}[1]{\|#1\|}
\newcommand{\transp}[1]{#1^\top}
\newcommand{\ind}[1]{\text{I}(#1)}
\newcommand{\stack}[1]{\begin{matrix} #1 \end{matrix}}
\newcommand{\partialfrac}[2]{\frac{\partial #1}{\partial #2}}
\newcommand{\mx}[1]{\mathbf{#1}}

\newcommand{\ZZ}{\mathbb{Z}}
\newcommand{\RR}{\mathbb{R}}
\newcommand{\FF}{\mathbb{F}}
\newcommand{\CC}{\mathbb{C}}
\newcommand{\NN}{\mathbb{N}}
\newcommand{\vecx}{\vec{x}}
\newcommand{\vecy}{\vec{y}}
\newcommand{\hamming}{\chi}
\newcommand{\rank}{\text{rank }}
	

\newcommand{\refeq}[1]{(\ref{#1})}

\fancypagestyle{titlepage}{
	\fancyhf{
		\begin{center}
			\begin{tabularx}{\textwidth}[c]{c p{12cm}}
				\raisebox{-1.1\totalheight}{\includegraphics[width=0.15\linewidth]{./logo/bauman_logo.png}}
				&\begin{center}
					\textbf{ Министерство науки и высшего образования Российской Федерации \\
						Федеральное государственное бюджетное образовательное учреждение 
						высшего образования \\
						«Московский государственный технический университет
						имени Н.Э. Баумана \\
						(национальный исследовательский университет)» \\ (МГТУ им. Н.Э. Баумана)} \\
					
				\end{center}
			\end{tabularx}
	\end{center}}
	\fancyfoot[C]{Москва 2024 г.}
	\renewcommand{\headrulewidth}{0pt}
}

\title{~\\~\\~\\~\\~\\~\\~\\~\\~\\Криптографические методы защиты информации}
\author{Конспект лекций}
\date{МГТУ им. Н.Э. Баумана}

\begin{document}
	\maketitle
	\thispagestyle{titlepage}
	\pagebreak
	\tableofcontents
	
	\section*{Disclaimer}
	\addcontentsline{toc}{section}{Disclaimer}
	Конспект создан студентами для подготовки к экзамену ввиду отсутствия удобного формата лекций. Поэтому он может содержать ошибки, опечатки и многое другое, за что можно получить автомат с сапогами в придачу.
	
	\textit{(Тим Кравченко aka Fe-Ti)}
	

	\pagebreak
	\lection{Лекция 1}
	
	\paragraph{Задачи для разминки}
	\begin{itemize}
	\item Доказать, что $x \rightarrow (45^x \mod 257) \mod 256$ --- подстановка над $\ZZ_{2^8}$.
	\item XOR-ортоморфизм --- подстановка $\sigma: \begin{aligned}
	S &\rightarrow S \\
	a &\rightarrow \sigma(a)
	\end{aligned}$, если $\sigma: \begin{aligned}
		S &\rightarrow S \\
		a &\rightarrow \sigma(a)\xor a\end{aligned}$ ---подстановка. Являются ли следующие отображения XOR-ортоморфизмами:
		\begin{itemize}
		\item $w : x \rightarrow ROT^4(x\xor (x >> 4))$,
		\item $\pi : x\rightarrow (\mathtt{0xAA} \cdot x) \xor ROT^1(x)$?
		\end{itemize}
	\end{itemize}
	
	\subsection{Псевдобулева функция}
	Пусть $F$ --- произвольное поле, $p \ge 2$ --- простое число, $V_n(p) = GF(p)^n$ --- $n$-мерное пространство над полем $GF(p)$.
	
	\definition
	Псевдобулевой функцией от $n$ переменных называется произвольное отображение $f: V_n(2) \rightarrow F$. \\
	
	Если $F = GF(2)$, то функция называется булевой (двоичной). $f: V_n(p) \rightarrow F$ --- обобщение псевдобулевой функции.
	
	Обозначим $F(p,n) = \{f ~| ~ f:V_n(p) \rightarrow F\}$ и зададим на нём операции сложения и умножения на элемент поля:
	$$
	\begin{aligned}
		(f_1 + f_2) (x) &= f_1(x) + f_2 (x) \\
		(r \cdot f) (x) &= rf(x), ~ r \in F.
	\end{aligned}
	$$
	
	\statement{Относительно заданных операций $F(p,n)$ является векторным пространством размерности $p^n$ над $F$.}
	\proof
	(Д/з: проверить, что это векторное пространство)
	
	$\left\{ h_\alpha (\vec{x}) = \left[ \begin{aligned}
		1, ~ \iftext ~ \vec{x} = \vec{\alpha} \\
		0, ~ \iftext ~ \vec{x} \ne \vec{\alpha}
	\end{aligned} \right. ~\vline~ \vec{\alpha} \in GF(p)^n \right\}$ --- один из базисов пространства. Линейная независимость очевидна, и любая функция $f \in F(p,n)$:
	$$
	f(\vec{x}) = \sum_{\vec{\alpha} \in GF(p)^n} f(\vec{\alpha}) h_\alpha(\vec{x}). ~\blacksquare
	$$
	
	\subsection{Скалярное произведение}
	\definition 
	Скалярным произведением векторов $\vec{\alpha} = (\alpha_1, ..., \alpha_n)$, $\vec{\beta} = (\beta_1, ..., \beta_n)$, $\vec{\alpha}, \vec{\beta} \in V_n(2)$ называется $$\angles{\vec{\alpha},\vec{\beta}} = \bigoplus_{i=1}^{n} \alpha_i \beta_i.$$
	
	\definition
	Два вектора $\vec{\alpha}, \vec{\beta}$ ортогональны, если $\angles{\vec{\alpha}, \vec{\beta}} = 0$.
	
	\subsection{Аффинные функции}
	\definition
	$\mathcal{AL}_n = \{f \in \mathbb{F}_2 (n) ~|~ \deg f \le 1 \}$ --- множество аффинных функций.
	
	\definition
	$\mathcal{L}_n = \{f \in \mathcal{AL}_n ~|~ f(\vec{0} = 0)\}$ --- множество всех линейных функций.
	
	$$f \in \mathcal{AL}_n ~\Leftrightarrow ~ f(\vec{x}) = \alpha_1 x_1 \oplus ... \oplus \alpha_n x_n \oplus \alpha_0 = \angles{\vec{x}, \vec{\alpha}} \oplus \alpha_0.$$
	
	\paragraph{Задача} Доказать:
	$$
	\begin{aligned}
		|\mathcal{L}_n| &= 2^n \\
		|\mathcal{AL}_n| &= 2^{n+1}
	\end{aligned}
	$$
	
	\subsection{Равновероятная функция}
	Пусть $f = \vec{f} = (f(0), f(1), ..., f(2^n - 1))$ --- вектор столбец булевой функции $f(\vec{x})$, а $\weight{f} = \sum_{\vec{\alpha} \in V_n(2)} f(\vec{\alpha})$ --- вес этой функции.
	
	\definition
	Функция $f \in F_2(n)$ называется равновероятной (сбалансированной, равновесной), если $\weight{f} = 2^{n-1}$.
	
	\paragraph{Задача} Доказать, что если $f \in \mathcal{AL}_n$, то $f$ является равновесной.
	
	\paragraph{Задача} Пусть $f \in \mathbb{F}_2(n), ~ g \in \mathbb{F}_{2}(n+1)$, $g(x_1,...,x_{n+1}) = f(x_1, ...,x_n)$, $x_{n+1}$ --- фиктивная переменная функции $g$. Доказать, что $\weight{g} = 2\cdot \weight{f}$.
	
	\paragraph{Задача} Пусть $f\in \mathbb{F}_2(n), ~ g \in \mathbb{F}_2(n+2)$: $g(x_1,~ ...,~ x_{n+2}) = f(x_1,~ ...,~ x_{n-1},~ x_{n+1} \oplus x_{n+2})$. Известен $\weight{f}$. Найти $\weight{g}$.
	
	\paragraph{Задача} Пусть $f\in \mathbb{F}_2(n), ~ g \in \mathbb{F}_2(n+1)$: $g(x_1,~ ...,~ x_{n+1}) = f(x_1,~ ...,~ x_{n}) \oplus x_{n+1}$. Доказать, что функция $g$ равновероятна.
	
	\paragraph{Задача} Пусть $f \in \mathbb{F}_2(n)$. Доказать:
	$$ \weight{f} \equiv 1 ~ (\text{mod} 2) ~~ \Leftrightarrow ~~ \deg f = n.$$

	\subsection{Производная булевой функции}
	
	\definition
	Производной булевой функции $f \in \mathbb{F}_2(n)$ по направлению $\vec{\delta} \in V_n(2)$ называется булева функция
	$$
	d_{\vec{\delta}} f(\vec{x}) = f(\vec{x} \oplus \vec{\delta}) \oplus f(\vec{x}), ~~ \vec{x} \in V_n (2).
	$$
	
	\definition
	Производной булевой функции $f \in \mathbb{F}_2(n)$ по подпространству $L \le V_n(2)$ называется булева функция
	$$
	d_L f(\vec{x}) = \sum_{\vec{\delta} \in L} f(\vec{x}  \oplus \vec{\delta}), ~~ \vec{x} \in V_n (2).
	$$
	
	\paragraph{Свойство} Пусть $\vec{\delta}_1, ..., \vec{\delta}_k$ --- базис подпространства $L \le V_n(2)$. Тогда $$d_L f = d_{\vec{\delta}_k} ... d_{\vec{\delta}_1} f.$$
	
	\paragraph{Доказать утверждения:}
	\begin{itemize}
		\item $\forall f \in \mathbb{F}_2(n), \forall L \le V_n(2), \forall \vec{\epsilon} \in L:$  $d_L f(\vec{x}) = d_L f (\vec{x} \oplus \vec{\epsilon}).$ 
		
		\item $\forall f,g \in \mathbb{F}_2(n),  \forall L \le V_n(2):$ $d_L(f\oplus g)(\vec{x}) = d_L f(\vec{x}) \oplus d_L g(\vec{x})$.
		
		\item $\forall f \in \mathbb{F}_2(n), \forall \vec{\epsilon}_2, \vec{\epsilon}_2 \in V_n(2):$ $d_{\vec{\epsilon}_2 \oplus \vec{\epsilon}_2} f(\vec{x}) = d_{\vec{\epsilon}_1} f(\vec{x}) \oplus d_{\vec{\epsilon}_1} f(\vec{x} \oplus \vec{\epsilon}_1).$
		
		\item $\forall \vec{\delta} \in V_n(2): d_{\vec{\delta}} f(\vec{x}) = 0 \Leftrightarrow f \text{ --- константа.}$
		
		\item $\forall \vec{\delta} \in V_n(2): d_{\vec{\delta}} f(\vec{x})  \text{ --- константа.} \Leftrightarrow f \text{ --- аффинная функция.}$
	\end{itemize}
	
	\subsection{Гомоморфизмы (флешбэк)}
	\definition Отображение $\phi: (G, \circ) \rightarrow (H, \diamond)$ называется гомоморфизмом, если $\forall x,y \in G: ~ \phi(x \circ y) = \phi(x) \diamond \phi(y)$.
	
	\definition Биективный гомоморфизм называется изоморфизмом.
	
	\subsection{Гомоморфизмы $(V_n(p), +)$ в мультипликативную группу поля$\mathbb{C}$}
	
	Стандартный базис $V_n(p)$:
	$$
		\vec{e}_j = (0,...,0,\underbrace{1}_{j\text{-я позиция}},0,...,0), j = 1, ..., n. \\
	$$
	
	Базис пространства $F(p,n)$, разложение по которому часто используется на практике.
	
	\theorem{
			Множество всех гомоморфизмов $\phi : (V_n(p), +) \rightarrow (\mathbb{C}, \cdot)$ состоит из $p^n$ различных гомоморфизмов $\omega_{\vec{\alpha}}$, $\vec{\alpha} = (\alpha_1, ... , \alpha_n) \in V_n(p)$, каждый из которых однозначно определяется своим действием на векторах стандартного базиса $\vec{e}_j, ~j = 1,...,n,$ условием
			$$
			\omega_{\vec{\alpha}}(\vec{e}_j) = \exp\left({\frac{2 \pi i}{p} \alpha_j}\right),
			$$
			где $i$ --- мнимая единица.
		}
	\proof Так как $\phi$ --- гомоморфизм, то для $\forall \vec{\gamma} = (\gamma_1, ..., \gamma_n) \in V_n(p)$ имеем:
	$$
	\phi(\vec{\gamma}) = \phi \left( \sum_{j=1}^{n} \gamma_j \vec{e}_j \right) = \prod_{j=1}^{n} \phi (\vec{e}_j)^{\gamma_j}.
	$$
	
	$\Rightarrow$ каждый гомоморфизм определяется действием на векторах $\vec{e}_1,..., \vec{e}_n$.
	
	$\Rightarrow ~ \forall j \in {1,...,n}: ~ \phi(\underbrace{\vec{e}_j+...+\vec{e}_j}_p) = \phi(\vec{e}_j)^p = \phi(0) = 1.$
	
	$\Rightarrow$ $\phi(\vec{e}_j)$ --- корень степени $p$ из единицы и $\phi(\vec{e}_j) = \exp\left({\frac{2 \pi i}{p} k}\right)$ для некоторого $k \in {0, ..., p-1}.$
	
	$\Rightarrow$ $\phi$ есть один из $p^n$ гомоморфизмов вида $\omega_{\vec{\alpha}}$ определённых в условиях теоремы. $\blacksquare$
	
	\subsection{Аддитивные характеры}
	Пусть $
	\angles{\vec{\alpha}, \vec{\beta}} = \alpha_1 \beta_1 + ... + \alpha_n \beta_n,
	$ где "<+"> в поле $GF(p)$.
	
	$$
	\omega_{\vec{\alpha}}(\vec{\beta}) = \prod_{j=1}^{n} \left( \exp\left[{\frac{2 \pi i}{p} \alpha_j} \right]\right) ^{\beta_j} = \exp \left( \frac{2 \pi i}{p} (\alpha_1 \beta_1 + ... + \alpha_n \beta_n) \right) = \exp \left( \frac{2 \pi i}{p} \angles{\vec{\alpha}, \vec{\beta}} \right)
	$$
	
	Аддитивная группа поля $GF(p^n)$ может быть представлена как векторное пространство $(V_n(p), +)$.
	
	\definition
	Гомоморфизмы $\omega_{\vec{\alpha}}$ называются аддитивными характерами поля $GF(p^n).$
	
	\subsubsection{Основное свойство характеров}
	Обозначим комплексное сопряжение как $\omega_{\vec{\alpha}}(\vec{e}_j) = \exp\left(-{\frac{2 \pi i}{p} \angles{\vec{\alpha}, \vec{\beta}} }\right).$
	
	\definition $\ind{A} = \left\{ \begin{aligned}
		&1, ~ \text{если условие А справедливо} \\
		&0, ~ \text{если условие А несправедливо}
	\end{aligned} \right.$ называется индикатором.
	
	\statement{$\forall \vec{\alpha}, \vec{\beta} \in V_n(p)$ справедливо равенство $$
		\frac{1}{p^n} \sum_{\vec{\gamma} \in V_n(p)} \omega_{\vec{\alpha}}(\vec{\gamma}) \overline{\omega_{\vec{\beta}}(\vec{\gamma})} = \ind{\vec{\alpha} = \vec{\beta}}.
	$$
} 
\proof
\[
\begin{aligned}
	\frac{1}{p^n} \sum_{\vec{\gamma} \in V_n(p)} \omega_{\vec{\alpha}}(\vec{\gamma}) \overline{\omega_{\vec{\beta}}(\vec{\gamma})} &= \frac{1}{p^n} \sum_{\vec{\gamma} \in V_n(p)} \exp\left({\frac{2 \pi i}{p} \angles{\vec{\alpha}, \vec{\gamma}} }\right) \exp\left(-{\frac{2 \pi i}{p} \angles{\vec{\beta}, \vec{\gamma}} }\right) = \\
	& = \frac{1}{p^n} \sum_{\vec{\gamma} \in V_n(p)} \exp\left(-{\frac{2 \pi i}{p} \angles{\vec{\alpha} - \vec{\beta}, \vec{\gamma}} }\right)  \label{eq:charproof1}
\end{aligned}
\]

Если $\vec{\alpha} = \vec{\beta}$, то $\angles{\vec{\alpha}-\vec{\beta}, \vec{\gamma}} = 0$ и выражение \refeq{eq:charproof1} равно 1.

Если $\vec{\alpha} \ne \vec{\beta}$, то $\vec{\alpha} - \vec{\beta} = (\theta_1, ..., \theta_n) \ne \vec{0}$. В поле $GF(p)$ уравнение
$$
\theta_1 x_1 +...+ \theta_n x_n = c
$$
для каждого $c \in GF(p)$ будет иметь $p^{n-1}$ решений. Следовательно, если $\vec{\gamma}$ пробегает все значения из $V_n(p)$, то $\angles{\vec{\alpha} - \vec{\beta}, \vec{\gamma}}$ будет принимать каждое значение из $GF(p)$ ровно по $p^{n-1}$ раз. Следовательно
$$
\exp \left(\frac{2 \pi i}{p} \angles{\vec{\alpha} - \vec{\beta}, \vec{\gamma}}\right) = p^{n-1} \sum_{k=0}^{p-1} \exp \left(\frac{2 \pi i}{p}k \right) = p^{n-1} \frac{\exp \left( \frac{2 \pi i}{p} p\right) - 1}{\exp \left( \frac{2 \pi i}{p} \right) - 1}. ~ \blacksquare
$$

	\subsubsection{Базис пространства}
	Обозначим пространство $\mathbb{C}(p,n) = \{f ~|~ V_n(p) \rightarrow \mathbb{C}\}$.
	
	\theorem{$\{\omega_{\vec{\alpha}} ~|~ \vec{\alpha} \in V_n (p)\}$ --- базис пространства $\mathbb{C}(p,n)$.}
	\proof
	Так как число характеров равно размерности $\mathbb{C}(p,n)$, достаточно показать линейную независимость характеров.
	
	Предположим, что они зависимы. Тогда для некоторого набора $\{c_{\vec{\alpha}} ~|~ \vec{\alpha} \in V_n(p)\}$ справедливо равенство
	$$
	\sum_{\vec{\alpha} \in V_n(p)} c_{\vec{\alpha}} \omega_{\vec{\alpha}} (\vec{x}) = 0.
	$$
	
	Для каждого $\vec{\beta} \in V_n(p)$ умножим обе части равенства на $\overline{\omega_{\vec{\beta}}}$ и просуммируем по всем $\vec{x} \in V_n(p)$
	$$
	\sum_{\vec{x} \in V_n(p)} \sum_{\vec{\alpha} \in V_n(p)} c_{\vec{\alpha}} \omega_{\vec{\alpha}} (\vec{x}) \overline{\omega_{\vec{\beta}}(\vec{x})} = 0.
	$$
	
	Поменяв порядок суммирования, имеем:
	$$
	\sum_{\vec{x} \in V_n(p)} \sum_{\vec{\alpha} \in V_n(p)} c_{\vec{\alpha}} \omega_{\vec{\alpha}} (\vec{x}) \overline{\omega_{\vec{\beta}}(\vec{x})} = 
	\sum_{\vec{\alpha} \in V_n(p)} c_{\vec{\alpha}} \sum_{\vec{x} \in V_n(p)} \omega_{\vec{\alpha}} (\vec{x}) \overline{\omega_{\vec{\beta}}(\vec{x})} =
	p^n \sum_{\vec{\alpha} \in V_n(p)} c_{\vec{\alpha}} \ind{\vec{\alpha} = \vec{\beta}} = p^n c_{\vec{\beta}} = 0.
	$$
	Следовательно $c_{\vec{\alpha}} = 0$ для каждого $\vec{\alpha} \in V_n(p)$ и система линейно независима. $\blacksquare$
	
	
	\subsection{Преобразования Фурье}
	\definition
	Разложение произвольной функции $f \in \mathbb{C}(p,n)$ по базису характеров $\{\omega_{\vec{\alpha}} ~|~ \vec{\alpha} \in V_n(p)\}$:
	$$
	f(\vec{x}) = \frac{1}{p^n} \sum_{\vec{\alpha} \in V_n(p)} \tilde{w}_f(\vec{\alpha}) \omega_{\vec{\alpha}}(\vec{x})
	$$
	называется разложением в ряд Фурье.
	
	\definition
	Комплексное число $\tilde{w}_f(\vec{\alpha})$ называется коэффициентом Фурье, соответствующему набору $\vec{\alpha}$.
	
	\definition
	Отображение $\mathbb{C}(p,n) \rightarrow \mathbb{C}^n$, ставящее каждой функции в соответствие набор её коэффициентов Фурье (спектр Фурье), называется преобразованием Фурье. Также есть другие названия: преобразование Уолша-Адамара первого рода, преобразование Уолша (type I Walsh transform).
	
	\statement{Пусть $\vec{\gamma} \in V_n (p)$. Тогда
	$$
	\tilde{w}_f(\vec{\gamma}) = \sum_{\vec{\beta} \in V_n(p)} f(\vec{\beta}) \overline{\omega_{\vec{\gamma}}(\vec{\beta})}
	$$}
	
	\proof
	$$\sum_{\vec{\beta} \in V_n(p)} f(\vec{\beta}) \overline{\omega_{\vec{\gamma}}(\vec{\beta})} = \sum_{\vec{\beta} \in V_n(p)} p^{-n} \sum_{\vec{\alpha} \in V_n(p)} \tilde{w}_f (\vec{\alpha}) \omega_{\vec{\alpha}}(\vec{\beta}) \overline{\omega_{\vec{\gamma}}(\vec{\beta})} = \sum_{\vec{\alpha} \in V_n(p)} \tilde{w}_f (\vec{\alpha}) \ind{\vec{\alpha} = \vec{\gamma}} = \tilde{w}_f(\vec{\gamma}). ~\blacksquare$$
	
	\paragraph{Замечание} Далее $p = 2$.
	
	$$
	\omega_{\vec{\alpha}} (\vec{\beta}) = (-1) ^ {\angles{\vec{\alpha}, \vec{\beta}}}
	$$
	$$
	\Rightarrow ~~ \tilde{w}_f(\vec{\alpha}) = \sum_{\vec{\beta} \in V_n(2)} f (\vec{\beta}) (-1) ^ {\angles{\vec{\alpha}, \vec{\beta}}} .
	$$
	
	\hrule
	\paragraph{Замечание} В литературе иногда встречается и без коэффициента нормировки
	$$
	\tilde{w}_f (\vec{\alpha}) = 2^{-n} \sum_{\vec{\beta} \in V_n(2)} f (\vec{\beta}) (-1) ^ {\angles{\vec{\alpha}, \vec{\beta}}} .
	$$
	В таком случае:
	$$
	f(\vec{x}) = \sum_{\vec{\alpha} \in V_n(2)} \tilde{w}_f (\vec{\alpha}) \omega_{\vec{\alpha}}.
	$$ \hrule ~\\
	
	Соотношение ортогональности:
	$$
	\sum_{\vec{\gamma} \in V_n(2)} \omega_{\vec{\alpha}} (\vec{\gamma}) \omega_{\vec{\beta}} (\vec{\gamma}) = 2^{2n} \ind{\vec{\alpha} = \vec{\beta}}
	$$
	$$
	\sum_{\vec{\gamma} \in V_n(2)} (-1)^{\angles{\vec{\alpha} \oplus \vec{\beta}, \vec{\gamma}}} = 2^{2n} \ind{\vec{\alpha} = \vec{\beta}}
	$$
	
	\pagebreak
	
	\seminary{Семинар 1-4}
	\subsection{Коэффициенты Уолша-Адамара и иже с ними}
	
	\definition
	Пусть $f : \ZZ_2^n \rightarrow \mathbb{R}$, тогда $\hat{f} = (-1)^{f(x)}$ --- действительнозначный аналог булевой функции. \\
	
	
	Обозначим скалярное произведение как $\angles{a,b}$, матрицу Адамара как $H$, а коэффициенты\footnote{Они же появляются во второй лекции и имеют другие обозначения.} Уолша-Адамара второго рода как $\hat{F}_f (u)$.
	
	\statement{$$
		H \cdot \transp{H} = nE
		$$}
	
	\statement{
		$$
		\begin{aligned}
%			f &:~ \ZZ_2^n \rightarrow \ZZ_2 \\
			\hat{F}_f (u) &=\sum_{x \in \ZZ_2^n} (-1)^{\angles{u,x}} \hat{f}(x) = \sum_{x \in \ZZ_2^n} (-1)^{\angles{u,x}} (-1)^{f(x)} = \sum_{x \in \ZZ_2^n} (-1)^{\angles{u,x} \oplus f(x)}
		\end{aligned}
		$$}

	\statement{
	$$
	\sum_{u \in \ZZ_2^n} (-1)^{\angles{u,v}} = \left\{ \begin{aligned}
	&2^n, ~ v = 0; \\
	&0, ~ v \ne 0.
	\end{aligned}\right.
	$$
	}
	\proof
	\begin{enumerate}
	\item $v = 0 ~~\Rightarrow~~ \angles{u,v} = \angles{u,0} = 0 ~~\Rightarrow~~ \sum_{u \in \ZZ_2^n} (-1)^0 = 2^n$
	\item $$v \ne 0 ~~\Rightarrow~~ \sum_{\scriptsize\stack{u \in \ZZ_2^n \\ u': u_1 = 0}} (-1)^{\angles{u',v}} + \sum_{\scriptsize\stack{u \in \ZZ_2^n \\ u': u_1 = 1}} (-1)^{\angles{u',v}} = 2^\frac{n}{2} - 2^\frac{n}{2} = 0 ~~ \blacksquare$$
	\end{enumerate}
	
	\statement{$$
	(-1)^{f(x)} = \frac{1}{2^n} \sum_{u \in \ZZ_2^n} \hat{F}_f(u) \cdot (-1)^{\angles{u,x}}
	$$}
	\proof
	$$\begin{aligned}
	\frac{1}{2^n} \sum_{u \in \ZZ_2^n} \hat{F}_f(u) \cdot (-1)^{\angles{u,x}} &= 
	\frac{1}{2^n} \sum_{u \in \ZZ_2^n} \left[ \sum_{y \in \ZZ_2^n}(-1)^{\angles{y,u} \xor f(y)} \right] \cdot (-1)^{\angles{u,x}} = \\
	&= \frac{1}{2^n}  \sum_{y \in \ZZ_2^n} \sum_{u \in \ZZ_2^n} (-1)^{\angles{y,u}} \cdot (-1)^{f(y)} \cdot (-1)^{\angles{u,x}} = \\ &= 
	\frac{1}{2^n}  \sum_{y \in \ZZ_2^n} \cdot (-1)^{f(y)} \sum_{u \in \ZZ_2^n} (-1)^{\angles{y,u}} \cdot (-1)^{\angles{u,x}} = \\
	&= \frac{1}{2^n}  \sum_{y \in \ZZ_2^n} \cdot (-1)^{f(y)} \sum_{u \in \ZZ_2^n} (-1)^{\angles{y,u} \xor \angles{u,x}} = \\ &= 
	\frac{1}{2^n} \sum_{y \in \ZZ_2^n} \cdot (-1)^{f(y)} \sum_{u \in \ZZ_2^n} (-1)^{\angles{u, x \xor y}} = \\
	&= \frac{1}{2^n} \cdot (-1)^{f(x)} \cdot 2^n + \frac{1}{2^n}  \sum_{\scriptsize \stack{y \in \ZZ_2^n\\y\ne x}} (-1)^{f(y)} \underbrace{\sum_{u \in \ZZ_2^n} (-1)^{\angles{u, x \xor y}}}_{=0} = \\ &= (-1)^{f(x)} 
	\end{aligned}
	$$
	
%	\seminary{Семинар 2}
	\subsection{Теорема Руппеля}
	
	$$
	P\{f(x) = \angles{u,x}\} = \frac{2^n - d(f(x), \angles{u,x})}{2^n} = 1 - \frac{d(f(x), \angles{u,x})}{2^n} = 1 - \frac{2^n - \hat{F}_f (u)}{2^{n+1}} =
	$$
	$$
	= \frac{1}{2} + \frac{\hat{F}_f (u)}{2^{n+1}} =
	\vline \begin{matrix}
		\text{Обозначим} \\
		\frac{\hat{F}_f(u)}{2^n} = \hat{S}_f(u)
	\end{matrix} \vline
	=  \frac{1}{2} + \frac{\hat{S}_f (u)}{2}
	$$
	
	$$
	P\{f(x) = \angles{u, x} \oplus 1\} = \frac{1}{2} - \frac{\hat{S}_f (u)}{2}
	$$
	
	
	\definition
	Линейный статистический аналог булевой функции $f(x)$ --- это такая линейная функция $\angles{u,x}$, что расстояние $d(f(x), \angles{u,x})$ является минимальным.

	\theorem[Руппеля]{Пусть $u : |\hat{F}_f (n)|$ --- $\max$, тогда $\angles{u,x}\oplus c$ --- линейный статаналог для $f(x)$. Причём 
	$$
	\left\{
	\begin{aligned}
		c & = 0, \text{ если } \hat{F}_f (u) > 0, \\
		c & = 1, \text{ если } \hat{F}_f (u) < 0.
	\end{aligned}
	\right.
	$$}


%	\seminary{Семинар 3}
%	\hl{Семинар про бент-функции}
	
%	\seminary{Семинар 4}
	\subsection{Преобразование Мёбиуса}
	\paragraph{Пример и утверждение}
	$f = 10011101$
	$$
	f(\vec{x}= [x_1, x_2, x_3]) = a_0 \xor a_1 x_1 \xor a_2 x_2 \xor a_3 x_3 \xor a_{12} x_1 x_2 \xor a_{13} x_1 x_3 \xor a_{23} x_2 x_3 \xor a_{123} x_1 x_2 x_3
	$$
	$$\begin{aligned}
		a_0 &= 1 = f(000) \\
		a_3 &= 0 \xor a_0 = f(001) \xor a_0 = f(001) \xor f(000) \\
		a_2  = a_{010} &= f(010) \xor f(000) \\
		a_{23} =  a_{011} & = f(011) \xor f(010) \xor f(001) \xor f(000) \\
		...\\
		a_{23} x_2 x_3 &: ~~ x_2 x_3 \Rightarrow x_1^0 x_2^1 x_3^1 \Rightarrow \vec{\sigma} = (011) \\
		\Rightarrow & a_{\vec{\sigma}} = \bigoplus_{\vec{\alpha} \le \vec{\sigma}} f(\vec{\alpha})
	\end{aligned}
	$$
	\proof
	\begin{enumerate}
		\item База индукции: $a_0 \Rightarrow \vec{\sigma} = \vec{0} \Rightarrow x_1^0 ... x_n^0, ~ a_{\vec{\sigma}} = f(\vec{\sigma}) = \bigoplus_{\vec{\alpha} \le \vec{\sigma}} f(\vec{\alpha}) $. Для $a_1 = a_{10...0} = f(10...0)\xor a_0$. Аналогично для $a_2,...,a_n$.
	
	\item Пусть для $\forall \vec{\sigma} \in V_n(2), \weight{\vec{\sigma}} = k: ~~ a_{\vec{\sigma}} = \bigoplus_{\vec{\alpha} \le \vec{\sigma}} f(\vec{\sigma})$.
	
	\hl{На индукционном переходе мы застряли и взялись за другую задачу. Поэтому текст ниже должен расцениваться как потенциально неверный, ибо был дописан из собственных размышлений автора.}
	
	\item Тогда для $\vec{\nu} \in V_n(2), \weight{\vec{\nu}} = k+1$:
	%Пусть $\vec{\mu}: ~ d(\nu, \mu) = 1, \weight{\mu} = k$, $i$ --- номер разряда, где $\mu_i = 0, ~ \nu_i = 1$.
	$$
	f(x_1, ..., x_n) = a_0 \xor a_1 x_1 \xor ... \xor a_n x_n \xor a_{12} x_1 x_2 \xor ... \xor a_{n-1, n} x_{n-1} x_n \xor ... \xor a_{1...n} x_1 ... x_n
	$$
	
	Выразим $a_{i_1...i_{k+1}} x_{i_1} ... x_{i_{k+1}}$, которое соответствует вектору $\vec{\nu}$. Получим следующее:
	$$
	a_{i_1...i_{k+1}} x_{i_1} ... x_{i_{k+1}} =  f(x_1, ..., x_n) \xor a_0 \xor a_1 x_1 \xor ... \xor a_n x_n \xor a_{12} x_1 x_2 \xor ... \xor a_{n-1, n} x_{n-1} x_n \xor ...
	$$
	
	Чтобы получить значение коэффициента подставим в $ x_{i_1} ... x_{i_{k+1}}$ единицы, а в остальные переменные нули. Тогда в правой части останутся конъюнкции содержавшие только $ x_{i_1} ... x_{i_{k+1}}$, коэффициент $a_0$ и функция от вектора $\vec{\nu}$.
	$$
	a_{i_1...i_{k+1}} =  f(x_1, ..., x_n) \xor (a_{i_2 ... i_{k+1}} \xor a_{i_1 i_3 ... i_{k+1}} \xor ... \xor a_{i_1 ... i_k}) \xor (a_{i_3 ... i_{k+1}} \xor ... \xor a_{i_1 ... i_{k-1}}) \xor ... \xor a_0
	$$
	
	Заменим обозначения на бинарные вектора (как в примере выше) и свернём суммирование в скобках:
	$$
	a_{\vec{\nu}} = f(\vec{\nu}) \xor
	\bigoplus_{\vec{\sigma}: {\tiny \stack{\vec{\nu} > \vec{\sigma} \\
				\weight{\vec{\sigma}} = k}} } a_{\vec{\sigma}} \xor
	\bigoplus_{\vec{\sigma}: {\tiny \stack{\vec{\nu} > \vec{\sigma} \\
				\weight{\vec{\sigma}} = k-1}} } a_{\vec{\sigma}} \xor
	...
	\xor
	\bigoplus_{\vec{\sigma}: {\tiny \stack{\vec{\nu} > \vec{\sigma} \\
				\weight{\vec{\sigma}} = 1}} } a_{\vec{\sigma}} \xor
	a_{\vec{0}}
	$$
	
	
\end{enumerate}

	\pagebreak
	\lection{Лекция 2}
	\subsection{Преобразование Фурье (продолжение)}
	
	\paragraph{Задача}
	$\weight{\vec{\alpha} \oplus \vec{\beta}} = \weight{\vec{\alpha}} + \weight{\vec{\beta}} - 2 \weight{\vec{\alpha} \cdot \vec{\beta}}$
	
	\statement{Пусть $f \in \mathbb{F}_2(n)$. Тогда $$\tilde{w}_f (\vec{\alpha}) = \left\{ \begin{aligned}
			&\weight{f}, ~\iftext~ \vec{\alpha} = \vec{0} \\
			&\weight{f(\vec{x}) \oplus \angles{\vec{\alpha}, \vec{x}}} - 2^{n-1}, ~\iftext~ \vec{\alpha} \ne \vec{0}
		\end{aligned} \right.$$}
	
	\proof
	Пусть 
	$$
	\begin{aligned}
	\Omega_0 &= \{\vec{\beta} \in V_n(2) ~|~ f(\vec{\beta}) = 1 \text{ и } \angles{\vec{\alpha}, \vec{\beta}} = 0\}, \\
	\Omega_1 &= \{\vec{\beta} \in V_n(2) ~|~ f(\vec{\beta}) = 1 \text{ и } \angles{\vec{\alpha}, \vec{\beta}} = 1\}.	
	\end{aligned}
	$$
	
	Тогда
	$$
	\tilde{w}_f (\vec{\alpha}) = \sum_{\vec{\beta} \in V_n(2)} f(\vec{\beta}) (-1)^{\angles{\vec{\alpha}, \vec{\beta}}} = \left(\sum_{\vec{\beta} \in \Omega_0} 1 - \sum_{\vec{\beta} \in \Omega_1} \right)
	$$
	Необходимо найти мощности $\Omega_0, \Omega_1$. Из их определения
	$$\begin{aligned}
		|\Omega_0| &= \weight{f(\vec{x}) \cdot (\angles{\vec{\alpha},\vec{x}} \oplus 1)} = 
		\weight{f(\vec{x}) \cdot \angles{\vec{\alpha}, \vec{x}} \oplus f(\vec{x})} = \\ 
		& =\weight{f(\vec{x}) \cdot \angles{\vec{\alpha}, \vec{x}}} + \weight{f(\vec{x})} - 2 \weight{f(\vec{x}) \cdot \angles{\vec{\alpha}, \vec{x}} \cdot f(\vec{x})} = \\
		& = \weight{f(\vec{x})} - \weight{f(\vec{x}) \cdot \angles{\vec{\alpha}, \vec{x}}},
	\end{aligned}
	$$
	$$
	|\Omega_1| = \weight{f(\vec{x}) \cdot \angles{\vec{\alpha}, \vec{x}}}.
	$$
	
	Отсюда следует, что
	$$
	\tilde{w}_f (\vec\alpha) = (\weight{f(\vecx)} - 2\weight{f(\vecx) \cdot \angles{\vec\alpha, \vecx}}) = \underbrace{\weight{f(\vecx)} + \weight{\angles{\vec\alpha, \vecx}} - 2 \weight{f(\vecx) \cdot \angles{\vec\alpha, \vecx}}}_{=\weight{f(\vecx) \xor \angles{\vec\alpha, \vecx}}} - \weight{\angles{\vec\alpha, \vecx}}
	$$
	
	Тогда возможны два случая:
	\begin{itemize}
	\item Если $\vec\alpha = \vec0 ~~\Rightarrow~~ \weight{\angles{\vec\alpha, \vecx}} = \vec0 ~~\Rightarrow~~ \tilde{w}_f(\vec\alpha) = \weight{f}$
	\item Если $\vec\alpha \ne \vec0$, то из равновероятности $\angles{\vec\alpha, \vecx}$ 
	$\tilde{w}_f(\vec\alpha) = \weight{f(\vecx) \xor \angles{\vec\alpha, \vecx}} - 2^{n-1}$. $\blacksquare$
	\end{itemize}
	
	\subsection{Коэффициенты Уолша-Адамара}
	\definition Лекционное обозначение действительнозначного аналога булевой функции
	$$
	\delta_f (x)= (-1)^{f(x)}, \text{ для } f:\ZZ_2^n \rightarrow \ZZ_2.
	$$
	
	Коэффициенты Фурье для $\delta_f$ называются также коэффициентами Уолша-Адамара второго рода:
	$$
	w_f(\vec\alpha) = \sum_{\vec\beta \in V_n(2)} (-1)^{f(\vec\beta)\xor \angles{\vec\alpha, \vec\beta}}.
	$$
	
	\hrule
	\paragraph{Замечание}
	В литературе встречается и с коэффициентом нормировки
	$$
	w_f(\vec\alpha) = 2^{-n} \sum_{\vec\beta \in V_n(2)} (-1)^{f(\vec\beta)\xor \angles{\vec\alpha, \vec\beta}}.
	$$
	$$
	\Rightarrow ~~ \delta_f (\vecx) = \sum_{\vec\alpha \in V_n(2)} w_f(\vec\alpha) \omega_{\vec\alpha}.
	$$
	\hrule
	
	\paragraph{Задача} Пусть $f, g \in \FF_2(n)$. Доказать:
	\begin{itemize}
	\item $\delta_{f \xor g} = \delta_f \delta_g$;
	\item $2 \delta_{f g} = 1 + \delta_f + \delta_g - \delta_f \delta_g$.
	\end{itemize}
	Позсказка: $\delta_f(\vecx) = 1 - 2f(\vecx)$.

	\statement{
	Пусть $f \in \FF_2(n)$. Тогда
	$$
	w_f(\vec\alpha) = 2^n - 2 \weight{f(\vecx) \xor \angles{\vec\alpha, \vecx}}
	$$
	}
	\proof Д/З $~\cap\_\cap$

	\corollary{
	Пусть $f \in \FF_2(n)$. Тогда
	\begin{itemize}
	\item если $\vec\alpha \ne \vec0$ $w_f(\vec\alpha) = -2 \tilde{w}_f (\vec\alpha)$;
	\item если $\vec\alpha = \vec0$ $w_f(\vec\alpha) = 2^n -2 \tilde{w}_f (\vec\alpha)$.
	\end{itemize}
	}
	
	\statement{
	Пусть $f \in \FF_2(n)$. Тогда
	$$
	\sum_{\vec\alpha \in V_n(2)} w_f(\vec\alpha) = 2^n (-1)^{f(0)}.
	$$
	}
	
	\proof
	$$
        \begin{aligned}
	\sum_{\vec\alpha \in V_n(2)} w_f(\vec\alpha) &=
          \sum_{\vec{\alpha} \in V_n(2)} \sum_{\vec{\beta} \in V_n(2)}
          (-1)^{f(\vec\beta)\xor \angles{\vec\alpha, \vec\beta}} =
          \sum_{\vec{\beta} \in V_n(2)} (-1)^{f(\vec\beta)}
          \sum_{\vec{\alpha} \in V_n(2)} (-1)^{\angles{\vec\alpha,
                                                       \vec\beta}}\\
                                                &= \sum_{\vec{\beta} \in V_n(2)}
          (-1)^{f(\vec\beta)} 2^n \ind{\vec\beta = \vec0} = 2^n
          (-1)^{f(0)}.
        \end{aligned}
        ~~~ \blacksquare
	$$
	
	\theorem[Равенство Парсеваля]{Пусть $f \in \FF_2(n)$. Тогда
	$$
	\sum_{\vec{\alpha} \in V_n(2)} \left(w_f(\vec{\alpha})\right)^2 = 2^{2n}.
	$$
	}
	
	\theorem[Соотношение ортогональности]{
	$$
	\sum_{\vec{\alpha} \in V_n(2)} w_f(\vec\alpha) w_f (\vec\alpha \xor \vec\beta) = 2^{2n} \cdot \ind{\vec\beta = \vec0}.
	$$
	}
	
	\statement{
	Пусть $h \in \FF_2 (n+m)$ такова, что $h(\vecx, \vecy) = f(\vecx) \xor g(\vecy)$ для некоторых $f \in \FF_2(n)$, $g \in \FF_2(m)$, $\vecx \in V_n(2), y\in V_m(2)$. Тогда
	$$
	\forall \vec\alpha \in V_n(2), \forall \vec\beta \in V_m(2): ~~ w_h(\vec\alpha, \vec\beta) = w_f (\vec\alpha) w_g (\vec\beta).
	$$
	}
	
	\subsection{Расстояние Хемминга}
	\definition
	Расстояние (Хемминга) $\hamming(f, g)$ между булевыми функциями $f,g \in \FF_2(n)$ есть $$
	\hamming(f,g) = |\{\vec\alpha \in V_n(2): ~ f(\vec\alpha) \ne g(\vec\alpha)\}|,
	$$$$\Downarrow$$$$
	\hamming (f,g) = \weight{f\xor g}.
	$$
	
	\definition Расстояние $\hamming (f,g)$ от функции $f \in \FF_2(n)$ до множества $C \subseteq \FF_2(n)$ определяется как$$
	\hamming(f, C) = \min_{g \in C(n)} \hamming(f,g).
	$$
	
	\subsection{Нелинейность}
	\definition Нелинейностью $\nonlinty{f}$ называется наименьшее расстояние от функции $f \in \FF_2 (n)$ до множества всех аффинных функций $$
	\nonlinty{f} = \hamming (f, \mathcal{AL}_n).
	$$
	
	Из равенства
	$$
	wf_(\vec\alpha) = 2^n - 2 \weight{f(\vecx)\xor \angles{\vec\alpha, \vecx}}
	$$$$\Downarrow$$$$
	\nonlinty{f} = 2^{n-1} - \frac{1}{2} \max_{\vec\alpha \in V_n(2)} w_f (\vec\alpha).
	$$
	
	\paragraph{Пример}
	$
	f\in \FF_2(3)
	$ $$
	f(x_1,x_2,x_3) = 1 \xor x_1 \xor x_2 \xor x_2 x_3 \xor x_1 x_2 x_3 ~~\Rightarrow~~ \vec{f} = (1101~0011).
	$$
	Коэффициенты Уолша-Адамара: $(-2,2,2,-2, -2,2,-6,-2)$.$$
	\Rightarrow |w_f (6)| = 6 ~~ \Rightarrow ~~\text{ближайшая аффинная функция } g = 1 \xor x_1 \xor x_2.
	$$
	$$
	\hamming (f,g) = 1.
	$$
	
	\subsection{Бент-функции}
	\theorem{Для каждой $f \in \FF_2 (n)$ справедливо неравенство
	$$
	\nonlinty{f} \le 2^{n-1} - 2^{\frac{n}{2} -1}.
	$$}

	\definition Функция $f \in \FF_2 (n)$ называется бент-функцией, если $$\forall \vec\alpha \in V_n(2): ~~w_f(\vec\alpha) \in \{-2^{\frac{n}{2}}, 2^{\frac{n}{2}}\}.
	$$
	
	\statement{Очевидно, что если $f$ --- бент-функция, то:
	\begin{itemize}
	\item $\forall \vec\alpha \in V_n(2):~~ \left(w_f(\vec\alpha)\right)^2 = 2^n$
	\item Бент-функции существуют только для чётных $n$.
	\end{itemize}
	}

	\paragraph{Пример}
	\begin{itemize}
	\item $f(x_1, x_2) = x_1 x_2$;
	\item $f(x_1,,x_4) = 1 \xor x_1 x_2 \xor x_1 x_3 \xor x_1 x_4 \xor x_2 x_3 \xor x_2 x_4 \xor x_3 x_4$.
	\end{itemize}	
	
	\subsection{Построение бент-функции}
	\theorem{Пусть $h \in \FF_2 (n+m)$ такова, что $h(\vecx, \vecy) = f(\vecx) \xor g(\vecy)$ для некоторых $f \in \FF_2(n)$, $g \in \FF_2(m)$, $\vecx \in V_n(2), y \in V_m(2)$. Тогда$$
	h \text{ --- бент-функция} ~~ \Leftrightarrow ~~ f,g \text{ --- бент-функции}.
	$$
	}
	\corollary{Функция $f \in \FF_2 (2k),~ k \ge 1,~ f(\vecx) = x_1 x_2 \xor x_3 x_4 \xor ... \xor x_{2k-1} x_{2k}$ является бент-функцией.}
	
	\theorem{Пусть $\pi \in S(V_n),~ g\in \FF_2 (n),~ \vecx,\vecy \in V_n(2)$.\\ Тогда функция $f \in \FF_2 (2n), ~ f(\vecx, \vecy) = \angles{\pi(\vecx), \vecy} \xor g(\vecx)$ есть бент-функция.
	}
	
	\subsection{Фильтрующие генераторы}
	На рисунке \ref{fig:filtering_gen} изображена общая схема фильтрующего генератора. ЛРСОС (он же LFSR, он же сдвиговый регистр с линейной обратной связью) выдаёт свои разряды на вход функции $f$.
	\begin{figure}
		\centering
		\begin{tikzpicture}
		[node distance = 2cm, on grid]
		\node[draw,rounded corners, minimum width=5cm,align=center] (lfsr) at (0, 0) {ЛРСОС};
		\node[isosceles triangle,isosceles triangle stretches,shape border rotate=270, minimum width=5cm,draw,rounded corners,below of=lfsr] (f) {f};

		\node[align=center,rounded corners,below of=f] (outsup) {~};
		\node[align=center,rounded corners,right of=outsup] (out) {$z_1, z_2...z_l$};

		\draw[rounded corners,->] (lfsr.east) -- (5,0) -- (5,1) -- (-5,1) -- (-5,0) -- (lfsr.west);
		\draw[rounded corners,double=white, line width=1 pt,->] (lfsr.south) -- (f.north);
		\draw[rounded corners,->] (f.south) -- (outsup.center) -- (out.west);
		\end{tikzpicture}
		\caption{Общая схема фильтрующего генератора}
		\label{fig:filtering_gen}
	\end{figure}
	
	Линейная рекуррентная последовательность (ЛРП):
	$$
	x_{n+i} = x_i + \sum_{j=1}^{n-1} c_j x_{j+1}, ~i = 1,2,...
	$$

	Преобразование ЛРСОС $\delta$ (функция переходов состояний)
	$$
	\delta(\vecx) = \left( x_2, x_3, ..., x_n, x_1 + \sum_{j=2}^{n} c_{j-1} x_j \right),
	$$ где: \begin{itemize}
	\item[--] $\vecx = (x_1, x_2,...,x_n) \in \ZZ_2^n$,
	\item[--] $\delta: ~ \ZZ_2^n \rightarrow \ZZ_2^n$,
	\item[--] $c_1, ..., c_{n-1} \in \ZZ_2$ --- коэффициенты обратной связи.
	\end{itemize}
	Преобразованию соответствует сопровождающая $(n \times n)$-матрица
	$$
	\mx{a} = \begin{pmatrix}
	0&1&0&0&...&0 \\
	0&0&1&0&...&0 \\
	...&...&...&...&...&... \\
	0&0&0&0&...&1 \\
	1&c_1&c_2&c_3&...&c_{n-1}
	\end{pmatrix} = \begin{pmatrix}
	\vec0^{\downarrow} & \mx{E}_{n-1} \\
	1 & \vec{c}
	\end{pmatrix}.
	$$
	ЛРСОС(ЛРС) с точки зрения автоматной модели --- автономный автомат $\mathbb{A} = (\ZZ_2^n, \ZZ_2, \delta, f)$:
	\begin{itemize}
	\item $\ZZ_2^n$ --- множество состояний (все заполнения ЛРСОС);
	\item $\delta: \ZZ_2^n \rightarrow \ZZ_2^n$ --- функция переходов;
	\item $f: \ZZ_2^n \rightarrow \ZZ_2$ --- функция выходов;
	\item $x^{(0)} \in \ZZ_2^n$ --- начальное состояние.
	\end{itemize}

	Фильтрующая функция задана многочленом Жегалкина ($\deg f > 1$):$$
	f(x_1,...,x_n) = b_0 + \sum_{r\in \{1,...,n\}} \sum_{1 \le i_1 < ... < i+r \le n} b_{i_1, ..., i_r} x_{i_1} ... x_{i_r}.
	$$
	j-й знак гаммы$$
	z_j = f\left(\delta^{j-1}(x)\right) = f\left(\mx{a}^{j-1}x\right), ~ j = 1,2,...
	$$

	\theorem{Максимальный период ЛРС длины $n$ над $\FF_q$ равен $q^n - 1$}
	
	\property{Если ЛРС максимального периода и f --- сбалансированная функция, то период выходной последовательности фильтрующего генератора также максимален и равен $q^n - 1$}\\

	Фильтрующая функция должна обеспечивать:\begin{itemize}
	\item хорошие статистические свойства $t$-грамм гаммы;
	\item большую линейную сложность.
	\end{itemize}
	
	Линейная сложность периодической последовательности --- степень её минимального многочлена.
	Она же --- минимальная длина линейного регистра сдвига с обратной связью, порождающего данную последовательность
	

	\seminary{Семинар 5-6}
	\paragraph{Летучка} Найти вес функции и определить является ли она бент-функцией.
	\begin{enumerate}
		\item $f(x_1,...,x_8) = x_1 x_3 x_4 + x_2 x_6 + x_5 + x_7 x_8$
		\item $f(x_1,...,x_6) = x_1 x_4 x_5 + x_2 x_3 x_5$
		\item $f(x_1,...,x_10)= x_1 x_2 x_4 x_9 x_{10} + x_1 x_2 x_3 + x_1 x_2 x_5 x_6 x_7 x_8$
	\end{enumerate}
	
	\paragraph{Решение варианта №3}
	Вынесем $x_1, x_2$ за скобки, тогда
	$$
	f(x_1,...,x_10)= x_1 x_2 x_4 x_9 x_{10} \xor x_1 x_2 x_3 \xor x_1 x_2 x_5 x_6 x_7 x_8 = 
	x_1 x_2 (x_4 x_9 x_{10} \xor x_3 \xor x_5 x_6 x_7 x_8)
	$$

	Тогда 
	$$
	\begin{aligned}
		h(x_1 x_2) &= x_1 x_2 & \Rightarrow & \weight{h} = 1 \cdot 2^8 = 256\\ 
		g(x_4 x_9 x_{10} x_3 x_5 x_6 x_7 x_8) &= x_4 x_9 x_{10} \xor x_3 \xor x_5 x_6 x_7 x_8 & \Rightarrow & \weight{g}= 2^7 \cdot 2^2 = 2^9 \\
	\end{aligned}
	$$
	%TODO: закончить семинар 5-6

	\pagebreak

	\lection{Лекция 3}
	\subsection{Фильтрующий генератор}
	\property{При $q = 2$ верхняя граница линейной сложности гаммы фильтрующего генератора равна \[ \sum_{i=0}^{deg f} \binom{n}{i} \]}
	Получение точных нижних оценок -- часто связано со значительными математическими трудностями
	-- Rueppel R. A. Analysis and design of stream ciphers. Berlin: Springer, 1986
	\theorem{Линейная сложность совпадает с размерностью векторного пространства, порождённого множеством всех функций на выходе фильтрующего генератора.}
	\proof (Необходимость)
	Пусть $l$ -- линейная сложность. Тогда $\forall j > l$ функция \[ z_j = z_j(x) = f(\delta^{j-1}(x)) \] над $\mathbb{F}_q$ линейно выражается через первые $l$ функций $z_1, ..., z_l$

	$L = \langle z_1(x), ..., z_l(x) \rangle$ -- линейное пространство на выходе фильтрующего генератора \[ dim L \le l ~~~\blacksquare \]
	\proof (Достаточность)
	Линейная сложность не может быть больше размерности пространства $L$, т.е. \[ dim L \ge l. \]
	Значит, $l = dim L ~~~\blacksquare$
	
	\property{$\dim L$ равна такому наибольшему числу $l \in L$ что
	\begin{itemize}
	\item первые $l$ функций $z_1, ..., z_l$ линейно независимы;
	\item $z_{l+1}$ --- линейная комбинация $z_1, ..., z_l$.
	\end{itemize}
	}
	
	\paragraph{Задача} Доказать для каждого $x \in \mathbb{F}_q^n$ равенство \[ z_j(x) = f(\delta^{j-1}(x)) = z_{j-1}(\delta(x)), ~~ j = 2,3,... \]
	
	\subsubsection{Задача о нахождении линейной сложности ФГ} Найти линейную сложность фильтрующего генератора над $\mathbb{F}_2$.
	
	$$
	n = 3, ~ \delta: \mathbb{F}_2^3 \rightarrow \mathbb{F}_2^3, ~ f: \mathbb{F}_2^3 \rightarrow \mathbb{F}_2^3, ~~
	\begin{aligned}
		\delta(x_1, x_2, x_3) &= (x_2, x_3, x_1 + x_2),\\
		f(x_1, x_2, x_3) &= x_1x_2 + x_1x_3 + x_2x_3.
	\end{aligned}
	$$
	
	Функция обратной связи ЛРС \[ b(x_1, x_2, x_3) = x_1 + x_2. \]
	\paragraph{Этап 1}
	\begin{enumerate}
	\item Найти характеристический многочлен ЛРС
	\item Установить его примитивность
	\item Найти период
	\end{enumerate}
	\paragraph{Решение}
	\begin{enumerate}
	\item Характеристический многочлен ЛРС -- $x^3 + x + 1$. Примитивен над $\mathbb{F}_2.$
	\item ЛРС максимального периода $2^3 - 1 = 3$
	\item $7$ -- простое $\implies$ период гаммы фильтрующего генератора равен $7$.
	\end{enumerate}
	\paragraph{Этап 2}
	Находим все функции \[ z_j(x) = f(\delta^{j-1}(x)) \] на периоде фильтрующего генератора, применяя равенство \[ z_j(x) = f(\delta^{j-1}(x)) = z_{j-1}(\delta(x)), j = 2,3,... \]
	Получаем
	$$
	\begin{aligned}
		z_1 &= f(x) & &= x_1x_2 + x_1x_3 + x_2x_3,\\
		z_2 &= z_1(\delta(x)) &= x_2x_3 + x_2(x_1 + x_2) &= x_1x_2 + x_1x_3 + x_2,\\
		z_3 &= z_2(\delta(x)) &= x_2x_3 + x_2(x_1 + x_2) + x_3 &= x_1x_2 + x_2x_3 + x_2 + x_3,\\
		z_4 &= z_3(\delta(x)) &= x_2x_3 + x_3(x_1 + x_2) + x_1 + x_2 + x_3 &= x_1x_3 + x_1 + x_2 + x_3,\\
		z_5 &= z_4(\delta(x)) &= x_2x_3 + x_4(x_1 + x_2) + x_2 + x_3 + x_1 + x_2 &= x_1x_2 + x_1 + x_2 + x_3,\\
		z_6 &= z_5(\delta(x)) & &= x_2x_3 + x_1 + x_3,\\
		z_6 &= z_6(\delta(x)) & &= x_1x_3 + x_2x_3 + x_1.
	\end{aligned}
	$$
	
	Проверить: $z_8 = z_7(\delta(x)) = z_1$
	\paragraph{Этап 3}
	Найдём максимальное линейное независимое подмножество \[ {z_1, ..., z_l} \subseteq {z_1, ..., z_7}. \]
	Используем общий вид квадратичной функции от 3-х переменных:
	
	$$
	\begin{aligned}
		\phi(0) &= 0,\\
		\phi(x_1, x_2, x_3) &= c_1x_1 + c_2x_2 + c_3x_1x_2 + c_4x_3 + c_5x_1x_3 + c_6x_2x_3.
	\end{aligned}
	$$
	
	Требуется найти $l = \rank M$. $i$-я строка $M$ есть коэффициенты многочлена Жегалкина функции $z_i(x), i = 1, ..., 6$
	
	$$
	\begin{pmatrix}
		0 & 0 & 1 & 0 & 1 & 1 \\
		0 & 1 & 1 & 0 & 1 & 0 \\
		0 & 1 & 1 & 1 & 0 & 1 \\
		1 & 1 & 0 & 1 & 1 & 0 \\
		1 & 1 & 1 & 1 & 0 & 0 \\
		1 & 0 & 0 & 1 & 0 & 1 \\
	\end{pmatrix}
	\begin{matrix}
		\rightarrow & z_1 \\
		\rightarrow & z_2 \\
		\rightarrow & z_3 \\
		\rightarrow & z_4 \\
		\rightarrow & z_5 \\
		\rightarrow & z_6 \\
	\end{matrix} \sim
	\begin{pmatrix}
		1 & 1 & 1 & 1 & 0 & 0 \\
		1 & 1 & 0 & 1 & 1 & 0 \\
		1 & 0 & 0 & 1 & 0 & 1 \\
		0 & 1 & 1 & 1 & 0 & 1 \\
		0 & 1 & 1 & 0 & 1 & 0 \\
		0 & 0 & 1 & 0 & 1 & 1 \\
	\end{pmatrix} \sim
	\begin{pmatrix}
		1 & 1 & 1 & 1 & 0 & 0 \\
		0 & 1 & 0 & 0 & 1 & 1 \\
		0 & 0 & 1 & 0 & 1 & 0 \\
		0 & 0 & 0 & 1 & 0 & 0 \\
		0 & 0 & 0 & 0 & 1 & 1 \\
		0 & 0 & 0 & 0 & 0 & 1 \\
	\end{pmatrix}
	$$
	
	Получаем $l = \rank M = 6$.
	
	\paragraph{Задача} Найти представление $z_7$ через $z_1, ..., z_6$

	\subsection{Сильно равновероятные двоичные функции}
	Условие сбалансированности двоичной функции: $f: V_n(2) \rightarrow \mathbb{F}_2$:
	\begin{itemize}
	\item $\forall \gamma \in \mathbb{F}_2$ уравнение \[ f(x_1, ..., x_n) = \gamma \] имеет ровно $2^{n-1}$ решений относительно $(x_1, ..., x_n) \in \mathbb{F}_2^n$
	\end{itemize}

	В ряде методов криптоанализа возникает система уравнений:

	$$
	\begin{aligned}
		f(x_1, ..., x_n) &= \gamma_1,\\
		f(x_2, ..., x_{n+1}) &= \gamma_2,\\
			...\\
		f(x_1, x_{i+1}, ..., x_{n+i-1}) &= \gamma_i,\\
		...\\
		f(x_m, x_{m+1}, ..., x_{n+m-1}) &= \gamma_m.
	\end{aligned}
	$$
	
	$m \ge 1$, $m$ -- число знаков гаммы

	Выходная $m$-грамма $(\gamma_1, ..., \gamma_m) \in V_m(2)$ содержит в себе следующую информацию:
	\begin{itemize}
		\item о функции $f$
		\item о входе $x_1, ..., x_n$\\ -- Данная информация может являться основой для разработки статистических методов криптоанализа.
	\end{itemize}

	\definition Функция $f$ называется
	\begin{itemize}
		\item $m$-равновероятной, если $\forall (\gamma_1, ..., \gamma_m) \in V_m(2)$ система уравнений (1) \[ f(x_i, x_{i+1}, ..., x_{n+i-1}) = \gamma_i, i = 1, ..., m \] имеет ровно $2^{n-1}$ решений относительно неизвестных $x_1, ..., x_{n+m-1}$.
		\item сильно равновероятной, если $f$ является $m$-равновероятной $\forall m \in \mathbb{N}$.\\ -- Сильная равновероятность $f$ означает равновероятность $\forall m \in \mathbb{N}$ выходной $m$-граммы $(\gamma_1, ..., \gamma_m)$, удовлетворяющей системе (1) \[ f(x_i, x_{i+1}, ..., x_{n+i-1}) = \gamma_i, i = 1, ..., m \] и получаемой на случайном равновероятном и независимом входе $x_1, ..., x_{n+m-1}$.
	\end{itemize}

	\definition Вектор $(\gamma_1, ..., \gamma_m) \in V_m(2)$ называется запретом функции $f: V_n(2) \rightarrow \mathbb{F}_2$, если система (1) \[ f(x_i, x_{i+1}, ..., x_{n+i-1}) = \gamma_i, i = 1, ..., m \] не имеет решений.

	\theorem[Сумароков]{Функция $f: V_n(2) \rightarrow \mathbb{F}_2$ не имеет запретов $\Leftrightarrow$ она сильно равновероятна.}

	\paragraph{Задачи}
	\begin{itemize}
		\item Если функция $f: V_n(2) \rightarrow \mathbb{F}_2$ представима в виде \[ f(x_1, ..., x_n) = g(x_1, ..., x_{n-1}) + x_n \] для некоторой функции $g: V_{n-1}(2) \rightarrow \mathbb{F}_2$, то $f$ сильно равновероятна.
		\item Если функция $f: V_n(2) \rightarrow \mathbb{F}_2$ представима в виде \[ f(x_1, ..., x_n) = g(x_1, ..., x_{n-1}) + x_n \] для некоторой функции $g: V_{n-1}(2) \rightarrow \mathbb{F}_2$, то $\forall x_1, ..., x_{n-1} \exists ! x_n, ..., x_{n+m-1}$, при которых справедлива система (1).
		\item Доказать следующие свойства:
			\begin{itemize}
			\item Функции \[ f(x_1, ..., x_n) и g(x_1, ..., x_n) = f(x_n, x_{n-1}, ..., x_1) \] одновременно имеют запреты или не имеют
			\item Функции \[ f(x_1, ..., x_n) и g(x_1, ..., x_n) = f(x_1 \otimes 1, ..., x_n \otimes 1) \] одновременно имеют запреты или не имеют
			\end{itemize}
	\end{itemize}

	\statement{Свойство сильной равновероятности инвариантно относительно действия группы $G$, порождённой следующими преобразованиями на $\mathbb{F}_n$:
		\begin{itemize}
		\item Инверсия функции
		\item Инверсия всех переменных функции
		\item Обратный порядок записи переменных
		\end{itemize}
	}

	\proof На дом
	\theorem{Функция $f: V_n(2) \rightarrow \mathbb{F}_2$ сильно равновероятна $\Leftrightarrow$ она $2^{n-1}$-равновероятна.}
	\theorem{Функция $f: V_n(2) \rightarrow \mathbb{F}_2$ $m$-равновероятна $\Leftrightarrow$ \[ \langle f, b \rangle = b_1f(x_1, ..., x_n) + ... + b_mf(x_m, ..., x_{n+m-1}) \] сбалансирована на каждом ненулевом векторе $b = (b_1, ..., b_m) \in V_m(2)$.}

	Построение сильно равновероятных функций из сильно равновероятных от меньшего числа переменных.
	\theorem{Пусть $h: V_m(2) \rightarrow \mathbb{F}_2, g: V_{n-m+1}(2) \rightarrow \mathbb{F}_2$. Функция $f: V_n(2) \rightarrow \mathbb{F}_2$, \[ f(x_1, ..., x_n) = g(h(x_1, ..., x_m), h(x_2, ..., x_{m+1}), ..., h(x_{n-m+1}, ..., x_n)) \] сильно равновероятна $\Leftrightarrow$ равновероятна каждая из функций $h$, $g$.}
	\paragraph{Пример}
	Перебором вероятностей всех 4-грамм получено, что функция $f: V_3(2) \rightarrow \mathbb{F}_2$ не имеет запрета $\Leftrightarrow$ она линейна по крайней переменной.

	Следующие двоичные функции не имеют запрета:

	$$
	\begin{aligned}
		f_1(x_1, x_2, x_3, x_4) &= x_1 + x_3 + x_2x_4 + x_1x_2x_4,\\
		f_2(x_1, x_2, x_3, x_4) &= x_2 + x_1x_3 + x_1x_3x_4,\\
		f_3(x_1, x_2, x_3, x_4) &= x_2 + x_3 + x_1x_3 + x_3x_4 + x_1x_3x_4, \\
		f_4(x_1, ..., x_5) &= x_1x_2 + x_2x_4 + x_4x_5 + x_3 + x_4 + x_5.
	\end{aligned}
	$$

	\paragraph{Задачи}
	\begin{enumerate}
		\item Среди двоичных функций от 2-х переменных найти все сильно равновероятные.
		\item Дана функция $f: V_3(2) \rightarrow \mathbb{F}_2$, \[ f(x_1, x_2, x_3) = x_1x_2 + x_2x_3 + x_1x_3 + 1 \]
		\begin{itemize}
			\item Найти вероятности всех её 2-грамм.
			\item Является ли она 2-равновероятной?
			\item Является ли она 3-равновероятной?
			\item Является ли она сильно равновероятной?
			\item Какие выходные последовательности для $f$ являются запретными: $(\gamma_1, ..., \gamma_5) \in {(1,1,1,1,1), (0,1,0,0,1), (1,0,1,1,0), (0,1,1,0,1), (1,0,0,1,0)}$ ?
		\end{itemize}
	\end{enumerate}

	
	\seminary{Семинар 6}
	\subsection{Линейная структура булевой функции}
	$$
	\begin{aligned}
		f(x_1, ..., x_n), ~ a \not = 0 \in \ZZ_2^n
	\end{aligned}
	$$
	Доказать, что
	$$
	f(x) \xor f(x \xor a) = \partialfrac{f}{a}, ~ \deg \partialfrac{f}{a} \le \deg f -1
	$$
	
	\proof
	$$
	\begin{aligned}
		~& f &= \gamma_0 \xor \gamma_1 x_1 \xor ... \xor \gamma_{1...n}x_1 ... x_n \\
		\xor &~&~ \\
		~& f(x \xor a) &= \gamma_0 \xor \gamma_1(x_1 \xor a_1) \xor ... \xor \gamma_{1...n} (x_1 \xor a_1) ... (x_n\xor a_n)\\
		\\
		~&f \xor f(x \xor a)& \\
	\end{aligned}
	$$
	
	\paragraph{Пример}
	$f(x_1, x_2, x_3) = x_1 x_2 \xor x_2 x_3 \xor x_3, ~~ \vec{a} = \overset{a_1}{1}\overset{a_2}{0}\overset{a_3}{0}$
	
	\definition Если для $f(x_1, ..., x_n), ~ \vec{a} \ne 0 \in \ZZ_2^n$ на $\forall \vec{x} \in \ZZ_2^n:~ \partialfrac{f}{\vec{a}} = \delta \in \ZZ_2^n$, то пара $(\vec{a}, \delta)$ называется линейной структурой булевой функции $f$.
	
	\definition $LS(n)$ --- все булевы функции от n переменных, у которых есть линейная структура, $L(n) \subseteq LS(n)$.
	
	
	\statement{
	Пусть $f(x_1, ..., x_n)$, тогда
	
	$$NL(f) \le 2^{n-1} - \frac{1}{2} 2^{\frac{n}{2}}, ~~ NL(f) = \min d(\angles{u, x}, f)$$
	
	$$NLS(f) = \min_{g \in LS(n)} d(f,g)$$
	
	$$NLS(f) \le 2^{n-2}, ~~ NLS(f) = 2^{n-2}~ \Leftrightarrow ~\forall a \in \ZZ_2^n,~ a \ne 0,~ \weight{\partialfrac{f}{a}} = 2^{n-1}$$
	}
	
	\lection{Лекция 4}
        \subsection{Алгебраическая иммунность}
        % TODO: move to bibliography
        -- Courtouis N. and Meier W. Algebraic attack on stream
        ciphers with linear feedback //
        LNCS. 2003. V. 2656. P. 345-359
        -- Meier W., Pasalic E., and Carlet C. Algebraic attack and
        decomposition of Boolean functions //
        LNCS. 2004. V. 3027. P. 474-491

        \subsection{Применение аннигиляторов}
        \[ \mathbb{F}_n = { f | f: \mathbb{Z}_2^n \rightarrow
            \mathbb{Z}_2 }. \]
        - Для $g \in \mathbb{F}_n$ обозначим $g \equiv 0$, если $g(x)
        = 0 \forall x \in \mathbb{Z}_2^n$. - $g \not\equiv 0$, если
        $g(\alpha) = 1$ для некоторого $\alpha \in \mathbb{Z}_2^n$.

        \definition Пусть $f: \mathbb{Z}_2^n \rightarrow
        \mathbb{Z}_2$. Функция $g \in \mathbb{F}_n$ называется
        \textit{аннигилятором} функции $f$, если \[ fg \equiv 0
          \textrm{ и } g \not\equiv 0 \]
        Равносильно: \[ f(x)g(x) = 0 \forall x \in \mathbb{Z}_2^n
          \textrm{ и } g \not\equiv 0 \]

        \begin{itemize}
          \item $Ann(f) = \{ g \in \mathbb{F}_n | fg \equiv 0 \}$ --
            \textit{множество всех аннигиляторов} функции $f$.
          \item $supp(f) = \{ x \in \mathbb{Z}_2^n | f(x) = 1 \}$ --
            \textit{носитель} функции $f$.
        \end{itemize}

        \paragraph{Пример} Пусть $f(x_1, x_2, x_3) = (01010011)$

        Пусть $\mathfrak{F}_{2, n}$ -- кольцо многочленов из
        $\mathbb{F}_2[x_1, ..., x_n]$, степень которых по каждой
        переменной не превосходит 1.
        \begin{itemize}
          \item $\mathfrak{F}_{2, n}$ -- множество всех многочленов Жегалкина
        над $\mathbb{Z}_2$
        \[ \mathbb{F}_n \cong \mathfrak{F}_{2, n} \]

          \item $||f||$ -- вес (Хемминга функции $f \in \mathbb{F}_n$. --
            число единиц в её таблице инстинности
          \end{itemize}

        \theorem{Пусть $f \in \mathbb{F}_n \cong \mathfrak{F}_{2, n}.$
        Тогда $Ann(f)$ -- главный идеал кольца $\mathfrak{F}_{2, n}$,
        порождённый функцией $1 + f$: \[ Ann(f) = \{ (1 + f)g | g \in
            \mathfrak{F}_{2, n} \} = (1 + f),\\
          |Ann(f)| = 2^{2^n-||f||}
        \]}
        \proof Так как $(1 + f) \in Ann(f)$, то $Ann(f) \neq
        \emptyset$. Пусть $r \in \mathfrak{F}_{2, n}, g \in Ann(f)$,
        тогда \[ r * g \in Ann(f) \]

        -- $Ann(f)$ -- идеал

        Докажем $Ann(f) = (1 + f)$. Очевидно, что \[ Ann(f) \supseteq
          (1 + f) \]

        Покажем что $Ann(f) \subseteq (1 + f)$ (в обратную сторону).
        Пусть $h \in Ann(f)$. \[ fh \equiv 0 \implies (f + 1)h \equiv
          h \implies h \in (1 + f) \]
        $Ann(f) \subseteq (1 + f)$.

        Так как \[ Ann(f) \supseteq (1 + f), Ann(f) \subseteq (1 +
          f), \] то $Ann(f) = (1 + f)$.

        Докажем $|Ann(f)| = 2^{2^n - ||f||}$. \[ g \in Ann(f) \implies
          \forall x \in supp(f) \implies x \notin supp(g) \]
        Значения g на множестве $V_n \setminus supp(f)$ выбираются
        произвольно.
        Так как \[ |V_n \setminus supp(f)| = 2^n - ||f||, \] то \[
          Ann(f) = 2^{2^n - ||f||}. \]

        \paragraph{Задача} Если $f \in \mathbb{F}_n$ является
        равновероятной, то \[|Ann(f)| = 2^{2^{n-1}}. \]

        \definition Порядком алгебраической иммунности функции $f \in
        \mathbb{F}_n$ называется число \[ Al(f) = min\{deg(g) | g \in
            \mathbb{F}_n, g \not\equiv 0, f g \equiv 0 \vee (f + 1)g
            \equiv 0 \} \]

        \paragraph{Задача} Пусть $f \in \mathbb{F}_n$ и $f g \equiv h$
        для некоторых $g, h \in \mathbb{F}_n$, \[ deg g \leq d, deg f
          \leq d, g \not\equiv 0. \] Доказать, что $Al(f) \leq d$.

        \statement{Пусть $f \in \mathbb{F}_n$. Тогда $Al(f) \leq \lceil
        n/2 \rceil$.}
        \proof Из функций $f, f + 1$ выберем функцию, вес которой не
        более $2^{n-1}$. Без ограничения общности, пусть $||f|| \leq
        2^{n-1}$.

        Строим таблицу. \[ supp(f) = {y_1, y_2, ..., y_d}, d =
          ||f||. \]
        \begin{itemize}
          \item каждая строка соответствует $y \in supp(f)$
          \item столбцы -- всем мономам степеней $0, 1, ..., \lceil
            \frac{n}{2} \rceil$.
        \end{itemize}
        Число всех таких мономов равно \[ \sum_{i = 0}^{\lceil
            \frac{n}{2} \rceil} \binom{n}{i} > 2^{n-2} \geq ||f|| \]
        \proof На пересечении строки $y$ и столбца $h$ записывается
        $h(y)$. Так как столбцов больше чем строк \[ \sum_{i = 0}^{\lceil
            \frac{n}{2} \rceil} \binom{n}{i} > ||f||, \], то
        существует их ненулевая линейная комбинация $g$, равная $0
        \forall x \in V_n$, \[ g(x) = g(x_1, ..., x_n) = \sum_{i = 0}^{\lceil
            \frac{n}{2} \rceil} \sum_{\substack{J \subseteq \{1, ...,
              n\}\\i = |J|}}  c_J \prod_{j \in J} x_j = 0 \]

        \begin{itemize}
          \item $deg(g) \leq \lceil \frac{n}{2} \rceil$
          \item $g(x) = 0 \forall x \in {y \in V_n | f(y) = 1}$
          \end{itemize}

        Пусть $\mathbb{A}_n$ -- множество всех аффинных функций от $n$
        переменных.

        \paragraph{Задача} Доказать что каждый ненулевой аннигилятор
        $g \in \mathbb{F}_n$ аффинной функции $a \in \mathbb{A}_n,
        deg(a) = 1$ представим в виде \[ g \equiv h(1 + a), \] где $h$
        -- произвольная функция из $\mathbb{F}, deg(h) = deg(g) - 1$.

        \subsubsection{Алгоритм нахождения множества $Ann_r(f)$}
        \begin{itemize}
          \item $Ann_r(f) = \{ g \in \mathbb{F}_n | fg \equiv 0,
            deg(g) \leq r \}$
          \item Из доказательства утверждения следует алгоритм
            нахождения множества $Ann_r(f)$
        \end{itemize}

        \paragraph{Алгоритм} (нахождение $Ann_r(f)$ методом
        неопределённых коэффициентов)

        \paragraph{Вход}: $f \in \mathbb{F}_n$

        \begin{enumerate}
          \item Представим аннигилятор $g \in Ann_r(f)$ в виде
            многочлена Жегалкина с неопределёнными коэффициентами
            $c_0, c_{i_1,i_2}, ..., c_{i_1,..., i_r}$:
            \[ g(x_1, ..., x_n) = c_0 + \sum_{i=1}^n c_ix_i + \sum_{i
                \leq i_1 < i_2 \leq n} c_i, i_2 x_{i_1} x_{i_2} +
              ... + \sum_{i
                \leq i_1 < i_2 \leq n} c_i, i_2 x_{i_1} x_{i_2}
              ... x{i_r} \]
          \item Выписываем СЛУ. Для всех $(x_1, ..., x_n) \in supp(f)$
            полагаем $g(x_1, ..., x_n) = 0$

            Получаем однородную систему из $||f||$ линейных уравнений
            относительно \[ \sum_{i=0}^r \binom{n}{i} \]
            неизвестных.\\
            Решением её являются коэффициенты многочленов Жегалкина
            аннигиляторов из $Ann_r(f)$ и только они.
        \end{enumerate}
        \paragraph{Выход}: $Ann_r(f)$

        \paragraph{Задача} Пусть $a \in \mathbb{A}_n, deg(a) =
        1$. Доказать что для каждого $t \in \mathbb{N}$ справедливо
        равенство \[ dim Ann_t(a) = \sum_{i=0}^{t-1} \binom{n-1}{i} \]

        \paragraph{Задача}
        \statement{Пусть  \[ f \in \mathbb{F}_n, a \in \mathbb{A}_n,
            deg(a) = 1, d = Al(f). \]
          Тогда \[ d - 1 \leq Al(f + a) \leq d + 1 \]}
        \proof Пусть $g \in Ann(f) \cup Ann(f + 1), d =
        deg(g)$. Рассмотрим два случая: \[ fg \equiv 0 \textrm{ и } fg
          \not\equiv 0 \].
        Для каждого из них разбираем: \[ (a + 1)g \equiv 0 \textrm{ и }
          (a + 1)g \not\equiv 0 \]

        \begin{enumerate}
        \item Пусть $fg \equiv 0$
          Рассмотрим два случая: \[ (a + 1)g \equiv 0 \textrm{ и } (a
            + 1) \not\equiv 0.\]
          Если $(a + 1)g \equiv 0$, то $(f + a + 1)g \equiv 0
          \implies$ \[ g \in Ann(f + a + 1). \]
          Если $(a + 1)g \not\equiv 0$, то из равенства \[ (f + a)(a +
            1)g = f(a + 1)g \equiv 0 \] $\implies$ \[ (a + 1)g \in
            Ann(f + a). \]
        \item Пусть $fg \equiv 0$. Рассмотрим два случая:
          \[ (a + 1)g \equiv 0 \textrm{ и } (a
            + 1) \not\equiv 0. \]
          Если $(a + 1)g \equiv 0$, то $(f + a + 1)g \equiv 0
          \implies$ \[ g \in Ann(f + a + 1). \]
          Если $(a + 1)g \not\equiv 0$, то из равенства \[ (f + a)(a +
            1)g = f(a + 1)g \equiv 0 \] $\implies$ \[ (a + 1)g \in
            Ann(f + a). \]
        \end{enumerate}

        \subsection{Свойства $Ann_r(f)$}
        \statement{Пусть $f_1, f_2 \in \mathbb{F}_n, d_i = Al(f_i), i
          = 1,2,$ \[ g(x_1, ..., x_n, x_{n+1}) = (x_{n+1} + 1) f_1 +
            x_{n+1}f_2. \] Тогда \[ Al(f) \leq min\{d_1, d_2\} +
            1. \]}
        \paragraph{Вывод} Из подфункций с ``низкой'' алгебраической
        иммунностью нельзя получить функцию с ``высокой''
        алгебраической иммунностью.
        \proof Пусть $g \in Ann(f_1) \cup Ann(f_1 + 1)$ и $d_1 =
        deg(g)$. Если $gf_1 \equiv 0$, то \[ (x_{n+1} + 1)gf_1 \equiv
          0. \]
        Если $g(f_1 + 1) \equiv 0$, то \[ (x_{n+1} + 1)gf_1 \equiv
          0. \] $\implies$ \[ Al(f_1) \leq d_1 + 1. \]
        Аналогично, $Al(f_2) \leq d_2 + 1 \implies$ \[ Al(f_2) \leq
          d_2 + 1 \]
        $\blacksquare$
        \paragraph{Следствие} Пусть $n$ чётно, $f_1, f_2 \in
        \mathbb{F}_n$, \[g(x_1, ..., x_n, x_{n+1}) = (x_{n+1} + 1)f_1
          + x_{n+1}f_2, \\ Al(g) = 1 + \frac{n}{2}. \]
        Тогда $Al(f_i) = \frac{n}{2}, i = 1, 2.$
        \begin{itemize}
          \item $Al(g) = 1 + \frac{n}{2}.$ (максимально возможная)
          \item $Al(f_i) = \frac{n}{2}$ (тоже максимально возможная)
        \end{itemize}

        \statement{Пусть $f \in \mathbb{F}_n, d \in \{1, ..., n\}$ и
          $d = Al(f)$. Тогда: \[ \sum_{i=0}^{d-1} \binom{n}{i} \leq
            ||f|| \leq \sum_{i=0}^{n-d} \binom{n}{i} \]}
        \proof Для доказательства применим метод, используемый для
        доказательства
        \begin{itemize}
          \item \textbf{Утв.} Пусть $f \in \mathbb{F}_n$. Тогда $Al(f)
              \leq \lceil \frac{n}{2} \rceil$.
          \item Построим аннигилятор $g$ функции $f$, $deg(g) \leq d -
            1$.
            Строим таблицу. \[ supp(f) = \{y_1, y_2, ..., y_d\}, d =
              ||f||. \]
          \item Каждая строка соответствует $y \in supp(f)$
          \item Столбцы -- всем мономам степеней $0, 1, ..., \lceil
            \frac{n}{2} \rceil$.
          \item На пересечении строки $y$ и столбца $h$ записывается
            $h(y)$.
          \item Число строк в таблице равно $||f||$, число столбцов
            (мономов) -- $\sum_{i=0}^{d-1} \binom{n}{i}$
        \end{itemize}
        Если число столбцов $>$ числа строк, то существует их линейная
        зависимость $\implies$ существует аннигилятор $g, deg(g) \leq
        d - 1$\\
        -- Нижняя оценка доказана $\blacksquare$

        С другой стороны, у функции $f + 1$ не существует
        аннигиляторов \[ h \in Ann(f), deg(h) \leq d - 1. \]
        $\implies$ \[ 2^n - ||f|| = ||f + 1|| \geq \sum_{i=0}^{d-1}
          \binom{n}{i} \]
        $\implies$ \[ ||f|| \leq 2^n - \sum_{i=0}^{d-1} \binom{n}{i} =
          \sum_{i=d}^n \binom{n}{i} = \sum_{i=0}^{n-d} \binom{n}{i}
        \]
        $\blacksquare$

        \paragraph{Следствие} Если $n$ нечётно, $f \in \mathbb{F}_n,
        Al(f) = \frac{n + 1}{2}$, то $f$ -- \textit{равновероятная
          функция}.
        \begin{itemize}
          \item $Al(f) = \frac{n + 1}{2}$ (максимально возможное)
        \end{itemize}

	\lection{Лекция 5}
	\subsection{Автокорреляция и взаимная корреляция}
        Автокорреляция и взаимная автокорреляция:
        \begin{itemize}
          \item применяются для исследования свойств криптографических отображений
          \item с помощью них характеризуются различные свойства двоичных функций
          \item выявляются ``нетривиальные'' связи
        \end{itemize}

        Пусть $f, g \in \mathbb{F}_2(n)$, \[ \mathbb{F}_2(n) = \{ f | f: V_n(2) \rightarrow \{ 0, 1 \} \}. \]

        \definition Функция $\delta_{f,g}: V_n(2) \rightarrow
        \mathbb{Z}$, \[ \delta_{f,g}(\epsilon) = \sum_{\beta \in
            V_n(2)} (-1)^{f(\beta) \oplus g(\beta \oplus \epsilon)},
          \forall \epsilon \in V_n(2) \]
        называется \textit{функцией взаимной корреляции} или
        \textit{взаимной корреляцией} (cross-correlation) двоичных
        функций $f, g$.
        \definition Функция $\delta_f : V_n(2) \rightarrow
        \mathbb{Z}$, \[ \delta_{f}(\epsilon) = \delta_{f, f}(\epsilon)
          = \sum_{\beta \in
            V_n(2)} (-1)^{f(\beta) \oplus f(\beta \oplus \epsilon)},
          \forall \epsilon \in V_n(2) \]
        называется \textit{функцией автокорреляции} или
        \textit{автокорреляцией} (auto-correlation) двоичной функции
        $f$.

        \subsection{Коэффициенты Уолша-Адамара 2-го рода}
        Коэффициентами Уолша-Адамара 2-го рода функции $f \in
        \mathbb{F}_2(n)$ называется: \[ w_f(\alpha) = \sum_{\beta \in
            V_n(2)} (-1)^{f(\beta) \oplus (\alpha, \beta)},
          \alpha \in V_n(2) \]
        \textit{Обратное преобразование}
        \[ \sigma_f(x) = (-1)^{f(x)} = 2^{-n} \sum_{\alpha \in V_n(2)}
          w_f(\alpha)\omega_{\alpha} \]
        где
        \begin{itemize}
          \item $\langle \alpha, \beta \rangle$ -- скалярное произведение
            $\alpha, \beta \in V_n(2)$,
            \item $\omega_{\alpha}(\beta) = (-1)^{\langle \alpha,
                \beta \rangle}$,
              т.е. \[ (-1)^{f(x)} = 2^{-n} \sum_{\alpha \in V_n(2)}
          w_f(\alpha)(-1)^{\langle \alpha,
            \beta \rangle} \]
        \end{itemize}

        \subsection{Взаимная корреляция}
        Пусть $f, g \in \mathbb{F}_2(n), \epsilon \in V_n(2)$. Тогда:
        \begin{enumerate}
          \item $\delta_{f, g}(\epsilon) = \delta_{g, f}(\epsilon)$,
          \item $\delta_f(\epsilon) = w_{d_{\epsilon}f}(0_n)$, где
            $d_{\epsilon}f$ -- производная функции $f$ по направлению
            $\epsilon$,
          \item $d_{\epsilon}f(x) = f(x \oplus \epsilon) \oplus f(x)$
        \end{enumerate}

        \textbf{Замечание}

        \begin{itemize}
        \item Не каждый набор из $2^n$ целых чисел $(y_1, ...,
          y_{2^n}) \in \mathbb{Z}^{2^n}$ может быть набором значений
          автокорреляции некоторой двоичной функции
        \item Для заданного набора $(y_1, ..., y_{2^n}) \in
          \mathbb{Z}^{2^n}$ установить язвляется ли он набором
          значений автокорреляции двоичной функции может оказаться
          сложно.
        \end{itemize}

        \subsection{Матричная форма преобразования Уолша-Адамара}
        \textbf{Свойства}

        \begin{itemize}
        \item ассоциативности \[ (a \otimes b) \otimes d = a \otimes (b \otimes
            d) \]
        \item левой и правой дистрибутивности
          \[
            \begin{aligned}
              (a + b) \otimes d = a \otimes d + b \otimes d,\\
              d \otimes (a + b) = d \otimes a + d \otimes b
            \end{aligned}
          \]
        \end{itemize}

        \subsection{Матрица Адамара}
        Пусть $n \in \mathbb{N}$

        \definition \textit{Матрицей Адамара типа Сильвестра}
        называется квадратная ($2^n \times 2^n$)-матрица $H_n$,
        заданная условием \[ H_1 = \begin{pmatrix}
          1 & 1 \\
          1 & -1
        \end{pmatrix} \]
        \[ H_n = H_1 \otimes H_{n-1} при n \ge 2 \],
        \begin{itemize}
        \item $\otimes$ -- тензорное произведение матриц
        \item ``Важное'' свойство матрицы Адамара: \[ H_n = H_n^t =
            2^nH_n^{-1} \]
        \end{itemize}
        Отсюда вытекает: \[ H_n H_n^t = H_n H_n = 2^n E_n \]
        где $E_n$ -- единичная $(2^n \times 2^n)$-матрица.

        \begin{itemize}
        \item элементы матрицы $H_n = [[h_{\alpha, beta}]]$ заданы
          равенством \[ h_{\alpha, beta} = (-1)^{\langle \alpha,
              \beta \rangle} для \alpha, beta \in V_n(2), \]
          -- вектору $\alpha = (\alpha_1, ..., \alpha_n)$
          соответствует число $\alpha_1 2^{n-1} + \alpha_2 2^{n-2} +
          ... + \alpha_n$.
        \item $H_n$ -- матрица значений характеров элементарной
          2-абелевой группы
        \end{itemize}

        \subsection{Совершенно некоррелированные функции}

        \definition Функции $f, g \in \mathbb{F}_2(n)$ называются:
        \begin{itemize}
        \item совершенно некоррелированными (perfectly uncorrelated),
          если \[ \delta_{f, g}(\epsilon) = 0 \forall \epsilon \in
            V_n(2). \]
        \item некоррелированными порядка $k, k \in \{ 0, ..., n \}$
          (uncorrelated of degree $k$), если \[ \delta_{f, g}(\epsilon) = 0 \forall \epsilon \in
            V_n(2), \|\epsilon\| \le k. \]
        \end{itemize}

        \subsection{Теорема о взаимной корреляции}
        \theorem[о взаимной корреляции]{Пусть $f, g \in
          \mathbb{F}_2(n)$. Тогда \[ (\delta_{f, g}(0), ...,
            \delta_{f, g}(2^n - 1))H_n = (w_f(0)w_g(0), ..., w_f(2^n -
            1)w_g(2^n - 1)) \]
        }
        \proof
        \[ \sum_{\alpha \in V_n(2)} \delta_{f, g}(\alpha)(-1)^{\langle
            \alpha, \beta \rangle} = \sum_{\alpha \in V_n(2)}
          \sum_{\Theta \in V_n(2)} (-1)^{f(\Theta) \oplus g(\Theta \oplus
            \alpha)} (-1)^{\langle \alpha, \beta \rangle} =
          \sum_{\Theta \in V_n(2)} (-1)^{f(\Theta)} \sum_{\alpha \in
            V_n(2)} (-1)^{g(\Theta \oplus \alpha) \oplus \langle \alpha,
            \beta \rangle} \]
        Сделаем замену $\tau = \Theta \oplus \alpha$
        \[ \sum_{\Theta \in V_n(2)} (-1)^{f(\Theta)} \sum_{\alpha \in
            V_n(2)} (-1)^{g(\tau) \oplus \langle \tau \oplus \Theta,
            \beta \rangle} = \sum_{\Theta \in V_n(2)} (-1)^{f(\Theta)
            \oplus \langle \tau \oplus \Theta,
            \beta \rangle} \sum_{\alpha \in
            V_n(2)} (-1)^{g(\tau) \oplus \langle \tau,
            \beta \rangle} = w_f(\beta) w_g(\beta) \] $\blacksquare$

        \textbf{Следствие 2} Пусть $f \in \mathbb{F}_2(n)$. Тогда \[
          (\delta_f(0), ..., \delta_f(2^n - 1)) H_n = (w_f(0)^2, ...,
          w_f(2^n - 1)^2) \]
        \proof следует из теоремы при $f = g$ $\blacksquare$

        \textbf{Следствие 3} Пусть $f \in \mathbb{F}_2(n)$. Тогда \[
          2^n(\delta_{f, g}(0), ..., \delta_{f, g}(2^n - 1)) =
          (w_f(0)w_g(0), ..., w_f(2^n - 1)w_g(2^n - 1)) H_n \]
        \proof следует из теоремы и равенства $H_n H_n = 2^n E_n$
        $\blacksquare$

        Из следствия вытекает равенство Парсеваля, \[ \sum_{\alpha \in
            V_n(2)} w_f^2(\alpha) = 2^n \delta_f(0) = 2^{2n} \]

        \paragraph{Задача}
        Пусть $h \in \mathbb{F}_2(n+m)$. \[ h(x, y) = f(x) \oplus g(y),
          x \in V_n(2), y \in V_m(2) \] для некоторых $f \in
        \mathbb{F}_2(m)$. Тогда \[ \delta_h(\alpha, \beta) =
          \delta_f(\alpha)\delta_g(\beta) \forall \alpha \in V_n(2),
          \beta \in V_m(2) \].

        \subsection{Теорема о свёртке}
        \textbf{Следствие 4 (теорема о свёртке)}. Пусть $f,g,h \in
        \mathbb{F}_2(n)$, \[ h(x) = f(x) \oplus g(x) \forall x \in
          V_n(2) \]
        Тогда \[ w_h(\beta) = 2^{-n} \sum_{\alpha \in V_n(2)}
          w_f(\alpha) w_f(\alpha \oplus \beta) \forall \beta \in
          V_n(2)\]
        \proof Зафиксируем произвольное $\beta \in V_n(2)$ и положим \[
          g_{\beta}(x) = g(x) \oplus \langle \beta, x \rangle \]
        Тогда \[w_{g_{\beta}}(\alpha) = w_g(\alpha \oplus
          \beta). \] \[w_h(\alpha) = \delta_{f, g_{\beta}}(0) \]
        для всех $\alpha \in V_n(2)$

        Из следствия 3 вытекает \[ w_h(\alpha) = 2^{-n} \sum_{\alpha
            \in V_n(2)} w_f(\alpha)w_{g_{\beta}}(\alpha) = 2^{-n}
          \sum_{\alpha in V_n(2)} w_f(\alpha) w_g(\alpha \oplus
          \beta) \]

        \subsection{Коэффициенты Уолша-Адамара и криптографические
          свойства}
        -- Многие важные криптографические свойства функции $f$
        зависят от числа ненулевых коэффициентов Уолша-Адамара

        Для $f \in \mathbb{F}_2(n)$ положим
        \[
          \begin{aligned}
            W_f^{(0)} = \{ \beta \in V_n(2) | w_f(\beta) = 0 \},\\
            W_f^{(1)} = \{ \beta \in V_n(2) | w_f(\beta) \ne 0 \},\\
            \delta_f^{(1)} = \{ beta \in V_n(2) | \delta_f(\beta) \ne
            0 \}.
          \end{aligned}
        \]

        \definition Функции $f, g \in \mathbb{F}_2(n)$ имеют
        \textit{непересекающиеся спектры}, если \[ W_f^{(1)} \cap
          W_g^{(1)} = \varnothing \]

        \statement{Функции $f, g \in \mathbb{F}_2(n)$ имеют
          непересекающиеся спектры $\Leftrightarrow$ $f, g$ совершенно
          некоррелированные}

        \lemme{Пусть $f, g \in \mathbb{F}_2(n)$. Тогда \[ \sum_{\alpha
              \in V_n(2)} \delta_{f, g}^2(\alpha) \le 2^{3n} \]}

        \proof Положим
        \[
          \begin{aligned}
            \delta_{f, g} = (\delta_{f, g}(0), ..., \delta_{f, g}(2^n
            - 1)),\\
            W_{f, g} = (w_f(0)w_g(0), ..., w_f(2^n - 1)w_g(2^n - 1))
          \end{aligned}
        \]

        Применяя теорему о взаимной корреляции, имеем \[ \langle W_{f,
            g}, W_{f, g} \rangle = \langle \delta_{f, g} H_n,
          \delta_{f, g} H_n \rangle = \langle \delta_{f, g} H_n H_n^T,
          \delta_{f, g} \rangle = 2^n \langle \delta_{f, g},
          \delta_{f, g} \rangle. \]
        \[ 2^n \sum_{\alpha \in V_n(2)} \delta_{f, g}^2(\alpha) =
          \sum_{\alpha \in V_n(2)} w_f(\alpha) w_g(\alpha) \le
          (\sum_{\alpha \in V_n(2)} w_f^2(\alpha))(\sum_{\alpha \in
            V_n(2)} w_g^2(\alpha)) = 2^{2n} * 2^{2n} \]
        Отсюда следует неравенство $\blacksquare$

        \statement{Для каждой $f \in \mathbb{F}_2(n)$ справедливо
          неравенство \[ |\delta_f^{(1)}| * |W_f^{(1)}| \ge 2^n \]
          где \[ W_f^{(1)} = \{ \beta \in V_n(2) | w_f(\beta) \ne 0
            \} \] }

        \textbf{Замечания}
        \begin{itemize}
        \item для некоторых приложений нужно большое число ненулевых
          коэффициентов Уолша-Адамара
        \item для других приложений нужно большое число нулей функции
          автокорреляции
        \end{itemize}
        $\implies$
        \begin{itemize}
        \item из неравенства следует, что одновременно обеспечить оба
          требования невозможно
        \item если параметр ``хороший'', то другой ``плохой''. Нельзя
          сделать обе характеристики хорошими
        \end{itemize}

        \subsection{Энтропия}

        Пусть $\xi$ -- дискретная случайная величина, заданная на
        конечном множестве $X = \{x_1, ..., x_n\}$
        \begin{itemize}
        \item $\mathbb{R}^+ = \{x \in \mathbb{R} | x > 0\}$,
        \item $\alpha \in \mathbb{R}^+$
        \end{itemize}
        \definition Мерой (логарифмической) неопределённости события
        $\{\xi = x\}$ при $x \in X$ называется \[ I(\xi = x) =
          -\log_{\alpha} P\{\xi = x\} \]

        \begin{itemize}
        \item основание логарифма $\alpha \in \mathbb{R}^+$
          -- зависит от выбора шкалы измерения информации (двоичной,
          натуральной и т.д.)
        \item $I$ является случайной величиной, принимающей значение
          $I(\xi = x)$ с вероятностью $P\{\xi = x\}$
        \end{itemize}
        \begin{itemize}
        \item Пусть $\eta$ -- случайная величина, заданная на
          множестве $Y$,
        \item $P\{\xi = x, \eta = y\}$ -- совместное распределение
          пары $(\xi, \eta)$ случайных величин на $X \times Y, (x, y)
          \in X \times Y$.
          \[ \sum_{y \in Y} P\{\xi = x, \eta = y\} = P\{\xi = x\} \]
          \[ \sum_{x \in X} P\{\xi = x, \eta = y\} = P\{\eta = y\} \]
        \end{itemize}

        \definition Случайные величины $\xi, \eta$ называются
        независимыми, если \[ P\{\xi = x, \eta = y\} = P\{\xi =
          x,\}P\{\eta = y\} \] для каждой $(x, y) \in X \times Y$

        \subsection{Условная энтропия}
        \begin{itemize}
        \item Условная энтропия случайной величины $\eta$ при условии
          выполнения события $\{ \xi = x \}$: \[
            \begin{aligned}
              H(\eta | \xi = x) &= -
            \sum_{y \in Y} P\{\eta = y | \xi = x \} \log_a P\{\eta = y
                                  | \xi = x \} \\
                                &= - \sum_{x \in X} \sum_{y \in Y} P\{ \xi =
              x, \eta = y \} \log_a P\{ \eta = y \}
            \end{aligned}
              \] %TODO: INCOMPLETE
        \end{itemize}
        \subsection{Взаимная информация}
        \definition Взаимная информация $I(\xi, \eta)$ случайных
        величин $\xi, \eta$ задаётся: \[ I(\xi, \eta) = \sum_{x \in X}
          \sum_{y \in Y} P\{ \xi = x, \eta = y \} \log_a \frac{P\{\eta
            = y, \xi = x\}}{P\{\eta = y\}P\{\xi = x\}} \]

        \lemme{$I(\xi, \eta) = 0 \Leftrightarrow \xi, \eta$
          независимые случайные величины}

        \subsection{Корреляционная имунность и устойчивость двоичных
          функций}

        \begin{itemize}
        \item \textit{Корреляционная имунность} -- мера
          противодействия корреляционному методу анализа
        \item Отсутствие какой-нибудь информации о некоторых значениях
          функции от ``некоторого'' подмножества её аргументов.
          \begin{itemize}
          \item если такая информация есть -- то её можно применять для
          нахождения ключа
          \item подход зависит от выбранной вероятностной модели
          \end{itemize}
        \end{itemize}

        \subsection{Корреляционная иммунность}
        \begin{itemize}
        \item $\mathbb{F}_2(n) = \{f | f: V_n(2) \rightarrow \{0,
          1\}\}$
        \item Пусть $f: V_n(2) \rightarrow \{0, 1\}$. $\xi_1, ...,
          \xi_n$ -- независимые двоичные одинаково распределённые
          случайные величины, \[ P\{\xi_i = 0\} = P\{\xi_i = 1\} =
            \frac{1}{2}, i = 1, ..., n \]
        \item Двоичная случайная величина \[ \theta = f(\xi_1, ...,
            \xi_n) \] имеет распределение \[ P\{\theta = 1\} = \|f\| *
            2^{-n}, P\{\theta = 0\} = 1 - \|f\| * 2^{-n} \]
        \end{itemize}

        \definition Двоичная функция $f: V_n(2) \rightarrow \{0, 1\}$
        называется корреляционно-иммунной порядка $m \in \{1, ..,
        n\}$, если для каждого набора $i = (i_1, i_2, ..., i_m), 1 \le
        i_1 < i_2 < ... < i_m \le m$ справедливо условие \[
          I(\xi^{(i_1, i_2, ..., i_m)}, \theta) = 0 \], где $\xi^{(i)}
        = \xi^{(i_1, i_2, ..., i_m)} = (\xi_{i_1}, ..., \xi_{i_m})$

        \definition Двоичная функция $f: V_n(2) \rightarrow \{0, 1\}$
        называется корреляционно-иммунной порядка $m \in \{1, ..,
        n\}$, если для каждого набора $i = (i_1, i_2, ..., i_m), 1 \le
        i_1 < i_2 < ... < i_m \le m$ случайные величины $\xi_{i_1}, ...,
        \xi_{i_m}$ и [DATA EXPUNGED] % TODO: ??

        \subsection{Критерий корреляционной иммунности}

        $Cor_n^{(m)}$

        \statement Функция $f \in \mathbb{F}_2(n)$ является
        корреляционно-иммунной порядка $m \in \{1, ..., n - 1 \}
        \Leftrightarrow$ для каждого набора \[ i = (i_1, i_2, ...,
          i_m) \in I_n^{(m)} и вектора \alpha = (\alpha_1, \alpha_2,
          ..., \alpha_m) \in V_m(2)\]
        справедливо \[ \|f_{i_1, ..., i_m}^{(\alpha_1, ...,
            \alpha_m)}\| = 2^{-m} * \|f\| \]
        \[ I_n^{(m)} = \{(i_1, ..., i_m) \in \{1, ..., n\}^m | 1 \le
          i_1 < i_2 < ... < i_m \le n \]
        -- $I_n^{(m)}$ -- множество всех строго упорядоченных наборов
        из $m$ элементов из множества n

        \proof (Необходимость)\\
        Пусть $f \in Cor_n^{(m)}$. Из леммы следует что для каждого
        набора $i \in I_n^{(m)}$ заданы случайные величины \[
          \xi^{(i)} = (\xi_{i_1}, ..., \xi_{i_m}) и \theta = f(\xi_1, ...,
          \xi_n) \]
        Значит для каждых $\alpha \in V_m(2), \beta \in \{ 0, 1 \}$
        имеем \[ P\{\xi^{(i)} = \alpha|\theta = \beta\} = \frac{
            P\{\xi^{(i)} = \alpha, \theta = \beta\}}{ P\{\theta =
            \beta\}} = P\{ \xi^{(i)} = \alpha \} \]

        Випишем вероятности для $\beta = 1$.

        \[
          \begin{aligned}
            P\{\theta = 1\} = 2^{-n}\|f\|,\\
            P\{\xi^{(i)} = \alpha\} = 2^{-m}
          \end{aligned}
        \]

        \proof (Достаточность)\\
        Пусть на всех $i \in I_n^{(m)}$ и $\alpha \in V_m(2)$
        выполняется \[ \|f_i^{(\alpha)}\| = 2^{-m}\|f\| \]
        Проведя рассуждения в обратном порядке докажем равенство при
        $\theta = 1$
        \[
          \begin{aligned}
            P\{\xi^{(i)} = \alpha | \theta = 0\} &= \frac{ P\{\xi^{(i)}
            = \alpha, \theta = 0\}}{ P\{\theta = 0\}} = \frac{
            P\{\xi^{(i)} = \alpha\}}{1 - P\{\theta = 1\}} - \frac{
            p\{\xi^{(i)} = \alpha, \theta = 1\}}{1 - P\{\theta = 1\}}
            \\
            &= \frac{2^{-m} - 2^{-n} \|f_i^{(\alpha)}\|}{1 - 2^{-n}\|f\|}
          = \frac{2^{-m} - 2^{-n-m} \|f\|}{1 - 2^{-n}\|f\|} = 2^{-m} =
            P\{\xi^{(i)} = \alpha\}
          \end{aligned}
        \]
        $\implies$
        [DATA EXPUNGED]
        % TODO: implies what? didn't have time to even take a pic

        \subsection{Корреляционная иммунность и коэффициенты
          Уолша-Адамара}
        -- Характеризация двоичной корреляционно-иммунной функции
        через коэффициенты Уолша-Адамара.

        Для $\alpha \in V_n(2), \alpha = (\alpha_1, \alpha_2, ...,
        \alpha_n)$ положим \[ supp(\alpha) = \{j \in \{1, ..., n\} |
          \alpha_j = 1 \} \]

        \lection{Лекция 6}
	
	\lection{Лекция 7}
	
	\lection{Лекция 8}
	
\end{document}
