\documentclass[a4paper,12pt]{article}
\usepackage{amsmath,amssymb,amsfonts}
\usepackage{mathtext}
\usepackage[english,russian]{babel}
\usepackage[utf8]{inputenc}
\usepackage[T2A]{fontenc}
\usepackage{graphicx}
\usepackage{textcomp}
\usepackage{geometry}
\geometry{left=3cm}
\geometry{right=1.5cm}
\geometry{top=2cm}
\geometry{bottom=2cm}
\usepackage{tikz}
\usepackage{titling}
\usepackage{indentfirst}
\setlength{\parindent}{1cm}
\usepackage{soul}
\usepackage{enumitem}
\usepackage{listings}
\usepackage{fvextra}
\usepackage{tabularx}
\usepackage{fancyhdr}
\usepackage{setspace}
\usepackage{tikz}
\usepackage{tikz-cd}
\usepackage{ifthen} % provides \isempty test
\usepackage{hyperref}

\hypersetup{
	colorlinks,
	citecolor=black,
	filecolor=black,
	linkcolor=black,
	urlcolor=black
}

%\documentclass{beamer}
%\usepackage{lmodern}
%\usepackage{tikz}
\usetikzlibrary{decorations.pathmorphing}
\tikzset{wavy/.style={decorate, decoration=snake}}

\onehalfspacing
\makeatletter
\AddEnumerateCounter{\asbuk}{\russian@alph}{щ}
\makeatother
\fvset{breaklines=true, breakafter=\space}

\DeclareRobustCommand{\divby}{%
	\mathrel{\vbox{\baselineskip.65ex\lineskiplimit0pt\hbox{.}\hbox{.}\hbox{.}}}%
}

\newcommand{\lection}[1]{\section*{#1}
	\addtocounter{section}{1}
	\addcontentsline{toc}{section}{#1}
	\setcounter{subsection}{0}
}

\newcommand{\seminary}[1]{\section*{#1}
	\addcontentsline{toc}{section}{#1}
}

\newcommand{\defn}{\stepcounter{paragraph}\paragraph{Определение \arabic{section}.\arabic{subsection}.\arabic{paragraph}}
}

\newcommand{\angles}[1]{\langle #1 \rangle}
\newcommand{\iftext}{\text{если}}
\newcommand{\weight}[1]{\|#1\|}
\newcommand{\transp}[1]{#1^\top}
\newcommand{\ind}[1]{\text{I}(#1)}


\newcommand{\refeq}[1]{(\ref{#1})}

\newcommand{\statement}[1]{\stepcounter{paragraph}\paragraph{Утв.} #1}
\newcommand{\proof}{\paragraph{Доказательство}}


\newcounter{theorem}
\newcommand{\theorem}[2][]{\stepcounter{theorem}\paragraph{Теорема \arabic{section}.\arabic{subsection}.\arabic{theorem}} \ifthenelse{\equal{#1}{}}{}{(#1)} {\itshape #2 }
}

\fancypagestyle{titlepage}{
	\fancyhf{
		\begin{center}
			\begin{tabularx}{\textwidth}[c]{c p{12cm}}
				\raisebox{-1.1\totalheight}{\includegraphics[width=0.15\linewidth]{./logo/bauman_logo.png}}
				&\begin{center}
					\textbf{ Министерство науки и высшего образования Российской Федерации \\
						Федеральное государственное бюджетное образовательное учреждение 
						высшего образования \\
						«Московский государственный технический университет
						имени Н.Э. Баумана \\
						(национальный исследовательский университет)» \\ (МГТУ им. Н.Э. Баумана)} \\
					
				\end{center}
			\end{tabularx}
	\end{center}}
	\fancyfoot[C]{Москва 2024 г.}
	\renewcommand{\headrulewidth}{0pt}
}

\title{~\\~\\~\\~\\~\\~\\~\\~\\~\\Криптографические методы защиты информации}
\author{Конспект лекций}
\date{МГТУ им. Н.Э. Баумана}

\begin{document}
	\maketitle
	\thispagestyle{titlepage}
	\pagebreak
	\tableofcontents
	
	\section*{Disclaimer}
	\addcontentsline{toc}{section}{Disclaimer}
	Конспект создан студентами для подготовки к экзамену ввиду отсутствия удобного формата лекций. Поэтому он может содержать ошибки, опечатки и многое другое, за что можно получить автомат с сапогами в придачу.
	
	\textit{(Тим Кравченко aka Fe-Ti)}
	

	\pagebreak
	\lection{Лекция 1}
	
	\subsection{Псевдобулева функция}
	Пусть $F$ --- произвольное поле, $p \ge 2$ --- простое число, $V_n(p) = GF(p)^n$ --- $n$-мерное пространство над полем $GF(p)$.
	
	\defn
	Псевдобулевой функцией от $n$ переменных называется произвольное отображение $f: V_n(2) \rightarrow F$. \\
	
	Если $F = GF(2)$, то функция называется булевой (двоичной). $f: V_n(p) \rightarrow F$ --- обобщение псевдобулевой функции.
	
	Обозначим $F(p,n) = \{f ~| ~ f:V_n(p) \rightarrow F\}$ и зададим на нём операции сложения и умножения на элемент поля:
	$$
	\begin{aligned}
		(f_1 + f_2) (x) &= f_1(x) + f_2 (x) \\
		(r \cdot f) (x) &= rf(x), ~ r \in F.
	\end{aligned}
	$$
	
	\statement{Относительно заданных операций $F(p,n)$ является векторным пространством размерности $p^n$ над $F$.}
	\proof
	(Д/з: проверить, что это векторное пространство)
	
	$\left\{ h_\alpha (\vec{x}) = \left[ \begin{aligned}
		1, ~ \iftext ~ \vec{x} = \vec{\alpha} \\
		0, ~ \iftext ~ \vec{x} \ne \vec{\alpha}
	\end{aligned} \right. ~\vline~ \vec{\alpha} \in GF(p)^n \right\}$ --- один из базисов пространства. Линейная независимость очевидна, и любая функция $f \in F(p,n)$:
	$$
	f(\vec{x}) = \sum_{\vec{\alpha} \in GF(p)^n} f(\vec{\alpha}) h_\alpha(\vec{x}). ~\blacksquare
	$$
	
	\subsection{Скалярное произведение}
	\defn 
	Скалярным произведением векторов $\vec{\alpha} = (\alpha_1, ..., \alpha_n)$, $\vec{\beta} = (\beta_1, ..., \beta_n)$, $\vec{\alpha}, \vec{\beta} \in V_n(2)$ называется $$\angles{\vec{\alpha},\vec{\beta}} = \bigoplus_{i=1}^{n} \alpha_i \beta_i.$$
	
	\defn
	Два вектора $\vec{\alpha}, \vec{\beta}$ ортогональны, если $\angles{\vec{\alpha}, \vec{\beta}} = 0$.
	
	\subsection{Аффинные функции}
	\defn
	$\mathcal{AL}_n = \{f \in \mathbb{F}_2 (n) ~|~ \deg f \le 1 \}$ --- множество аффинных функций.
	
	\defn
	$\mathcal{L}_n = \{f \in \mathcal{AL}_n ~|~ f(\vec{0} = 0)\}$ --- множество всех линейных функций.
	
	$$f \in \mathcal{AL}_n ~\Leftrightarrow ~ f(\vec{x}) = \alpha_1 x_1 \oplus ... \oplus \alpha_n x_n \oplus \alpha_0 = \angles{\vec{x}, \vec{\alpha}} \oplus \alpha_0.$$
	
	\paragraph{Задача} Доказать:
	$$
	\begin{aligned}
		|\mathcal{L}_n| &= 2^n \\
		|\mathcal{AL}_n| &= 2^{n+1}
	\end{aligned}
	$$
	
	\subsection{Равновероятная функция}
	Пусть $f = \vec{f} = (f(0), f(1), ..., f(2^n - 1))$ --- вектор столбец булевой функции $f(\vec{x})$, а $\weight{f} = \sum_{\vec{\alpha} \in V_n(2)} f(\vec{\alpha})$ --- вес этой функции.
	
	\defn
	Функция $f \in F_2(n)$ называется равновероятной (сбалансированной, равновесной), если $\weight{f} = 2^{n-1}$.
	
	\paragraph{Задача} Доказать, что если $f \in \mathcal{AL}_n$, то $f$ является равновесной.
	
	\paragraph{Задача} Пусть $f \in \mathbb{F}_2(n), ~ g \in \mathbb{F}_{2}(n+1)$, $g(x_1,...,x_{n+1}) = f(x_1, ...,x_n)$, $x_{n+1}$ --- фиктивная переменная функции $g$. Доказать, что $\weight{g} = 2\cdot \weight{f}$.
	
	\paragraph{Задача} Пусть $f\in \mathbb{F}_2(n), ~ g \in \mathbb{F}_2(n+2)$: $g(x_1,~ ...,~ x_{n+2}) = f(x_1,~ ...,~ x_{n-1},~ x_{n+1} \oplus x_{n+2})$. Известен $\weight{f}$. Найти $\weight{g}$.
	
	\paragraph{Задача} Пусть $f\in \mathbb{F}_2(n), ~ g \in \mathbb{F}_2(n+1)$: $g(x_1,~ ...,~ x_{n+1}) = f(x_1,~ ...,~ x_{n}) \oplus x_{n+1}$. Доказать, что функция $g$ равновероятна.
	
	\paragraph{Задача} Пусть $f \in \mathbb{F}_2(n)$. Доказать:
	$$ \weight{f} \equiv 1 ~ (\text{mod} 2) ~~ \Leftrightarrow ~~ \deg f = n.$$

	\subsection{Производная булевой функции}
	
	\defn
	Производной булевой функции $f \in \mathbb{F}_2(n)$ по направлению $\vec{\delta} \in V_n(2)$ называется булева функция
	$$
	d_{\vec{\delta}} f(\vec{x}) = f(\vec{x} \oplus \vec{\delta}) \oplus f(\vec{x}), ~~ \vec{x} \in V_n (2).
	$$
	
	\defn
	Производной булевой функции $f \in \mathbb{F}_2(n)$ по подпространству $L \le V_n(2)$ называется булева функция
	$$
	d_L f(\vec{x}) = \sum_{\vec{\delta} \in L} f(\vec{x}  \oplus \vec{\delta}), ~~ \vec{x} \in V_n (2).
	$$
	
	\paragraph{Свойство} Пусть $\vec{\delta}_1, ..., \vec{\delta}_k$ --- базис подпространства $L \le V_n(2)$. Тогда $$d_L f = d_{\vec{\delta}_k} ... d_{\vec{\delta}_1} f.$$
	
	\paragraph{Доказать утверждения:}
	\begin{itemize}
		\item $\forall f \in \mathbb{F}_2(n), \forall L \le V_n(2), \forall \vec{\epsilon} \in L:$  $d_L f(\vec{x}) = d_L f (\vec{x} \oplus \vec{\epsilon}).$ 
		
		\item $\forall f,g \in \mathbb{F}_2(n),  \forall L \le V_n(2):$ $d_L(f\oplus g)(\vec{x}) = d_L f(\vec{x}) \oplus d_L g(\vec{x})$.
		
		\item $\forall f \in \mathbb{F}_2(n), \forall \vec{\epsilon}_2, \vec{\epsilon}_2 \in V_n(2):$ $d_{\vec{\epsilon}_2 \oplus \vec{\epsilon}_2} f(\vec{x}) = d_{\vec{\epsilon}_1} f(\vec{x}) \oplus d_{\vec{\epsilon}_1} f(\vec{x} \oplus \vec{\epsilon}_1).$
		
		\item $\forall \vec{\delta} \in V_n(2): d_{\vec{\delta}} f(\vec{x}) = 0 \Leftrightarrow f \text{ --- константа.}$
		
		\item $\forall \vec{\delta} \in V_n(2): d_{\vec{\delta}} f(\vec{x})  \text{ --- константа.} \Leftrightarrow f \text{ --- аффинная функция.}$
	\end{itemize}
	
	\subsection{Гомоморфизмы (флешбэк)}
	\defn Отображение $\phi: (G, \circ) \rightarrow (H, \diamond)$ называется гомоморфизмом, если $\forall x,y \in G: ~ \phi(x \circ y) = \phi(x) \diamond \phi(y)$.
	
	\defn Биективный гомоморфизм называется изоморфизмом.
	
	\subsection{Гомоморфизмы $(V_n(p), +)$ в мультипликативную группу поля$\mathbb{C}$}
	
	Стандартный базис $V_n(p)$:
	$$
		\vec{e}_j = (0,...,0,\underbrace{1}_{j\text{-я позиция}},0,...,0), j = 1, ..., n. \\
	$$
	
	Базис пространства $F(p,n)$, разложение по которому часто используется на практике.
	
	\theorem{
			Множество всех гомоморфизмов $\phi : (V_n(p), +) \rightarrow (\mathbb{C}, \cdot)$ состоит из $p^n$ различных гомоморфизмов $\omega_{\vec{\alpha}}$, $\vec{\alpha} = (\alpha_1, ... , \alpha_n) \in V_n(p)$, каждый из которых однозначно определяется своим действием на векторах стандартного базиса $\vec{e}_j, ~j = 1,...,n,$ условием
			$$
			\omega_{\vec{\alpha}}(\vec{e}_j) = \exp\left({\frac{2 \pi i}{p} \alpha_j}\right),
			$$
			где $i$ --- мнимая единица.
		}
	\proof Так как $\phi$ --- гомоморфизм, то для $\forall \vec{\gamma} = (\gamma_1, ..., \gamma_n) \in V_n(p)$ имеем:
	$$
	\phi(\vec{\gamma}) = \phi \left( \sum_{j=1}^{n} \gamma_j \vec{e}_j \right) = \prod_{j=1}^{n} \phi (\vec{e}_j)^{\gamma_j}.
	$$
	
	$\Rightarrow$ каждый гомоморфизм определяется действием на векторах $\vec{e}_1,..., \vec{e}_n$.
	
	$\Rightarrow ~ \forall j \in {1,...,n}: ~ \phi(\underbrace{\vec{e}_j+...+\vec{e}_j}_p) = \phi(\vec{e}_j)^p = \phi(0) = 1.$
	
	$\Rightarrow$ $\phi(\vec{e}_j)$ --- корень степени $p$ из единицы и $\phi(\vec{e}_j) = \exp\left({\frac{2 \pi i}{p} k}\right)$ для некоторого $k \in {0, ..., p-1}.$
	
	$\Rightarrow$ $\phi$ есть один из $p^n$ гомоморфизмов вида $\omega_{\vec{\alpha}}$ определённых в условиях теоремы. $\blacksquare$
	
	\subsection{Аддитивные характеры}
	Пусть $
	\angles{\vec{\alpha}, \vec{\beta}} = \alpha_1 \beta_1 + ... + \alpha_n \beta_n,
	$ где "<+"> в поле $GF(p)$.
	
	$$
	\omega_{\vec{\alpha}}(\vec{\beta}) = \prod_{j=1}^{n} \left( \exp\left[{\frac{2 \pi i}{p} \alpha_j} \right]\right) ^{\beta_j} = \exp \left( \frac{2 \pi i}{p} (\alpha_1 \beta_1 + ... + \alpha_n \beta_n) \right) = \exp \left( \frac{2 \pi i}{p} \angles{\vec{\alpha}, \vec{\beta}} \right)
	$$
	
	Аддитивная группа поля $GF(p^n)$ может быть представлена как векторное пространство $(V_n(p), +)$.
	
	\defn
	Гомоморфизмы $\omega_{\vec{\alpha}}$ называются аддитивными характерами поля $GF(p^n).$
	
	\subsubsection{Основное свойство характеров}
	Обозначим комплексное сопряжение как $\omega_{\vec{\alpha}}(\vec{e}_j) = \exp\left(-{\frac{2 \pi i}{p} \angles{\vec{\alpha}, \vec{\beta}} }\right).$
	
	\defn $\ind{A} = \left\{ \begin{aligned}
		&1, ~ \text{если условие А справедливо} \\
		&0, ~ \text{если условие А несправедливо}
	\end{aligned} \right.$ называется индикатором.
	
	\statement{$\forall \vec{\alpha}, \vec{\beta} \in V_n(p)$ справедливо равенство $$
		\frac{1}{p^n} \sum_{\vec{\gamma} \in V_n(p)} \omega_{\vec{\alpha}}(\vec{\gamma}) \overline{\omega_{\vec{\beta}}(\vec{\gamma})} = \ind{\vec{\alpha} = \vec{\beta}}.
	$$
} 
\proof
\begin{equation}	
\begin{aligned}
	\frac{1}{p^n} \sum_{\vec{\gamma} \in V_n(p)} \omega_{\vec{\alpha}}(\vec{\gamma}) \overline{\omega_{\vec{\beta}}(\vec{\gamma})} &= \frac{1}{p^n} \sum_{\vec{\gamma} \in V_n(p)} \exp\left({\frac{2 \pi i}{p} \angles{\vec{\alpha}, \vec{\gamma}} }\right) \exp\left(-{\frac{2 \pi i}{p} \angles{\vec{\beta}, \vec{\gamma}} }\right) = \\
	& = \frac{1}{p^n} \sum_{\vec{\gamma} \in V_n(p)} \exp\left(-{\frac{2 \pi i}{p} \angles{\vec{\alpha} - \vec{\beta}, \vec{\gamma}} }\right)  \label{eq:charproof1}
\end{aligned}
\end{equation}

Если $\vec{\alpha} = \vec{\beta}$, то $\angles{\vec{\alpha}-\vec{\beta}, \vec{\gamma}} = 0$ и выражение \refeq{eq:charproof1} равно 1.

Если $\vec{\alpha} \ne \vec{\beta}$, то $\vec{\alpha} - \vec{\beta} = (\theta_1, ..., \theta_n) \ne \vec{0}$. В поле $GF(p)$ уравнение
$$
\theta_1 x_1 +...+ \theta_n x_n = c
$$
для каждого $c \in GF(p)$ будет иметь $p^{n-1}$ решений. Следовательно, если $\vec{\gamma}$ пробегает все значения из $V_n(p)$, то $\angles{\vec{\alpha} - \vec{\beta}, \vec{\gamma}}$ будет принимать каждое значение из $GF(p)$ ровно по $p^{n-1}$ раз. Следовательно
$$
\exp \left(\frac{2 \pi i}{p} \angles{\vec{\alpha} - \vec{\beta}, \vec{\gamma}}\right) = p^{n-1} \sum_{k=0}^{p-1} \exp \left(\frac{2 \pi i}{p}k \right) = p^{n-1} \frac{\exp \left( \frac{2 \pi i}{p} p\right) - 1}{\exp \left( \frac{2 \pi i}{p} \right) - 1}. ~ \blacksquare
$$

	\subsubsection{Базис пространства}
	Обозначим пространство $\mathbb{C}(p,n) = \{f ~|~ V_n(p) \rightarrow \mathbb{C}\}$.
	
	\theorem{$\{\omega_{\vec{\alpha}} ~|~ \vec{\alpha} \in V_n (p)\}$ --- базис пространства $\mathbb{C}(p,n)$.}
	\proof
	Так как число характеров равно размерности $\mathbb{C}(p,n)$, достаточно показать линейную независимость характеров.
	
	Предположим, что они зависимы. Тогда для некоторого набора $\{c_{\vec{\alpha}} ~|~ \vec{\alpha} \in V_n(p)\}$ справедливо равенство
	$$
	\sum_{\vec{\alpha} \in V_n(p)} c_{\vec{\alpha}} \omega_{\vec{\alpha}} (\vec{x}) = 0.
	$$
	
	Для каждого $\vec{\beta} \in V_n(p)$ умножим обе части равенства на $\overline{\omega_{\vec{\beta}}}$ и просуммируем по всем $\vec{x} \in V_n(p)$
	$$
	\sum_{\vec{x} \in V_n(p)} \sum_{\vec{\alpha} \in V_n(p)} c_{\vec{\alpha}} \omega_{\vec{\alpha}} (\vec{x}) \overline{\omega_{\vec{\beta}}(\vec{x})} = 0.
	$$
	
	Поменяв порядок суммирования, имеем:
	$$
	\sum_{\vec{x} \in V_n(p)} \sum_{\vec{\alpha} \in V_n(p)} c_{\vec{\alpha}} \omega_{\vec{\alpha}} (\vec{x}) \overline{\omega_{\vec{\beta}}(\vec{x})} = 
	\sum_{\vec{\alpha} \in V_n(p)} c_{\vec{\alpha}} \sum_{\vec{x} \in V_n(p)} \omega_{\vec{\alpha}} (\vec{x}) \overline{\omega_{\vec{\beta}}(\vec{x})} =
	p^n \sum_{\vec{\alpha} \in V_n(p)} c_{\vec{\alpha}} \ind{\vec{\alpha} = \vec{\beta}} = p^n c_{\vec{\beta}} = 0.
	$$
	Следовательно $c_{\vec{\alpha}} = 0$ для каждого $\vec{\alpha} \in V_n(p)$ и система линейно независима. $\blacksquare$
	
	
	\subsection{Преобразования Фурье}
	\defn
	Разложение произвольной функции $f \in \mathbb{C}(p,n)$ по базису характеров $\{\omega_{\vec{\alpha}} ~|~ \vec{\alpha} \in V_n(p)\}$:
	$$
	f(\vec{x}) = \frac{1}{p^n} \sum_{\vec{\alpha} \in V_n(p)} \tilde{w}_f(\vec{\alpha}) \omega_{\vec{\alpha}}(\vec{x})
	$$
	называется разложением в ряд Фурье.
	
	Комплексное число $\tilde{w}_f(\vec{\alpha})$ называется коэффициентом Фурье, соответствующему набору 
	
	
	\pagebreak
	
	\seminary{Семинар 1}
	\subsection{Коэффициенты Уолша-Адамара и иже с ними}
	
	\defn
	Пусть $f : \mathbb{Z}_2^n \rightarrow \mathbb{R}$, тогда $\hat{f} = (-1)^{f(x)}$ --- действительнозначный аналог булевой функции. \\
	
	
	Обозначим скалярное произведение как $\angles{a,b}$, коэффициент Уолша-Адамара 1-го рода $w_f$, матрицу Адамара как $H$.
	
	\statement{$$
		H \cdot \transp{H} = nE
		$$}
	
	\statement{
		$$
		\begin{aligned}
			f &:~ \mathbb{Z}_2^n \rightarrow \mathbb{Z}_2 \\
			\hat{F}_f (u) &= \sum_{x \in \mathbb{Z}_2^n} (-1)^{\angles{u,x} \oplus f(x)}
		\end{aligned}
		$$}
	
	\seminary{Семинар 2}
	\subsection{Теорема Руппеля}
	
	$$
	P\{f(x) = \angles{u,x}\} = \frac{2^n - d(f(x), \angles{u,x})}{2^n} = 1 - \frac{d(f(x), \angles{u,x})}{2^n} = 1 - \frac{2^n - \hat{F}_f (u)}{2^{n+1}} =
	$$
	$$
	= \frac{1}{2} + \frac{\hat{F}_f (u)}{2^{n+1}} = \vline \begin{matrix}
		\text{}
		
	\end{matrix}
	$$
	
	
	\defn
	Линейный статистический аналог булевой функции $f(x)$ --- это такая линейная функция $\angles{u,x}$, что расстояние $d(f(x), \angles{u,x})$ является минимальным.

	\theorem[Руппеля]{Пусть $u : |\hat{F}_f (n)|$ --- $\max$, тогда $\angles{u,x}\oplus c$ --- линейный статаналог для $f(x)$. Причём 
	$$
	\left\{
	\begin{aligned}
		c & = 0, \text{ если } \hat{F}_f (u) > 0, \\
		c & = 1, \text{ если } \hat{F}_f (u) < 0.
	\end{aligned}
	\right.
	$$}


%	\seminary{Семинар 3}
	
%	\subsection{Бент-функции}




	
	
\end{document}