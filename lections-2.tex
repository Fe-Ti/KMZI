\documentclass[a4paper,12pt]{article}
\usepackage{amsmath,amssymb,amsfonts}
\usepackage{mathtext}
\usepackage[english,russian]{babel}
\usepackage[utf8]{inputenc}
\usepackage[T2A]{fontenc}
\usepackage{graphicx}
\usepackage{textcomp}
\usepackage{geometry}
\geometry{left=3cm}
\geometry{right=1.5cm}
\geometry{top=2cm}
\geometry{bottom=2cm}
\usepackage{tikz}
\usepackage{titling}
\usepackage{indentfirst}
\setlength{\parindent}{1cm}
\usepackage{soul}
\usepackage{enumitem}
\usepackage{listings}
\usepackage{fvextra}
\usepackage{tabularx}
\usepackage{fancyhdr}
\usepackage{setspace}
\usepackage{tikz}
\usepackage{tikz-cd}
\usepackage{ifthen} % provides \isempty test
\usepackage{hyperref}

\hypersetup{
	colorlinks,
	citecolor=black,
	filecolor=black,
	linkcolor=black,
	urlcolor=black
}

%\documentclass{beamer}
%\usepackage{lmodern}
%\usepackage{tikz}
\usetikzlibrary{decorations.pathmorphing}
\tikzset{wavy/.style={decorate, decoration=snake}}

\onehalfspacing
\makeatletter
\AddEnumerateCounter{\asbuk}{\russian@alph}{щ}
\makeatother
\fvset{breaklines=true, breakafter=\space}

\DeclareRobustCommand{\divby}{%
	\mathrel{\vbox{\baselineskip.65ex\lineskiplimit0pt\hbox{.}\hbox{.}\hbox{.}}}%
}

\newcommand{\lection}[1]{\section*{#1}
	\addtocounter{section}{1}
	\addcontentsline{toc}{section}{#1}
	\setcounter{subsection}{0}
}

\newcommand{\seminary}[1]{\section*{#1}
	\addcontentsline{toc}{section}{#1}
}

\newcommand{\defn}{\stepcounter{paragraph}\paragraph{Определение \arabic{section}.\arabic{subsection}.\arabic{paragraph}}
}

\newcommand{\property}[1]{\textbf{#1}.
}

\newcommand{\angles}[1]{\langle #1 \rangle}
\newcommand{\iftext}{\text{если}}
\newcommand{\xor}{\oplus}
%\newcommand{\dist}[2]{d(#1,#2)}
\newcommand{\weight}[1]{\|#1\|}
\newcommand{\transp}[1]{#1^\top}
\newcommand{\ind}[1]{\text{I}(#1)}
\newcommand{\stack}[1]{\begin{matrix} #1 \end{matrix}}

\newcommand{\refeq}[1]{(\ref{#1})}

\newcommand{\statement}[1]{\stepcounter{paragraph}\paragraph{Утв.} #1}
\newcommand{\proof}{\paragraph{Доказательство}}


\newcounter{theorem}
\newcommand{\theorem}[2][]{\stepcounter{theorem}\paragraph{Теорема \arabic{section}.\arabic{subsection}.\arabic{theorem}} \ifthenelse{\equal{#1}{}}{}{(#1)} {\itshape #2 }
}

\fancypagestyle{titlepage}{
	\fancyhf{
		\begin{center}
			\begin{tabularx}{\textwidth}[c]{c p{12cm}}
				\raisebox{-1.1\totalheight}{\includegraphics[width=0.15\linewidth]{./logo/bauman_logo.png}}
				&\begin{center}
					\textbf{ Министерство науки и высшего образования Российской Федерации \\
						Федеральное государственное бюджетное образовательное учреждение 
						высшего образования \\
						«Московский государственный технический университет
						имени Н.Э. Баумана \\
						(национальный исследовательский университет)» \\ (МГТУ им. Н.Э. Баумана)} \\
					
				\end{center}
			\end{tabularx}
	\end{center}}
	\fancyfoot[C]{Москва 2024 г.}
	\renewcommand{\headrulewidth}{0pt}
}

\title{~\\~\\~\\~\\~\\~\\~\\~\\~\\Криптографические методы защиты информации}
\author{Конспект лекций}
\date{МГТУ им. Н.Э. Баумана}

\begin{document}
	\maketitle
	\thispagestyle{titlepage}
	\pagebreak
	\tableofcontents
	
	\section*{Disclaimer}
	\addcontentsline{toc}{section}{Disclaimer}
	Конспект создан студентами для подготовки к экзамену ввиду отсутствия удобного формата лекций. Поэтому он может содержать ошибки, опечатки и многое другое, за что можно получить автомат с сапогами в придачу.
	
	\textit{(Тим Кравченко aka Fe-Ti)}
	

	\pagebreak
	\lection{Лекция 1}
	
	\subsection{Псевдобулева функция}
	Пусть $F$ --- произвольное поле, $p \ge 2$ --- простое число, $V_n(p) = GF(p)^n$ --- $n$-мерное пространство над полем $GF(p)$.
	
	\defn
	Псевдобулевой функцией от $n$ переменных называется произвольное отображение $f: V_n(2) \rightarrow F$. \\
	
	Если $F = GF(2)$, то функция называется булевой (двоичной). $f: V_n(p) \rightarrow F$ --- обобщение псевдобулевой функции.
	
	Обозначим $F(p,n) = \{f ~| ~ f:V_n(p) \rightarrow F\}$ и зададим на нём операции сложения и умножения на элемент поля:
	$$
	\begin{aligned}
		(f_1 + f_2) (x) &= f_1(x) + f_2 (x) \\
		(r \cdot f) (x) &= rf(x), ~ r \in F.
	\end{aligned}
	$$
	
	\statement{Относительно заданных операций $F(p,n)$ является векторным пространством размерности $p^n$ над $F$.}
	\proof
	(Д/з: проверить, что это векторное пространство)
	
	$\left\{ h_\alpha (\vec{x}) = \left[ \begin{aligned}
		1, ~ \iftext ~ \vec{x} = \vec{\alpha} \\
		0, ~ \iftext ~ \vec{x} \ne \vec{\alpha}
	\end{aligned} \right. ~\vline~ \vec{\alpha} \in GF(p)^n \right\}$ --- один из базисов пространства. Линейная независимость очевидна, и любая функция $f \in F(p,n)$:
	$$
	f(\vec{x}) = \sum_{\vec{\alpha} \in GF(p)^n} f(\vec{\alpha}) h_\alpha(\vec{x}). ~\blacksquare
	$$
	
	\subsection{Скалярное произведение}
	\defn 
	Скалярным произведением векторов $\vec{\alpha} = (\alpha_1, ..., \alpha_n)$, $\vec{\beta} = (\beta_1, ..., \beta_n)$, $\vec{\alpha}, \vec{\beta} \in V_n(2)$ называется $$\angles{\vec{\alpha},\vec{\beta}} = \bigoplus_{i=1}^{n} \alpha_i \beta_i.$$
	
	\defn
	Два вектора $\vec{\alpha}, \vec{\beta}$ ортогональны, если $\angles{\vec{\alpha}, \vec{\beta}} = 0$.
	
	\subsection{Аффинные функции}
	\defn
	$\mathcal{AL}_n = \{f \in \mathbb{F}_2 (n) ~|~ \deg f \le 1 \}$ --- множество аффинных функций.
	
	\defn
	$\mathcal{L}_n = \{f \in \mathcal{AL}_n ~|~ f(\vec{0} = 0)\}$ --- множество всех линейных функций.
	
	$$f \in \mathcal{AL}_n ~\Leftrightarrow ~ f(\vec{x}) = \alpha_1 x_1 \oplus ... \oplus \alpha_n x_n \oplus \alpha_0 = \angles{\vec{x}, \vec{\alpha}} \oplus \alpha_0.$$
	
	\paragraph{Задача} Доказать:
	$$
	\begin{aligned}
		|\mathcal{L}_n| &= 2^n \\
		|\mathcal{AL}_n| &= 2^{n+1}
	\end{aligned}
	$$
	
	\subsection{Равновероятная функция}
	Пусть $f = \vec{f} = (f(0), f(1), ..., f(2^n - 1))$ --- вектор столбец булевой функции $f(\vec{x})$, а $\weight{f} = \sum_{\vec{\alpha} \in V_n(2)} f(\vec{\alpha})$ --- вес этой функции.
	
	\defn
	Функция $f \in F_2(n)$ называется равновероятной (сбалансированной, равновесной), если $\weight{f} = 2^{n-1}$.
	
	\paragraph{Задача} Доказать, что если $f \in \mathcal{AL}_n$, то $f$ является равновесной.
	
	\paragraph{Задача} Пусть $f \in \mathbb{F}_2(n), ~ g \in \mathbb{F}_{2}(n+1)$, $g(x_1,...,x_{n+1}) = f(x_1, ...,x_n)$, $x_{n+1}$ --- фиктивная переменная функции $g$. Доказать, что $\weight{g} = 2\cdot \weight{f}$.
	
	\paragraph{Задача} Пусть $f\in \mathbb{F}_2(n), ~ g \in \mathbb{F}_2(n+2)$: $g(x_1,~ ...,~ x_{n+2}) = f(x_1,~ ...,~ x_{n-1},~ x_{n+1} \oplus x_{n+2})$. Известен $\weight{f}$. Найти $\weight{g}$.
	
	\paragraph{Задача} Пусть $f\in \mathbb{F}_2(n), ~ g \in \mathbb{F}_2(n+1)$: $g(x_1,~ ...,~ x_{n+1}) = f(x_1,~ ...,~ x_{n}) \oplus x_{n+1}$. Доказать, что функция $g$ равновероятна.
	
	\paragraph{Задача} Пусть $f \in \mathbb{F}_2(n)$. Доказать:
	$$ \weight{f} \equiv 1 ~ (\text{mod} 2) ~~ \Leftrightarrow ~~ \deg f = n.$$

	\subsection{Производная булевой функции}
	
	\defn
	Производной булевой функции $f \in \mathbb{F}_2(n)$ по направлению $\vec{\delta} \in V_n(2)$ называется булева функция
	$$
	d_{\vec{\delta}} f(\vec{x}) = f(\vec{x} \oplus \vec{\delta}) \oplus f(\vec{x}), ~~ \vec{x} \in V_n (2).
	$$
	
	\defn
	Производной булевой функции $f \in \mathbb{F}_2(n)$ по подпространству $L \le V_n(2)$ называется булева функция
	$$
	d_L f(\vec{x}) = \sum_{\vec{\delta} \in L} f(\vec{x}  \oplus \vec{\delta}), ~~ \vec{x} \in V_n (2).
	$$
	
	\paragraph{Свойство} Пусть $\vec{\delta}_1, ..., \vec{\delta}_k$ --- базис подпространства $L \le V_n(2)$. Тогда $$d_L f = d_{\vec{\delta}_k} ... d_{\vec{\delta}_1} f.$$
	
	\paragraph{Доказать утверждения:}
	\begin{itemize}
		\item $\forall f \in \mathbb{F}_2(n), \forall L \le V_n(2), \forall \vec{\epsilon} \in L:$  $d_L f(\vec{x}) = d_L f (\vec{x} \oplus \vec{\epsilon}).$ 
		
		\item $\forall f,g \in \mathbb{F}_2(n),  \forall L \le V_n(2):$ $d_L(f\oplus g)(\vec{x}) = d_L f(\vec{x}) \oplus d_L g(\vec{x})$.
		
		\item $\forall f \in \mathbb{F}_2(n), \forall \vec{\epsilon}_2, \vec{\epsilon}_2 \in V_n(2):$ $d_{\vec{\epsilon}_2 \oplus \vec{\epsilon}_2} f(\vec{x}) = d_{\vec{\epsilon}_1} f(\vec{x}) \oplus d_{\vec{\epsilon}_1} f(\vec{x} \oplus \vec{\epsilon}_1).$
		
		\item $\forall \vec{\delta} \in V_n(2): d_{\vec{\delta}} f(\vec{x}) = 0 \Leftrightarrow f \text{ --- константа.}$
		
		\item $\forall \vec{\delta} \in V_n(2): d_{\vec{\delta}} f(\vec{x})  \text{ --- константа.} \Leftrightarrow f \text{ --- аффинная функция.}$
	\end{itemize}
	
	\subsection{Гомоморфизмы (флешбэк)}
	\defn Отображение $\phi: (G, \circ) \rightarrow (H, \diamond)$ называется гомоморфизмом, если $\forall x,y \in G: ~ \phi(x \circ y) = \phi(x) \diamond \phi(y)$.
	
	\defn Биективный гомоморфизм называется изоморфизмом.
	
	\subsection{Гомоморфизмы $(V_n(p), +)$ в мультипликативную группу поля$\mathbb{C}$}
	
	Стандартный базис $V_n(p)$:
	$$
		\vec{e}_j = (0,...,0,\underbrace{1}_{j\text{-я позиция}},0,...,0), j = 1, ..., n. \\
	$$
	
	Базис пространства $F(p,n)$, разложение по которому часто используется на практике.
	
	\theorem{
			Множество всех гомоморфизмов $\phi : (V_n(p), +) \rightarrow (\mathbb{C}, \cdot)$ состоит из $p^n$ различных гомоморфизмов $\omega_{\vec{\alpha}}$, $\vec{\alpha} = (\alpha_1, ... , \alpha_n) \in V_n(p)$, каждый из которых однозначно определяется своим действием на векторах стандартного базиса $\vec{e}_j, ~j = 1,...,n,$ условием
			$$
			\omega_{\vec{\alpha}}(\vec{e}_j) = \exp\left({\frac{2 \pi i}{p} \alpha_j}\right),
			$$
			где $i$ --- мнимая единица.
		}
	\proof Так как $\phi$ --- гомоморфизм, то для $\forall \vec{\gamma} = (\gamma_1, ..., \gamma_n) \in V_n(p)$ имеем:
	$$
	\phi(\vec{\gamma}) = \phi \left( \sum_{j=1}^{n} \gamma_j \vec{e}_j \right) = \prod_{j=1}^{n} \phi (\vec{e}_j)^{\gamma_j}.
	$$
	
	$\Rightarrow$ каждый гомоморфизм определяется действием на векторах $\vec{e}_1,..., \vec{e}_n$.
	
	$\Rightarrow ~ \forall j \in {1,...,n}: ~ \phi(\underbrace{\vec{e}_j+...+\vec{e}_j}_p) = \phi(\vec{e}_j)^p = \phi(0) = 1.$
	
	$\Rightarrow$ $\phi(\vec{e}_j)$ --- корень степени $p$ из единицы и $\phi(\vec{e}_j) = \exp\left({\frac{2 \pi i}{p} k}\right)$ для некоторого $k \in {0, ..., p-1}.$
	
	$\Rightarrow$ $\phi$ есть один из $p^n$ гомоморфизмов вида $\omega_{\vec{\alpha}}$ определённых в условиях теоремы. $\blacksquare$
	
	\subsection{Аддитивные характеры}
	Пусть $
	\angles{\vec{\alpha}, \vec{\beta}} = \alpha_1 \beta_1 + ... + \alpha_n \beta_n,
	$ где "<+"> в поле $GF(p)$.
	
	$$
	\omega_{\vec{\alpha}}(\vec{\beta}) = \prod_{j=1}^{n} \left( \exp\left[{\frac{2 \pi i}{p} \alpha_j} \right]\right) ^{\beta_j} = \exp \left( \frac{2 \pi i}{p} (\alpha_1 \beta_1 + ... + \alpha_n \beta_n) \right) = \exp \left( \frac{2 \pi i}{p} \angles{\vec{\alpha}, \vec{\beta}} \right)
	$$
	
	Аддитивная группа поля $GF(p^n)$ может быть представлена как векторное пространство $(V_n(p), +)$.
	
	\defn
	Гомоморфизмы $\omega_{\vec{\alpha}}$ называются аддитивными характерами поля $GF(p^n).$
	
	\subsubsection{Основное свойство характеров}
	Обозначим комплексное сопряжение как $\omega_{\vec{\alpha}}(\vec{e}_j) = \exp\left(-{\frac{2 \pi i}{p} \angles{\vec{\alpha}, \vec{\beta}} }\right).$
	
	\defn $\ind{A} = \left\{ \begin{aligned}
		&1, ~ \text{если условие А справедливо} \\
		&0, ~ \text{если условие А несправедливо}
	\end{aligned} \right.$ называется индикатором.
	
	\statement{$\forall \vec{\alpha}, \vec{\beta} \in V_n(p)$ справедливо равенство $$
		\frac{1}{p^n} \sum_{\vec{\gamma} \in V_n(p)} \omega_{\vec{\alpha}}(\vec{\gamma}) \overline{\omega_{\vec{\beta}}(\vec{\gamma})} = \ind{\vec{\alpha} = \vec{\beta}}.
	$$
} 
\proof
\begin{equation}	
\begin{aligned}
	\frac{1}{p^n} \sum_{\vec{\gamma} \in V_n(p)} \omega_{\vec{\alpha}}(\vec{\gamma}) \overline{\omega_{\vec{\beta}}(\vec{\gamma})} &= \frac{1}{p^n} \sum_{\vec{\gamma} \in V_n(p)} \exp\left({\frac{2 \pi i}{p} \angles{\vec{\alpha}, \vec{\gamma}} }\right) \exp\left(-{\frac{2 \pi i}{p} \angles{\vec{\beta}, \vec{\gamma}} }\right) = \\
	& = \frac{1}{p^n} \sum_{\vec{\gamma} \in V_n(p)} \exp\left(-{\frac{2 \pi i}{p} \angles{\vec{\alpha} - \vec{\beta}, \vec{\gamma}} }\right)  \label{eq:charproof1}
\end{aligned}
\end{equation}

Если $\vec{\alpha} = \vec{\beta}$, то $\angles{\vec{\alpha}-\vec{\beta}, \vec{\gamma}} = 0$ и выражение \refeq{eq:charproof1} равно 1.

Если $\vec{\alpha} \ne \vec{\beta}$, то $\vec{\alpha} - \vec{\beta} = (\theta_1, ..., \theta_n) \ne \vec{0}$. В поле $GF(p)$ уравнение
$$
\theta_1 x_1 +...+ \theta_n x_n = c
$$
для каждого $c \in GF(p)$ будет иметь $p^{n-1}$ решений. Следовательно, если $\vec{\gamma}$ пробегает все значения из $V_n(p)$, то $\angles{\vec{\alpha} - \vec{\beta}, \vec{\gamma}}$ будет принимать каждое значение из $GF(p)$ ровно по $p^{n-1}$ раз. Следовательно
$$
\exp \left(\frac{2 \pi i}{p} \angles{\vec{\alpha} - \vec{\beta}, \vec{\gamma}}\right) = p^{n-1} \sum_{k=0}^{p-1} \exp \left(\frac{2 \pi i}{p}k \right) = p^{n-1} \frac{\exp \left( \frac{2 \pi i}{p} p\right) - 1}{\exp \left( \frac{2 \pi i}{p} \right) - 1}. ~ \blacksquare
$$

	\subsubsection{Базис пространства}
	Обозначим пространство $\mathbb{C}(p,n) = \{f ~|~ V_n(p) \rightarrow \mathbb{C}\}$.
	
	\theorem{$\{\omega_{\vec{\alpha}} ~|~ \vec{\alpha} \in V_n (p)\}$ --- базис пространства $\mathbb{C}(p,n)$.}
	\proof
	Так как число характеров равно размерности $\mathbb{C}(p,n)$, достаточно показать линейную независимость характеров.
	
	Предположим, что они зависимы. Тогда для некоторого набора $\{c_{\vec{\alpha}} ~|~ \vec{\alpha} \in V_n(p)\}$ справедливо равенство
	$$
	\sum_{\vec{\alpha} \in V_n(p)} c_{\vec{\alpha}} \omega_{\vec{\alpha}} (\vec{x}) = 0.
	$$
	
	Для каждого $\vec{\beta} \in V_n(p)$ умножим обе части равенства на $\overline{\omega_{\vec{\beta}}}$ и просуммируем по всем $\vec{x} \in V_n(p)$
	$$
	\sum_{\vec{x} \in V_n(p)} \sum_{\vec{\alpha} \in V_n(p)} c_{\vec{\alpha}} \omega_{\vec{\alpha}} (\vec{x}) \overline{\omega_{\vec{\beta}}(\vec{x})} = 0.
	$$
	
	Поменяв порядок суммирования, имеем:
	$$
	\sum_{\vec{x} \in V_n(p)} \sum_{\vec{\alpha} \in V_n(p)} c_{\vec{\alpha}} \omega_{\vec{\alpha}} (\vec{x}) \overline{\omega_{\vec{\beta}}(\vec{x})} = 
	\sum_{\vec{\alpha} \in V_n(p)} c_{\vec{\alpha}} \sum_{\vec{x} \in V_n(p)} \omega_{\vec{\alpha}} (\vec{x}) \overline{\omega_{\vec{\beta}}(\vec{x})} =
	p^n \sum_{\vec{\alpha} \in V_n(p)} c_{\vec{\alpha}} \ind{\vec{\alpha} = \vec{\beta}} = p^n c_{\vec{\beta}} = 0.
	$$
	Следовательно $c_{\vec{\alpha}} = 0$ для каждого $\vec{\alpha} \in V_n(p)$ и система линейно независима. $\blacksquare$
	
	
	\subsection{Преобразования Фурье}
	\defn
	Разложение произвольной функции $f \in \mathbb{C}(p,n)$ по базису характеров $\{\omega_{\vec{\alpha}} ~|~ \vec{\alpha} \in V_n(p)\}$:
	$$
	f(\vec{x}) = \frac{1}{p^n} \sum_{\vec{\alpha} \in V_n(p)} \tilde{w}_f(\vec{\alpha}) \omega_{\vec{\alpha}}(\vec{x})
	$$
	называется разложением в ряд Фурье.
	
	\defn
	Комплексное число $\tilde{w}_f(\vec{\alpha})$ называется коэффициентом Фурье, соответствующему набору $\vec{\alpha}$.
	
	\defn
	Отображение $\mathbb{C}(p,n) \rightarrow \mathbb{C}^n$, ставящее каждой функции в соответствие набор её коэффициентов Фурье (спектр Фурье), называется преобразованием Фурье. Также есть другие названия: преобразование Уолша-Адамара первого рода, преобразование Уолша (type I Walsh transform).
	
	\statement{Пусть $\vec{\gamma} \in V_n (p)$. Тогда
	$$
	\tilde{w}_f(\vec{\gamma}) = \sum_{\vec{\beta} \in V_n(p)} f(\vec{\beta}) \overline{\omega_{\vec{\gamma}}(\vec{\beta})}
	$$}
	
	\proof
	$$\sum_{\vec{\beta} \in V_n(p)} f(\vec{\beta}) \overline{\omega_{\vec{\gamma}}(\vec{\beta})} = \sum_{\vec{\beta} \in V_n(p)} p^{-n} \sum_{\vec{\alpha} \in V_n(p)} \tilde{w}_f (\vec{\alpha}) \omega_{\vec{\alpha}}(\vec{\beta}) \overline{\omega_{\vec{\gamma}}(\vec{\beta})} = \sum_{\vec{\alpha} \in V_n(p)} \tilde{w}_f (\vec{\alpha}) \ind{\vec{\alpha} = \vec{\gamma}} = \tilde{w}_f(\vec{\gamma}). ~\blacksquare$$
	
	\paragraph{Замечание} Далее $p = 2$.
	
	$$
	\omega_{\vec{\alpha}} (\vec{\beta}) = (-1) ^ {\angles{\vec{\alpha}, \vec{\beta}}}
	$$
	$$
	\Rightarrow ~~ \tilde{w}_f(\vec{\alpha}) = \sum_{\vec{\beta} \in V_n(2)} f (\vec{\beta}) (-1) ^ {\angles{\vec{\alpha}, \vec{\beta}}} .
	$$
	
	\hrule
	\paragraph{Замечание} В литературе иногда встречается и без коэффициента нормировки
	$$
	\tilde{w}_f (\vec{\alpha}) = 2^{-n} \sum_{\vec{\beta} \in V_n(2)} f (\vec{\beta}) (-1) ^ {\angles{\vec{\alpha}, \vec{\beta}}} .
	$$
	В таком случае:
	$$
	f(\vec{x}) = \sum_{\vec{\alpha} \in V_n(2)} \tilde{w}_f (\vec{\alpha}) \omega_{\vec{\alpha}}.
	$$ \hrule ~\\
	
	Соотношение ортогональности:
	$$
	\sum_{\vec{\gamma} \in V_n(2)} \omega_{\vec{\alpha}} (\vec{\gamma}) \omega_{\vec{\beta}} (\vec{\gamma}) = 2^{2n} \ind{\vec{\alpha} = \vec{\beta}}
	$$
	$$
	\sum_{\vec{\gamma} \in V_n(2)} (-1)^{\angles{\vec{\alpha} \oplus \vec{\beta}, \vec{\gamma}}} = 2^{2n} \ind{\vec{\alpha} = \vec{\beta}}
	$$
	
	\pagebreak
	
	\seminary{Семинар 1}
	\subsection{Коэффициенты Уолша-Адамара и иже с ними}
	
	\defn
	Пусть $f : \mathbb{Z}_2^n \rightarrow \mathbb{R}$, тогда $\hat{f} = (-1)^{f(x)}$ --- действительнозначный аналог булевой функции. \\
	
	
	Обозначим скалярное произведение как $\angles{a,b}$, коэффициент Уолша-Адамара 1-го рода $w_f$, матрицу Адамара как $H$.
	
	\statement{$$
		H \cdot \transp{H} = nE
		$$}
	
	\statement{
		$$
		\begin{aligned}
			f &:~ \mathbb{Z}_2^n \rightarrow \mathbb{Z}_2 \\
			\hat{F}_f (u) &= \sum_{x \in \mathbb{Z}_2^n} (-1)^{\angles{u,x} \oplus f(x)}
		\end{aligned}
		$$}
	
	\seminary{Семинар 2}
	\subsection{Теорема Руппеля}
	
	$$
	P\{f(x) = \angles{u,x}\} = \frac{2^n - d(f(x), \angles{u,x})}{2^n} = 1 - \frac{d(f(x), \angles{u,x})}{2^n} = 1 - \frac{2^n - \hat{F}_f (u)}{2^{n+1}} =
	$$
	$$
	= \frac{1}{2} + \frac{\hat{F}_f (u)}{2^{n+1}} =
	\vline \begin{matrix}
		\text{Обозначим} \\
		\frac{\hat{F}_f(u)}{2^n} = \hat{S}_f(u)
	\end{matrix} \vline
	=  \frac{1}{2} + \frac{\hat{S}_f (u)}{2}
	$$
	
	$$
	P\{f(x) = \angles{u, x} \oplus 1\} = \frac{1}{2} - \frac{\hat{S}_f (u)}{2}
	$$
	
	
	\defn
	Линейный статистический аналог булевой функции $f(x)$ --- это такая линейная функция $\angles{u,x}$, что расстояние $d(f(x), \angles{u,x})$ является минимальным.

	\theorem[Руппеля]{Пусть $u : |\hat{F}_f (n)|$ --- $\max$, тогда $\angles{u,x}\oplus c$ --- линейный статаналог для $f(x)$. Причём 
	$$
	\left\{
	\begin{aligned}
		c & = 0, \text{ если } \hat{F}_f (u) > 0, \\
		c & = 1, \text{ если } \hat{F}_f (u) < 0.
	\end{aligned}
	\right.
	$$}


	\seminary{Семинар 3}
	\hl{Семинар про бент-функции}
	
	\seminary{Семинар 4}
	\subsection{Преобразование Мёбиуса}
	\paragraph{Пример и утверждение}
	$f = 10011101$
	$$
	f(\vec{x}= [x_1, x_2, x_3]) = a_0 \xor a_1 x_1 \xor a_2 x_2 \xor a_3 x_3 \xor a_{12} x_1 x_2 \xor a_{13} x_1 x_3 \xor a_{23} x_2 x_3 \xor a_{123} x_1 x_2 x_3
	$$
	$$\begin{aligned}
		a_0 &= 1 = f(000) \\
		a_3 &= 0 \xor a_0 = f(001) \xor a_0 = f(001) \xor f(000) \\
		a_2  = a_{010} &= f(010) \xor f(000) \\
		a_{23} =  a_{011} & = f(011) \xor f(010) \xor f(001) \xor f(000) \\
		...\\
		a_{23} x_2 x_3 &: ~~ x_2 x_3 \Rightarrow x_1^0 x_2^1 x_3^1 \Rightarrow \vec{\sigma} = (011) \\
		\Rightarrow & a_{\vec{\sigma}} = \bigoplus_{\vec{\alpha} \le \vec{\sigma}} f(\vec{\alpha})
	\end{aligned}
	$$
	\proof
	\begin{enumerate}
		\item База индукции: $a_0 \Rightarrow \vec{\sigma} = \vec{0} \Rightarrow x_1^0 ... x_n^0, ~ a_{\vec{\sigma}} = f(\vec{\sigma}) = \bigoplus_{\vec{\alpha} \le \vec{\sigma}} f(\vec{\alpha}) $. Для $a_1 = a_{10...0} = f(10...0)\xor a_0$. Аналогично для $a_2,...,a_n$.
	
	\item Пусть для $\forall \vec{\sigma} \in V_n(2), \weight{\vec{\sigma}} = k: ~~ a_{\vec{\sigma}} = \bigoplus_{\vec{\alpha} \le \vec{\sigma}} f(\vec{\sigma})$.
	
	\hl{На индукционном переходе мы застряли и взялись за другую задачу. Поэтому текст ниже должен расцениваться как потенциально неверный, ибо был дописан из собственных размышлений автора.}
	
	\item Тогда для $\vec{\nu} \in V_n(2), \weight{\vec{\nu}} = k+1$:
	%Пусть $\vec{\mu}: ~ d(\nu, \mu) = 1, \weight{\mu} = k$, $i$ --- номер разряда, где $\mu_i = 0, ~ \nu_i = 1$.
	$$
	f(x_1, ..., x_n) = a_0 \xor a_1 x_1 \xor ... \xor a_n x_n \xor a_{12} x_1 x_2 \xor ... \xor a_{n-1, n} x_{n-1} x_n \xor ... \xor a_{1...n} x_1 ... x_n
	$$
	
	Выразим $a_{i_1...i_{k+1}} x_{i_1} ... x_{i_{k+1}}$, которое соответствует вектору $\vec{\nu}$. Получим следующее:
	$$
	a_{i_1...i_{k+1}} x_{i_1} ... x_{i_{k+1}} =  f(x_1, ..., x_n) \xor a_0 \xor a_1 x_1 \xor ... \xor a_n x_n \xor a_{12} x_1 x_2 \xor ... \xor a_{n-1, n} x_{n-1} x_n \xor ...
	$$
	
	Чтобы получить значение коэффициента подставим в $ x_{i_1} ... x_{i_{k+1}}$ единицы, а в остальные переменные нули. Тогда в правой части останутся конъюнкции содержавшие только $ x_{i_1} ... x_{i_{k+1}}$, коэффициент $a_0$ и функция от вектора $\vec{\nu}$.
	$$
	a_{i_1...i_{k+1}} =  f(x_1, ..., x_n) \xor (a_{i_2 ... i_{k+1}} \xor a_{i_1 i_3 ... i_{k+1}} \xor ... \xor a_{i_1 ... i_k}) \xor (a_{i_3 ... i_{k+1}} \xor ... \xor a_{i_1 ... i_{k-1}}) \xor ... \xor a_0
	$$
	
	Заменим обозначения на бинарные вектора (как в примере выше) и свернём суммирование в скобках:
	$$
	a_{\vec{\nu}} = f(\vec{\nu}) \xor
	\bigoplus_{\vec{\sigma}: {\tiny \stack{\vec{\nu} > \vec{\sigma} \\
				\weight{\vec{\sigma}} = k}} } a_{\vec{\sigma}} \xor
	\bigoplus_{\vec{\sigma}: {\tiny \stack{\vec{\nu} > \vec{\sigma} \\
				\weight{\vec{\sigma}} = k-1}} } a_{\vec{\sigma}} \xor
	...
	\xor
	\bigoplus_{\vec{\sigma}: {\tiny \stack{\vec{\nu} > \vec{\sigma} \\
				\weight{\vec{\sigma}} = 1}} } a_{\vec{\sigma}} \xor
	a_{\vec{0}}
	$$
	
	
\end{enumerate}

	\pagebreak
	\lection{Лекция 2}
	\subsection{Преобразование Фурье (продолжение)}
	
	\paragraph{Задача}
	$\weight{\vec{\alpha} \oplus \vec{\beta}} = \weight{\vec{\alpha}} + \weight{\vec{\beta}} - 2 \weight{\vec{\alpha} \cdot \vec{\beta}}$
	
	\statement{Пусть $f \in \mathbb{F}_2(n)$. Тогда $$\tilde{w}_f (\vec{\alpha}) = \left\{ \begin{aligned}
			&\weight{f}, ~\iftext~ \vec{\alpha} = \vec{0} \\
			&\weight{f(\vec{x}) \oplus \angles{\vec{\alpha}, \vec{x}}} - 2^{n-1}, ~\iftext~ \vec{\alpha} \ne \vec{0}
		\end{aligned} \right.$$}
	
	\proof
	Пусть 
	$$
	\begin{aligned}
	\Omega_0 &= \{\vec{\beta} \in V_n(2) ~|~ f(\vec{\beta}) = 1 \text{ и } \angles{\vec{\alpha}, \vec{\beta}} = 0\}, \\
	\Omega_1 &= \{\vec{\beta} \in V_n(2) ~|~ f(\vec{\beta}) = 1 \text{ и } \angles{\vec{\alpha}, \vec{\beta}} = 1\}.	
	\end{aligned}
	$$
	
	Тогда
	$$
	\tilde{w}_f (\vec{\alpha}) = \sum_{\vec{\beta} \in V_n(2)} f(\vec{\beta}) (-1)^{\angles{\vec{\alpha}, \vec{\beta}}} = \left(\sum_{\vec{\beta} \in \Omega_0} 1 - \sum_{\vec{\beta} \in \Omega_1} \right)
	$$
	Необходимо найти мощности $\Omega_0, \Omega_1$. Из их определения
	$$\begin{aligned}
		|\Omega_0| &= \weight{f(\vec{x}) \cdot (\angles{\vec{\alpha},\vec{x}} \oplus 1)} = 
		\weight{f(\vec{x}) \cdot \angles{\vec{\alpha}, \vec{x}} \oplus f(\vec{x})} = \\ 
		& =\weight{f(\vec{x}) \cdot \angles{\vec{\alpha}, \vec{x}}} + \weight{f(\vec{x})} - 2 \weight{f(\vec{x}) \cdot \angles{\vec{\alpha}, \vec{x}} \cdot f(\vec{x})} = \\
		& = \weight{f(\vec{x})} - \weight{f(\vec{x}) \cdot \angles{\vec{\alpha}, \vec{x}}},
	\end{aligned}
	$$
	$$
	|\Omega_1| = \weight{f(\vec{x}) \cdot \angles{\vec{\alpha}, \vec{x}}}.
	$$
	
	Отсюда следует, что
	$$
	\tilde{w}_f (\vec{\alpha}) = (\weight{f(\vec{x})} - 2\weight{f(\vec{x}) \cdot \angles{\vec{\alpha}, \vec{x}}}) = 
	$$

	\seminary{Семинар 5}
	\paragraph{Летучка}
	\begin{enumerate}
		\item $f(x_1,...,x_8) = x_1 x_3 x_4 + x_2 x_6 + x_5 + x_7 x_8$
		\item $f(x_1,...,x_6) = x_1 x_4 x_5 + x_2 x_3 x_5$
		\item $f(x_1,...,x_10)= x_1 x_2 x_4 x_9 x_{10} + x_1 x_2 x_3 + x_1 x_2 x_5 x_6 x_7 x_8$
	\end{enumerate}
	
	\pagebreak
	
	\lection{Лекция 3}
	\subsection{Фильтрующий генератор}
	\property{При $q = 2$ верхняя граница линейной сложности гаммы фильтрующего генератора равна \[ \sum_{i=0}^{deg f} \binom{n}{i} \]}
	Получение точных нижних оценок -- часто связано со значительными математическими трудностями
	-- Rueppel R. A. Analysis and design of stream ciphers. Berlin: Springer, 1986
	\theorem{Линейная сложность совпадает с размерностью векторного пространства, порождённого множеством всех функций на выходе фильтрующего генератора.}
	\proof (Необходимость)
	Пусть $l$ -- линейная сложность. Тогда $\forall j > l$ функция \[ z_j = z_j(x) = f(\delta^{j-1}(x)) \] над $\mathbb{F}_q$ линейно выражается через первые $l$ функций $z_1, ..., z_l$

	$L = \langle z_1(x), ..., z_l(x) \rangle$ -- линейное пространство на выходе фильтрующего генератора \[ dim L \le l ~~~\blacksquare \]
	\proof (Достаточность)
	Линейная сложность не может быть больше размерности пространства $L$, т.е. \[ dim L \ge l \]. Значит, $l = dim L ~~~\blacksquare$
	\paragraph{Задача} Доказать для каждого $x \in \mathbb{F}_q^n$ равенство \[ z_j(x) = f(\delta^{j-1}(x)) = z_{j-1}(\delta(x)), j = 2,3,... \]
	\paragraph{Задача} Найти линейную сложность фильтрующего генератора над $\mathbb{F}_2$.
	
	$$
	n = 3, ~ \delta: \mathbb{F}_2^3 \rightarrow \mathbb{F}_2^3, ~ f: \mathbb{F}_2^3 \rightarrow \mathbb{F}_2^3, ~~
	\begin{aligned}
		\delta(x_1, x_2, x_3) &= (x_2, x_3, x_1 + x_2),\\
		f(x_1, x_2, x_3) &= x_1x_2 + x_1x_3 + x_2x_3.
	\end{aligned}
	$$
	
	Функция обратной связи ЛРС \[ b(x_1, x_2, x_3) = x_1 + x_2. \]
	\paragraph{Этап 1}
	\begin{enumerate}
	\item Найти характеристический многочлен ЛРС
	\item Установить его примитивность
	\item Найти период
	\end{enumerate}
	\paragraph{Решение}
	\begin{enumerate}
	\item Характеристический многочлен ЛРС -- $x^3 + x + 1$. Примитивен над $\mathbb{F}_2.$
	\item ЛРС максимального периода $2^3 - 1 = 3$
	\item $7$ -- простое $\implies$ период гаммы фильтрующего генератора равен $7$.
	\end{enumerate}
	\paragraph{Этап 2}
	Находим все функции \[ z_j(x) = f(\delta^{j-1}(x)) \] на периоде фильтрующего генератора, применяя равенство \[ z_j(x) = f(\delta^{j-1}(x)) = z_{j-1}(\delta(x)), j = 2,3,... \]
	Получаем
	
	$$
	\begin{aligned}
		z_1 &= f(x) & &= x_1x_2 + x_1x_3 + x_2x_3,\\
		z_2 &= z_1(\delta(x)) &= x_2x_3 + x_2(x_1 + x_2) &= x_1x_2 + x_1x_3 + x_2,\\
		z_3 &= z_2(\delta(x)) &= x_2x_3 + x_2(x_1 + x_2) + x_3 &= x_1x_2 + x_2x_3 + x_2 + x_3,\\
		z_4 &= z_3(\delta(x)) &= x_2x_3 + x_3(x_1 + x_2) + x_1 + x_2 + x_3 &= x_1x_3 + x_1 + x_2 + x_3,\\
		z_5 &= z_4(\delta(x)) &= x_2x_3 + x_4(x_1 + x_2) + x_2 + x_3 + x_1 + x_2 &= x_1x_2 + x_1 + x_2 + x_3,\\
		z_6 &= z_5(\delta(x)) & &= x_2x_3 + x_1 + x_3,\\
		z_6 &= z_6(\delta(x)) & &= x_1x_3 + x_2x_3 + x_1.
	\end{aligned}
	$$
	
	Проверить: $z_8 = z_7(\delta(x)) = z_1$
	\paragraph{Этап 3}
	Найдём максимальное линейное независимое подмножество \[ {z_1, ..., z_l} \subseteq {z_1, ..., z_7}. \]
	Используем общий вид квадратичной функции от 3-х переменных:
	
	$$
	\begin{aligned}
		\phi(0) &= 0,\\
		\phi(x_1, x_2, x_3) &= c_1x_1 + c_2x_2 + c_3x_1x_2 + c_4x_3 + c_5x_1x_3 + c_6x_2x_3.
	\end{aligned}
	$$
	
	Требуется найти $l = rank M$. $i$-я строка $M$ есть коэффициенты многочлена Жегалкина функции $z_i(x), i = 1, ..., 6$
	
	$$
	\begin{pmatrix}
		0 & 0 & 1 & 0 & 1 & 1 \\
		0 & 1 & 1 & 0 & 1 & 0 \\
		0 & 1 & 1 & 1 & 0 & 1 \\
		1 & 1 & 0 & 1 & 1 & 0 \\
		1 & 1 & 1 & 1 & 0 & 0 \\
		1 & 0 & 0 & 1 & 0 & 1 \\
	\end{pmatrix} \sim
	\begin{pmatrix}
		1 & 1 & 1 & 1 & 0 & 0 \\
		1 & 1 & 0 & 1 & 1 & 0 \\
		1 & 0 & 0 & 1 & 0 & 1 \\
		0 & 1 & 1 & 1 & 0 & 1 \\
		0 & 1 & 1 & 0 & 1 & 0 \\
		0 & 0 & 1 & 0 & 1 & 1 \\
	\end{pmatrix} \sim
	\begin{pmatrix}
		1 & 1 & 1 & 1 & 0 & 0 \\
		0 & 1 & 0 & 0 & 1 & 1 \\
		0 & 0 & 1 & 0 & 1 & 0 \\
		0 & 0 & 0 & 1 & 0 & 0 \\
		0 & 0 & 0 & 0 & 1 & 1 \\
		0 & 0 & 0 & 0 & 0 & 1 \\
	\end{pmatrix}
	$$
	
	\paragraph{Задача} Найти представление $z_7$ через $z_1, ..., z_6$

	\subsection{Сильно равновероятные двоичные функции}
	Условие сбалансированности двоичной функции: $f: V_n(2) \rightarrow \mathbb{F}_2$:
	\begin{itemize}
	\item $\forall \gamma \in \mathbb{F}_2$ уравнение \[ f(x_1, ..., x_n) = \gamma \] имеет ровно $2^{n-1}$ решений относительно $(x_1, ..., x_n) \in \mathbb{F}_2^n$
	\end{itemize}

	В ряде методов криптоанализа возникает система уравнений:

	$$
	\begin{aligned}
		f(x_1, ..., x_n) &= \gamma_1,\\
		f(x_2, ..., x_{n+1}) &= \gamma_2,\\
			...\\
		f(x_1, x_{i+1}, ..., x_{n+i-1}) &= \gamma_i,\\
		...\\
		f(x_m, x_{m+1}, ..., x_{n+m-1}) &= \gamma_m.
	\end{aligned}
	$$
	
	$m \ge 1$, $m$ -- число знаков гаммы

	Выходная $m$-грамма $(\gamma_1, ..., \gamma_m) \in V_m(2)$ содержит в себе следующую информацию:
	\begin{itemize}
		\item о функции $f$
		\item о входе $x_1, ..., x_n$\\ -- Данная информация может являться основой для разработки статистических методов криптоанализа.
	\end{itemize}

	\defn Функция $f$ называется
	\begin{itemize}
		\item $m$-равновероятной, если $\forall (\gamma_1, ..., \gamma_m) \in V_m(2)$ система уравнений (1) \[ f(x_i, x_{i+1}, ..., x_{n+i-1}) = \gamma_i, i = 1, ..., m \] имеет ровно $2^{n-1}$ решений относительно неизвестных $x_1, ..., x_{n+m-1}$.
		\item сильно равновероятной, если $f$ является $m$-равновероятной $\forall m \in \mathbb{N}$.\\ -- Сильная равновероятность $f$ означает равновероятность $\forall m \in \mathbb{N}$ выходной $m$-граммы $(\gamma_1, ..., \gamma_m)$, удовлетворяющей системе (1) \[ f(x_i, x_{i+1}, ..., x_{n+i-1}) = \gamma_i, i = 1, ..., m \] и получаемой на случайном равновероятном и независимом входе $x_1, ..., x_{n+m-1}$.
	\end{itemize}

	\defn Вектор $(\gamma_1, ..., \gamma_m) \in V_m(2)$ называется запретом функции $f: V_n(2) \rightarrow \mathbb{F}_2$, если система (1) \[ f(x_i, x_{i+1}, ..., x_{n+i-1}) = \gamma_i, i = 1, ..., m \] не имеет решений.

	\theorem[Сумароков]{Функция $f: V_n(2) \rightarrow \mathbb{F}_2$ не имеет запретов $\Leftrightarrow$ она сильно равновероятна.}

	\paragraph{Задачи}
	\begin{itemize}
		\item Если функция $f: V_n(2) \rightarrow \mathbb{F}_2$ представима в виде \[ f(x_1, ..., x_n) = g(x_1, ..., x_{n-1}) + x_n \] для некоторой функции $g: V_{n-1}(2) \rightarrow \mathbb{F}_2$, то $f$ сильно равновероятна.
		\item Если функция $f: V_n(2) \rightarrow \mathbb{F}_2$ представима в виде \[ f(x_1, ..., x_n) = g(x_1, ..., x_{n-1}) + x_n \] для некоторой функции $g: V_{n-1}(2) \rightarrow \mathbb{F}_2$, то $\forall x_1, ..., x_{n-1} \exists ! x_n, ..., x_{n+m-1}$, при которых справедлива система (1).
		\item Доказать следующие свойства:
			\begin{itemize}
			\item Функции \[ f(x_1, ..., x_n) и g(x_1, ..., x_n) = f(x_n, x_{n-1}, ..., x_1) \] одновременно имеют запреты или не имеют
			\item Функции \[ f(x_1, ..., x_n) и g(x_1, ..., x_n) = f(x_1 \otimes 1, ..., x_n \otimes 1) \] одновременно имеют запреты или не имеют
			\end{itemize}
	\end{itemize}

	\statement{Свойство сильной равновероятности инвариантно относительно действия группы $G$, порождённой следующими преобразованиями на $\mathbb{F}_n$:
		\begin{itemize}
		\item Инверсия функции
		\item Инверсия всех переменных функции
		\item Обратный порядок записи переменных
		\end{itemize}
	}

	\proof На дом
	\theorem{Функция $f: V_n(2) \rightarrow \mathbb{F}_2$ сильно равновероятна $\Leftrightarrow$ она $2^{n-1}$-равновероятна.}
	\theorem{Функция $f: V_n(2) \rightarrow \mathbb{F}_2$ $m$-равновероятна $\Leftrightarrow$ \[ \langle f, b \rangle = b_1f(x_1, ..., x_n) + ... + b_mf(x_m, ..., x_{n+m-1}) \] сбалансирована на каждом ненулевом векторе $b = (b_1, ..., b_m) \in V_m(2)$.}

	Построение сильно равновероятных функций из сильно равновероятных от меньшего числа переменных.
	\theorem{Пусть $h: V_m(2) \rightarrow \mathbb{F}_2, g: V_{n-m+1}(2) \rightarrow \mathbb{F}_2$. Функция $f: V_n(2) \rightarrow \mathbb{F}_2$, \[ f(x_1, ..., x_n) = g(h(x_1, ..., x_m), h(x_2, ..., x_{m+1}), ..., h(x_{n-m+1}, ..., x_n)) \] сильно равновероятна $\Leftrightarrow$ равновероятна каждая из функций $h$, $g$.}
	\paragraph{Пример}
	Перебором вероятностей всех 4-грамм получено, что функция $f: V_3(2) \rightarrow \mathbb{F}_2$ не имеет запрета $\Leftrightarrow$ она линейна по крайней переменной.

	Следующие двоичные функции не имеют запрета:

	$$
	\begin{aligned}
		f_1(x_1, x_2, x_3, x_4) &= x_1 + x_3 + x_2x_4 + x_1x_2x_4,\\
		f_2(x_1, x_2, x_3, x_4) &= x_2 + x_1x_3 + x_1x_3x_4,\\
		f_3(x_1, x_2, x_3, x_4) &= x_2 + x_3 + x_1x_3 + x_3x_4 + x_1x_3x_4, \\
		f_4(x_1, ..., x_5) &= x_1x_2 + x_2x_4 + x_4x_5 + x_3 + x_4 + x_5.
	\end{aligned}
	$$

	\paragraph{Задачи}
	\begin{enumerate}
		\item Среди двоичных функций от 2-х переменных найти все сильно равновероятные.
		\item Дана функция $f: V_3(2) \rightarrow \mathbb{F}_2$, \[ f(x_1, x_2, x_3) = x_1x_2 + x_2x_3 + x_1x_3 + 1 \]
		\begin{itemize}
			\item Найти вероятности всех её 2-грамм.
			\item Является ли она 2-равновероятной?
			\item Является ли она 3-равновероятной?
			\item Является ли она сильно равновероятной?
			\item Какие выходные последовательности для $f$ являются запретными: $(\gamma_1, ..., \gamma_5) \in {(1,1,1,1,1), (0,1,0,0,1), (1,0,1,1,0), (0,1,1,0,1), (1,0,0,1,0)}$ ?
		\end{itemize}
	\end{enumerate}
\end{document}
